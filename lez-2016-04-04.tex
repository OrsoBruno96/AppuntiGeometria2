% Valerio

% brematuro solo a destra in ipotesi di gatto

\newcommand*\Ps{\mathbb{P}} % Projective space
\newcommand*\tc{\ \text{t.c.} \ } % tale che
\newcommand*\dual{{^\ast}} % dual
\newcommand*\base[1][B]{\mathcal{#1}} % base

\titlet{Richiami della scorsa lezione}

Sia $M$ una varietà differenziabile con atlante massimale $\set{(U_i, \phi_i)}$
\marginpar{Più giù Luzio ha già osservato che i cocicli si definiscono nell'intersezione}
e sia $\set{ \fundef[\mu_{ij}]{U_i \cap U_j}{G} }$ un cociclo a valori in un gruppo $G \subseteq \Aut(F)$ (con $F$ varietà liscia). Ripetendo la costruzione usata per realizzare il fibrato tangente $T(M)$ (per cui s'era usato $G = \GL(n, \R)$ e $F=\R^n$, $n = \dim M$), ottengo un fibrato $\Fundef[\pi]{E}{M}$ di fibra $F$ e gruppo strutturale $G$.

\titlet{Altri fibrati su $M$}

Vogliamo innanzitutto definire i \emph{fibrati tensoriali} su $M$.

Si ricorda dalla scorsa lezione che dato $V$ spazio vettoriale su $\R$ di dimensione $n$, chiamando $V\dual$ il suo duale e identificando $V \dual \dual$ con $V$ (in virtù dell'isomorfismo canonico tra di essi), i corrispondenti spazi tensoriali sono definiti come: \[
T_h^k(V) \is V^{\otimes k} \otimes (V \dual)^{\otimes h} \is \Mult( (V\dual)^k \times V^h, \R)  \]
\begin{oss}
	$ T_0^0(V) = \R, \ T_1^0(V) = V \dual, \ T_0^1(V) = V \dual \dual = V $
\end{oss}	
Inoltre, data $\base \is \set{v_1, \dots, v_n}$ base di $V$, posso costruire $\base_h^k$ base di $T_h^k(V)$: sia innanzitutto $\base \dual = \set{v^i}_{i \leq n}$ la base duale di $\base$. Allora: \[
\base_h^k = \base ^{\otimes k} \otimes (\base \dual)^{\otimes h} \is
\set{v^{j_1 \dots j_h}_{i_1 \dots i_k} \is v_{i_1} \otimes \dots \otimes v_{i_k} \otimes v^{j_1} \otimes \dots \otimes v^{j_h} } _{i_l, j_m \leq n} \]
Dunque $\dim T_h^k(V) = n^{k+h}$ e dare una base $\base$ di $V$ stabilisce un isomorfismo (non canonico) tra $T_h^k(V)$ e $T_{h'}^{k'}(V)$ quando $k+h = k'+h'$.

Un $t \in T_h^k(V)$ è detto \emph{tensore di tipo $(k, h)$ su $V$} e si esprime in termini di $\base_h^k$ come: \[
t = t_{j_1 \dots j_h}^{i_1 \dots i_k} \  v^{j_1 \dots j_h}_{i_1 \dots i_k} \]
Dove gli indici ripetuti si intendono sommati, e i $t_{j_1 \dots j_h}^{i_1 \dots i_k} \in \R$ sono le coordinate di $t$ rispetto a $\base_h^k$. Ha dunque senso chiedersi che forma abbiano i cambi di coordinate in $T^k_h(V)$: siano $\base={v_i}$, $\base'={w_i}$ basi di $V$ (e $\base_h^k$, $\base_h^{\prime k}$ le basi corrispondenti di $T^k_h(V)$), e sia $A$ la matrice di cambio di base (in $V$) corrispondente, ovvero $A = \mathcal{M}_{\base}^{\base'}(\id_V)$.

Innanzitutto, per quanto riguarda il cambio di base in $V\dual$, abbiamo \[
\mathcal{M}_{\base' \dual}^{\base \dual}(\id_{V\dual}) = \Big( \mathcal{M}_{\base \dual}^{\base' \dual}(\id_{V\dual}) \Big)^{-1}
= (A^t)^{-1} \]
\begin{oss}
	L'applicazione $A \mapsto (A^t)^{-1}$ è, coefficiente per coefficiente, funzione razionale dei coefficienti. Inoltre, è un automorfismo di $\GL(n, \R)$.
\end{oss}
Sia ora $B = (A^t)^{-1}$, e siano i loro coefficienti $A = (a_{si}), \ B = (b^{rj})$, cosicché $v_i = a_{si} w_s$ e $v^j = b^{rj} w^r$. Allora: 
\begin{gather*}
t =  t_{j_1 \dots j_h}^{i_1 \dots i_k} \ v^{j_1 \dots j_h}_{i_1 \dots i_k} = t_{j_1 \dots j_h}^{i_1 \dots i_k} \ v_{i_1} \otimes \dots \otimes v_{i_k} \otimes v^{j_1} \otimes \dots \otimes v^{j_h} = \\
= t_{j_1 \dots j_h}^{i_1 \dots i_k} \ a_{s_1 i_1}w_{s_1} \otimes \dots \otimes a_{s_k i_k}w_{s_k} \otimes b^{r_1 j_1}w^{r_1} \otimes \dots \otimes b^{r_h j_h}w^{r_h} = \\
= t_{j_1 \dots j_h}^{i_1 \dots i_k} \ a_{s_1 i_1} \dotsm a_{s_k i_k} \ b^{r_1 j_1} \dotsm b^{r_h j_h} \ w_{s_1} \otimes \dots \otimes w_{s_k} \otimes w^{r_1} \otimes \dots \otimes w^{r_h} = \\
= t_{\ r_1 \dots r_h}^{\prime \ s_1 \dots s_k} \ w^{r_1 \dots r_h}_{s_1 \dots s_k}
\end{gather*}

Dunque, indicando i prodotti $a_{s_1 i_1} \dotsm a_{s_k i_k}$ e $b^{r_1 j_1} \dotsm b^{r_h j_h}$ con $a_{i_1 \dots i_k}^{s_1 \dots s_k}$ e $b_{r_1 \dots r_h}^{j_1 \dots j_h}$, otteniamo: \[
t_{\ r_1 \dots r_h}^{\prime \ s_1 \dots s_k} = a_{i_1 \dots i_k}^{s_1 \dots s_k} \ b_{r_1 \dots r_h}^{j_1 \dots j_h} \ t_{j_1 \dots j_h}^{i_1 \dots i_k} \]
Il cambio di coordinate resta dunque una funzione razionale nei coefficienti di $A$.

Ora, sia $\fundef[\rho_h^k]{\GL(n,\R)}{\GL(T_h^k(\R^n))}$ l'applicazione che manda cambi ci coordinate in cambi di coordinate corrispondenti: $A \mapsto A_h^k$, dove $A_h^k$ è la trasformazione lineare $t_{j_1 \dots j_h}^{i_1 \dots i_k} \mapsto a_{i_1 \dots i_k}^{s_1 \dots s_k} \ b_{r_1 \dots r_h}^{j_1 \dots j_h} \ t_{j_1 \dots j_h}^{i_1 \dots i_k}$, ovvero quella che ha coefficienti $(a_{i_1 \dots i_k}^{s_1 \dots s_k} \ b_{r_1 \dots r_h}^{j_1 \dots j_h})$.

Avendo su $M$ il fibrato tangente $T(M)$ definito dal cociclo $\set{ \fundef[\mu_{ij}]{U_i \cap U_j}{\GL(\R^n)} }$, il cociclo $\set{ \fundef[(\mu_h^k)_{ij}]{U_i \cap U_j}{\GL(T_h^k(\R^n))} }$, dove $(\mu_h^k)_{ij} = \rho_h^k \circ \mu_{ij}$, definisce un fibrato $\Fundef[\pi_h^k]{T_h^k(M)}{M}$ di fibra $T_h^k(\R^n)$, e si ha $(\pi_h^k)^{-1}(x) = T_h^k(T_x M)$.

\begin{defn}[Fibrato Tensoriale]
	$T_h^k(M)$ è detto \emph{fibrato tensoriale} di tipo $(k, h)$ su $M$.
	
	In particolare, $T_1^0(M) \si T \dual (M)$ è detto \emph{fibrato cotangente}.
\end{defn}

\begin{oss}
	Un fibrato tensoriale di tipo $(k, h)$ è sempre un fibrato vettoriale di rango $n^{k+h}$.
\end{oss}

Vediamo altri esempi di fibrati, realizzabili a partire dal cociclo $\set{\mu_{ij}}$ del fibrato tangente.

\begin{defn}[Fibrato delle basi]
	Il fibrato principale $\Fundef[\hat{\pi}]{B}{M}$ di gruppo $\GL(n, \R)$ (ovvero quello che ha per fibra $\GL(n, \R)$ stesso) costruito mediante il cociclo $\set{\mu_{ij}}$ è detto \emph{fibrato delle basi} su $M$.
	
	Infatti, $\hat{\pi}^{-1}(x)  \ (\simeq \set{x} \times \GL(n, \R) )$ è l'insieme delle basi di $T_x M \ (\simeq \set{x} \times \R^n)$.
\end{defn}

\begin{oss} 
	$ \fundef[\det]{\GL(n, \R)}{\R^*} $ e $\fundef[\sgn]{\R^*}{\set{\pm 1}}$ sono omomorfismi di gruppi.
\end{oss}

\begin{defn}
	$ \set{  \fundef[\det(\mu_{ij})]{U_i \cap U_j}{\R^*} } $ è un cociclo, detto \emph{cociclo determinante}. Il fibrato risultante di fibra $\R$ è detto \emph{fibrato determinante}.
\end{defn}

\begin{defn}{Cociclo segno}
	$ \set{  \fundef[\sgn(\det(\mu_{ij}))]{U_i \cap U_j}{\set{\pm 1}} } $ è detto \emph{cociclo segno}, e naturalmente vi si può costruire il fibrato principale, con fibra $\set{\pm 1}$, detto \emph{fibrato dei segni}.
\end{defn}
\begin{oss}
	Per $n$ pari, $\Fundef[\pi]{S^n}{\Ps^n}$ è un caso particolare di quest'ultimo tipo di fibrati.
\end{oss}

\titlet{Equivalenza tra fibrati}
Siano $\Fundef[\pi_1]{E_1}{M_1}$ e $\Fundef[\pi_2]{E_2}{M_2}$ fibrati con stessa fibra e stesso gruppo strutturale $G$. Essi sono \emph{isomorfi} se $\exists f, F$ diffeomorfismi tali che il seguente diagramma commuti: \[
\begin{tikzcd}
	E_1 \arrow[r, "F"] \arrow[d, "\pi_1"] & E_2 \arrow[d, "\pi_2"] \\
	M_1 \arrow[r, "f"]  & M_2
\end{tikzcd} \]
e valga $\forall x \in M_1: \ F|_{\pi_1^{-1}(x)} \in G$.

Il prototipo di questo tipo di equivalenza si realizza tra fibrati tangenti di varietà diffeomorfe: \[
\begin{tikzcd}
T(M_1) \arrow[r, "Df"] \arrow[d, "\pi_1"] & T(M_2) \arrow[d, "\pi_2"] \\
M_1 \arrow[r, "f"]  & M_2
\end{tikzcd} \]

Se  $M_1 = M_2 = M$, posso definire una relazione di \emph{equivalenza forte} tra fibrati su $M$ imponendo ulteriormente che sia $f = \id$.
