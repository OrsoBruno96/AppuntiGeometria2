% autore: Candido
\titlet{Fibrato tangente}

\subtitlet{Modello locale}

Si definisce di seguito il fibrato tangente di un aperto di $\R ^n$ e altre nozioni ad esso connesse.
Queste definizioni non saranno generali, poiché la teoria che si intende affrontare non si limita a varietà immerse, esse però costituiranno il modello locale delle strutture più generali quali le varietà differenziabili, come sono state precedentemente definite.

\begin{defn}[Fibrato tangente]
Si definisce fibrato tangente di un aperto $U$ di $\R ^n$ l'insieme $T(U) = U \times \R ^n$. Ogni $x\in U$ è detto \emph{punto} di $U$ mentre ogni $v\in \R ^n$ è detto \emph{vettore} di $\R ^n$.
\end{defn}

Si nota che, fissato un $x$ punto di $U$, la traslazione $v \mapsto v + x$~~$\forall v\in \R ^n$  consente di identificare $\R ^n$ con lo spazio dei vettori tangenti a $U$ nel punto $x$.

% FIGURA di un piano tangente a una superficie, o una sfera

\begin{defn}[Proiezione naturale]
Si definisce \emph{proiezione naturale} $\fundef[\pi _U]{T(U)~}{U}$ la mappa $(x,v) \mapsto x$
\end{defn}

Nella precedente definizione l'aggettivo \emph{naturale} serve solamente a specificare che la mappa cui ci si riferisce è quella indotta dalla definizione come prodotto. In seguito ci si riferirà ad essa anche solo come proiezione.

\begin{defn}[Fibra]
Si definisce fibra la controimmagine di un punto tramite la proiezione naturale, e si scrive $T_x U = \pi ^{-1} _U (x)$
\end{defn}

Alla luce di quest'ultima definizione si identificano:
\begin{itemize}
\item $T_x U = \set{x} \times \R ^n = \setdef[(x,v)\in T(U)]{v \in \R^n}$
\item $T(U) = \djunion_{x \in U} T_x U$
\end{itemize}
Cioè si identifica la fibra come lo spazio tangente del punto $x$, quindi lo spazio dei vettori applicati, mentre l'intero fibrato come unione disgiunta delle fibre.

%FIGURA proiezione delle fibre (rette parallele a y) sui punti, appartenenti a un segmento dell'asse x

\begin{defn}[Applicazione tangente]
Data un'applicazione liscia $\fundef[f]{U}{W}$ si definisce l'\emph{applicazione tangente} di $f$ la funzione $\fundef[Df]{T(U)}{T(W)}$ che manda $(x,v) \mapsto (f(x), \dd_xf(v))$, cioè quell'applicazione che fa commutare il seguente diagramma:
\begin{equation*}
\begin{tikzcd}
	T(U) \arrow[r, dashrightarrow, "Df"] \arrow[d, "\pi _U"]
		& T(W) \arrow[d, "\pi _W"] \\
	U \arrow[r, "f"]
		& W
\end{tikzcd}
\end{equation*}
\end{defn}

\begin{oss}
L'applicazione tangente manda fibre in fibre in modo lineare, poiché $\forall x\in U\ \ \fundef[\dd_xf]{T_x U}{T_x W}$, e il differenziale è un'applicazione lineare per definizione.
L'applicazione tangente gode inoltre delle seguenti proprietà:
\begin{itemize}
\item è un'applicazione liscia in tutte le sue variabili (infatti $f$ è liscia e, $\dd_xf$ è liscia perché lineare)
\item è functoriale: $h = g\circ f \implies Dh = Dg\circ Df$
\item $D(\id) = \id$
\item se $f$ è un diffeomorfismo anche $Df$ è un diffeomorfismo e manda fibra in fibra attraverso isomorfismi lineari
\end{itemize}
\end{oss}

\begin{defn}[Applicazioni fibrate]
Un applicazione che ha la proprietà di $Df$ di mandare fibra in fibra in modo lineare è detta applicazione \emph{fibrata}.
\end{defn}

Se $f$ è un diffeomorfismo si ha che $Df$ è quindi un diffeomorfismo fibrato (come si è già enunciato tra le proprietà di $Df$).

\subtitlet{Caratterizzazione differenziale di $T_x U$}

Di seguito sarà esposta una caratterizzazione più intrinseca delle definizioni che sono state nella sezione precedente.

\begin{defn}[Spazio dei germi]
Dato un punto $x\in U$ aperto di $\R ^n$, e un intorno $W$ di $x$ definisce dunque lo spazio dei germi $\Sge_x$ l'insieme quoziente delle funzioni lisce da un aperto di $W$ in $\R$ con la seguente relazione di equivalenza:
\begin{equation*}
(W_1, f_1) \approx (W_2, f_2) \iff \exists W_3 \subseteq W_1 \cap W_2\ \ \text{t.c.}\ \ f_1(y)=f_2(y)\ \ \forall y\in W_3
\end{equation*}
\end{defn}
\marginpar{di solito usiamo $\sim$ per le equivalenze}

%FIGURA intorni che si intersecano, W_3 annerito

Lo spazio dei germi $\Sge_x$ ha una  naturale struttura di $\R$-algebra, cioè:
\begin{itemize}
\item un $\R$-spazio vettoriale
\item munito di un prodotto (in questo caso commutativo)
\end{itemize}
Per verificarlo è sufficiente mostrare che la somma e il prodotto di due germi è un germe fissato, indipendentemente dai rappresentanti scelti.

\begin{oss}
Si ha che lo spazio $\Sge_x$ non dipende dall'aperto $U$, che costituisce in questo caso la struttura locale, infatti è caratterizzato dalle proprietà locali.
\end{oss}

\begin{defn}[Derivata direzionale]
Dato $v\in T_x U$ la derivata direzionale lungo il vettore $v$ nel punto $x$ è un'applicazione lineare sullo spazio dei germi definita nel modo seguente:
\begin{gather*}
\fundef[\delta_v]{\Sge_x~}{\R} \\
\delta_v([f]) \is \sum_j v^j \left(\pderiv{f}{x^j}(x)\right)
\end{gather*}
Dove i $v^j$ sono le componenti del vettore $v$ nella base canonica di $\R ^n$.
\end{defn}

\begin{oss}[Proprietà della derivata direzionale]
Si verifica facilmente che la derivata direzionale, in un punto $x$ fissato, ha le seguenti proprietà:
\begin{itemize}
\item è ben definita (cioè non dipende dal rappresentante del germe)
\item è $\R$-lineare
\item ha la proprietà di Leibniz:
\begin{equation*}
\delta_v(f\circ g) = f(x)\delta_v(g) + g(x)\delta_v(f)
\end{equation*}
\end{itemize}
Che si riassumono dicendo che $\delta_v$ è una \emph{derivazione} su $\Sge_x$ (questa definizione corrisponde agli ultimi due punti).
\end{oss}

Si ha inoltre che, sempre fissato il punto $x$, è definita la funzione $\fundef[\delta_v]{T_x U}{\Der(\Sge_x)}$ che ad ogni vettore associa una derivazione su $\Sge_x$.

\begin{lemma}
L'applicazione $\delta$ è un isomorfismo di spazi vettoriali
\end{lemma}
\begin{proof}
Si verificano le proprietà di un isomorfismo singolarmente:
\begin{description}
\item[Linearità] è evidente dalla definizione
\item[Iniettività] il nucleo di $\delta$ è evidentemente $\set{0}$, infatti la derivazione nulla è quella in cui $v^j = 0~ \forall j$
\item[Suriettività] si deve verificare che $\forall d\in \Der(\Sge_x)\ \ \exists v\in T_x U\ \text{t.c.}\ d = \delta_v$. Dato $f\in \Sge_x$ sia $h = f -f(x)$. Poiché $f$ è un germe si può restringere a piacere il dominio fino a renderlo convesso. Si applica dunque ad $h$ il lemma di Morse: $h(x) = \sum_j g_j(x)x_j$ con le $g_j$ lisce e $\text{t.c.}\ g_j(x) = \pderiv{f}{x^j}(x)$. Allora si ha:
\begin{equation*}
d(h) = d(f - f(x)) = d(f) - d(f(x)) = d(f) = \sum_jd(x^j)g_j(x) 
\end{equation*}
e quindi $v$ è definito per componenti $v^j = d(x^j)$
\end{description}
\end{proof}

\begin{oss}
Sia $\fundef[\varphi]{U}{W}$ un'applicazione liscia, e $\fundef[f]{W}{\R}$ un'altra applicazione liscia. Allora $\fundef[\delta_v]{\Sge_x}{\R}$ ha la seguente proprietà:
\begin{equation*}
\delta_v(f\circ \varphi) = \delta_{d\varphi (v)}(f)
\end{equation*}
\end{oss}

\subtitlet{Costruzione del fibrato tangente su varietà differenziabili}

Si vogliono ora estendere le definizioni date a generiche varietà differenziabili.

Sia dunque $M$ una n-varietà liscia e $\mathcal{A} = \set{(U_j, \varphi _j)}_j$ un atlante massimale.

\begin{oss}
Dato un cambiamento di carte $\varphi _j \circ \varphi _i ^{-1}$ si ha che esso è una funzione tra aperti di $\R ^n$ ed è perciò definita l'applicazione tangente:
\begin{equation*}
D(\varphi _j \circ \varphi _i ^{-1}): (x,v) \mapsto (\varphi _j \circ \varphi _i ^{-1}(x), \dd_x(\varphi _j \circ \varphi _i ^{-1})(v))
\end{equation*}
\end{oss}

Si considerano ora le applicazioni $\mu_{ji}$ definite come segue:
\begin{gather*}
\fundef[\mu_{ji}]{U_i\cap U_j}{\GL(n)} \\
x \mapsto \dd_{\varphi_j(x)(\varphi _j \circ \varphi _i ^{-1})}
\end{gather*}

\begin{oss}[Proprietà delle $\mu_{ji}$]
Si ha che le funzioni ora definite hanno le seguenti proprietà:
\begin{itemize}
\item ogni $\mu_{ji}$ è liscia
\item $\forall i$ si ha che $\fundef[\mu_{ii}]{U_i}{\GL(n)}$ ha come immagine $\set{\id}$
\item $\mu_{ji} = [\mu_{ji}]^{-1}$
\item $\forall x \in U_i\cap U_j\cap U_k$ si ha che $\mu_{ik}(x)\mu_{kj}(x)\mu_{ji}(x) = \id$
\end{itemize}
Una famiglia di funzioni che verifica le ultime tre proprietà è detta \emph{cociclo}.
\end{oss}

%citazione su cociclo: ultima frase su quaderno
