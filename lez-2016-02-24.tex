% trascrizione: Candido

% lo scrivo in formato lezione, andrà fatto il merge nella stesura finale

% spostata nella lezione del 17 febbraio (dove l'aveva detta sbagliata)
% \begin{oss}
% 	\marginpar{forse questa osservazione va anticipata e convertita in proposizione che dimostra l'implicazione con singoletti non aperti e poi controesempio su $\tau_\text{disc}$}
% 	Sia $X$ l'insieme ambiente e $x\in Y\subseteq X$.
% 	Si ha che:
% 	\[\text{$x$ interno a $Y$} \notimplies \text{$x$ di accumulazione per $Y$}\]
% 	Ad esempio per $(X, \tau_\text{disc})$ preso $Y\is\set{x}$ allora $x$ è interno a $Y$, poiché quest'ultimo è aperto, ma $\set{x}\setminus\set{x} = \nullset$ quindi non è di accumulazione.
% 	Tuttavia, basta supporre che i singoletti non siano aperti affinché invece l'implicazione sia valida.
% \end{oss}

\begin{prop}
	$X$ 1-numerabile e compatto $\implies X$ compatto per successioni
\end{prop}

\begin{proof}
	Sia $(a_n)_{n\in\N}$ una successione a valori in $X$. Si hanno allora due casi:
	\begin{itemize}
		\item se assume un numero finito di valori ne assumerà uno di questi infinite volte, quindi posso estrarre una sottosuccessione costante;
		\item se assume un numero infinito di valori distinti allora per il \autoref{th:infcompacc} si ha che $\exists x\in X$ di accumulazione per $\set{a_n}$.
	\end{itemize}
	Nel secondo caso la sottosuccessione si estrae per $1$-numerabilità, infatti, presa una base di intorni $\set{U_n}$ di $x$ \wlg annidati, costruisco induttivamente la sottosuccessione $(a_{n_k})$ in modo che $a_{n_k}\in U_k$:
	\begin{align*}
		n_0 &\is \min\setdef[m\in\N]{a_m\in U_0} \\
		n_k &\is \min\setdef[m\in\N]{a_m\in U_k\et m>n_{k-1}}
	\end{align*}
	Questa è una buona definizione perché gli $n_k$ sono strettamente crescenti e perché $x$ è di accumulazione per $\set{a_n}$.
	La sottosuccessione converge a $x$ perché abbiamo preso gli intorni annidati.
\end{proof}

\begin{oss}
	Se un punto $x$ è di accumulazione per un dato insieme in uno spazio $T_2$ si ha che ogni intorno di $x$ interseca l'insieme in infiniti punti.
	Infatti, supponendo che l'intersezione sia invece finita, sottraggo questi punti (tranne eventualmente $x$) all'intorno, che rimane un intorno di $x$ perché in uno spazio $T_2$ i singoletti sono chiusi e l'intersezione finita di aperti è aperta, ma non ha punti in comune con l'insieme a parte al più $x$, assurdo.
\end{oss}

\begin{prop}
	$X$ 2-numerabile e compatto per successioni $\implies$ $X$ compatto
\end{prop}

\begin{proof}
	Dato un ricoprimento $\mathcal R$ di $X$, da esso ne devo estrarre uno finito.
	
	Otteniamo prima un risultato intermedio, cioè l'estrazione di un sottoricoprimento numerabile.
	In questo primo passaggio non è coinvolta l'ipotesi di compattezza per successioni.
	Sia dunque $\mathcal B$ una base numerabile di aperti di $X$.
	Allora dalle definizioni di ricoprimento e base abbiamo che:
	\[\forall x\in X \quad
	\exists R_x \in \mathcal R : x\in R_x \quad
	\exists B_x \in \mathcal B : x\in B_x \subseteq R_x\]
	La famiglia dei $B_x$ è numerabile. Se per ogni $B\in\set{B_x}_{x\in X}$ scelgo un solo $R_B\in\setdef[R_x]{B_x=B}$ ho un sottoricoprimento numerabile.
	
	Il secondo passaggio è estrarre da un ricoprimento numerabile un sottoricoprimento finito.
	Per assurdo, se il ricoprimento $(R_n)_{n\in \N}$ non ammette un sottoricoprimento finito, posso costruire una successione tale che $a_n\notin\union_{i=0}^nR_i$ (altrimenti $\set{R_i}_{i\le n}$ ricopre).
	Ma una sottosuccessione convergente di $(a_n)$ sarebbe definitivamente contenuta in un $R_i$.
\end{proof}

\begin{cor}
	Per uno spazio 2-numerabile si ha che: compatto $\iff$ compatto per successioni.
\end{cor}

\subtitlet{Compattezza in spazi metrizzabili}

Se uno spazio metrizzabile è compatto questa proprietà risulta avere conseguenze ``importanti'' su ogni spazio metrico che induce la topologia.

\begin{defn}
	Sia $(X,d)$ uno spazio metrico. Esso si dice \emph{totalmente limitato} se $\forall \varepsilon > 0:X$ è ricoperto da un numero finito di palle di raggio $\varepsilon$.
\end{defn}

\begin{prop}
	Sia $(X,\tau _d)$ compatto per successioni, allora ogni metrica inducente $(X,d)$ è totalmente limitata.
\end{prop}

\begin{proof}
	Per assurdo, supponiamo $\exists \varepsilon > 0$ tale che $X$ non è ricoperto da un numero finito di palle di raggio $\varepsilon$.
	Allora dato $x_0 \in X$ esiste una successione $(x_n)$ tale che
	$x_{n+1}\not\in\union_{i=0}^n\ball[\varepsilon]{x_i}$.
	Questa successione non ammette una sottosuccessione convergente perché i punti distano fra loro almeno $\varepsilon$, da cui l'assurdo.
\end{proof}

\begin{prop}
	$(X, \tau _d)$ compatto per successioni $\implies X$ 2-numerabile
\end{prop}

\begin{proof}
	$\forall n \in \N$ esiste una famiglia finita $\mathcal{F}_n$ di palle aperte di raggio $2^{-n}$ che ricopre $X$, per totale limitatezza.
	Sia $\mathcal{B} \is \union _{n \in \N} \mathcal F_n$.
	Abbiamo che $\mathcal{B}$ è numerabile, per verificare che è anche una base di $\tau _d$ mostriamo prima che se $\mathcal{A}$ è un ricoprimento di $X$ allora esiste un $\bar n$ tale che ogni palla di $\mathcal F_{\bar n}$ è contenuta in uno degli aperti di $\mathcal{A}$.
	Poniamo:
	\[\forall x\in X\quad n_x\is\min\setdef[n\in\N]{\exists A\in\mathcal A:\ball[2^{-n}]x\subseteq A}\]
	È una buona definizione perché le palle $\ball[2^{-n}]x$ sono arbitrariamente piccole.
	Ora invertiamo $n_x$, cioè per ogni $n\in\set{n_x}_{x\in X}$ scegliamo un $x_n$ tale che $n_{x_n}=n$.
	Per assurdo, supponiamo che gli $n_x$ non abbiano massimo: allora $(x_n)$ è una successione e ha una sottosuccessione convergente a un punto $\bar x$.
	Quindi gli $x_n$ sono frequentemente contenuti in $\ball[2^{-(1+n_{\bar x})}]{\bar x}$, e quindi c'è un $m>1+n_{\bar x}$ tale che $n_{x_m}\le 1+n_{\bar x}$ \absurd.
	Il massimo degli $n_x$ è $\bar n$ che cercavamo.
	
	$\forall A \in \tau _d\,\forall x \in A$ consideriamo $\mathcal{A}\is\set{A, X\setminus\set x}$.
	È un ricoprimento perché in uno spazio metrico i singoletti sono chiusi.
	Allora $\ball[2^{-\bar n}]x\subseteq A$ e al variare di $x$ queste palle ricoprono $A$.
\end{proof}

\begin{cor}
	metrizzabile $\implies$ (compatto $\iff$ compatto per successioni)
\end{cor}

\begin{prop}
	Per un insieme in $(\R^n,\tau_E)$ vale che: compatto $\iff$ chiuso e limitato.
\end{prop}

\begin{proof}
	$(\R^n,\tau_E)$ è metrizzabile quindi compatto equivale a compatto per successioni.
	Mostriamo le due implicazioni:
	\begin{description}
		\item[\proofrightarrow]
			Per assurdo:
			\begin{description}
				\item[Limitatezza]
				Se l'insieme è illimitato allora contiene una successione che tende a~$\infty$.
				Le sottosuccessioni di questa tendono necessariamente anch'esse a~$\infty$.
				\item[Chiusura]
				Se l'insieme non è chiuso allora è più piccolo della sua chiusura, quindi c'è un punto di accumulazione $x$ fuori dall'insieme.
				Lo spazio è \mbox{1-numerabile} quindi c'è una successione contenuta nell'insieme che tende a $x$.
				Tutte le sottosuccessioni di questa convergono a $x$.
			\end{description}
		\item[\proofleftarrow]
			È sostanzialmente il teorema di Bolzano-Weierstrass.
			Osserviamo che funziona perché $\R^n$ ha dimensione finita.
		\qedhere
	\end{description}
\end{proof}

\begin{teo}
	$(X,\tau _d)$ compatto $\iff (X,d)$ completo e totalmente limitato
\end{teo}

\begin{proof}
	La dimostrazione è lasciata per esercizio al lettore, ma non è richiesta nel corso.
\end{proof}
