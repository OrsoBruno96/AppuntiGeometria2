% trascrizione: Candido

% lo scrivo in formato lezione, andrà fatto il merge nella stesura finale

\begin{oss}
	\marginpar{forse questa osservazione va anticipata e convertita in proposizione che dimostra l'implicazione con singoletti non aperti e poi controesempio su $\tau_\text{disc}$}
	Sia $X$ l'insieme ambiente e $x\in Y\subseteq X$.
	Si ha che:
	\[\text{$x$ interno a $Y$} \notimplies \text{$x$ di accumulazione per $Y$}\]
	Ad esempio per $(X, \tau_\text{disc})$ preso $Y\is\set{x}$ allora $x$ è interno a $Y$, poiché quest'ultimo è aperto, ma $\set{x}\setminus\set{x} = \nullset$ quindi non è di accumulazione.
	Tuttavia, basta supporre che i singoletti non siano aperti affinché invece l'implicazione sia valida.
\end{oss}

\begin{prop}
	$X$ 1-numerabile e compatto $\implies X$ compatto per successioni
\end{prop}

\begin{proof}
	Sia $(a_n)_{n\in\N}$ una successione a valori in $X$. Si hanno allora due casi:
	\begin{itemize}
		\item se assume un numero finito di valori ne assumerà uno di questi infinite volte, quindi posso estrarre una sottosuccessione costante;
		\item se assume un numero infinito di valori distinti allora per il \autoref{th:infcompacc} si ha che $\exists x\in X$ di accumulazione per $\set{a_n}$.
	\end{itemize}
	Nel secondo caso la sottosuccessione si estrae per $1$-numerabilità, infatti, presa una base di intorni $\set{U_n}$ di $x$ \wlg annidati, costruisco induttivamente la sottosuccessione $(a_{n_k})$ in modo che $a_{n_k}\in U_k$:
	\begin{align*}
		n_0 &\is \min\setdef[m\in\N]{a_m\in U_0} \\
		n_k &\is \min\setdef[m\in\N]{a_m\in U_k\et m>n_{k-1}}
	\end{align*}
	Questa è una buona definizione perché gli $n_k$ sono strettamente crescenti e perché $x$ è di accumulazione per $\set{a_n}$.
	La sottosuccessione converge a $x$ perché abbiamo preso gli intorni annidati.
\end{proof}

\begin{oss}
	Se un punto $x$ è di accumulazione per un dato insieme in uno spazio $T_2$ si ha che ogni intorno di $x$ interseca l'insieme in infiniti punti.
	Infatti, supponendo che l'intersezione sia invece finita, sottraggo questi punti (tranne eventualmente $x$) all'intorno, che rimane un intorno di $x$ perché in uno spazio $T_2$ i singoletti sono chiusi e l'intersezione finita di aperti è aperta, ma non ha punti in comune con l'insieme a parte al più $x$, assurdo.
\end{oss}

\begin{prop}
	$X$ 2-numerabile e compatto per successioni $\implies$ $X$ compatto
\end{prop}

\begin{proof}
	Dato un ricoprimento $\mathcal R$ di $X$, da esso ne devo estrarre uno finito.
	
	Otteniamo prima un risultato intermedio, cioè l'estrazione di un sottoricoprimento numerabile. In questo primo passaggio non è coinvolta l'ipotesi di compattezza per successioni.
	
	Sia dunque $\mathcal B$ una base numerabile di aperti di $X$.
	Allora dalle definizioni di base e ricoprimento abbiamo che:
	\[\forall x\in X \quad
	\exists R_x \in \mathcal R : x\in R_x \quad
	\exists B_x \in \mathcal B : x\in B_x \subseteq R_x\]
	La famiglia dei $B_x$ è numerabile. Se per ogni $B\in\set{B_x}_{x\in X}$ scelgo un solo $R_B\in\setdef[R_x]{B_x=B}$ ho un sottoricoprimento numerabile.
	
	Il secondo passaggio è estrarre da un ricoprimento numerabile un ricoprimento finito.
	
	Per fare ciò si procede per assurdo: se $(R_n)_{n\in \N}$ non ammettesse un sottoricoprimento finito posso costruire la mia successione in modo che $a_0 \not \in R_0$ e $a_1 \not \in R_1 \cup R_2$, \dots . Risulta che da questa successione è impossibile estrarre una sottosuccessione convergente, infatti se convergesse a un punto dell'insieme starebbe definitivamente dentro l'aperto che lo contiene, da cui l'assurdo.
\end{proof}

\begin{cor}
In uno spazio $2$-numerabile si ha che: compatto $\iff$ compatto per successioni
\end{cor}

\titlet{Compattezza in spazi metrizzabili}
Se uno spazio metrizzabile è compatto questa proprietà risulta avere conseguenze ``importanti'' su ogni spazio metrico che induce quella topologia.

\begin{defn}
Sia $(X,d)$ uno spazio metrico. Esso si dice \emph{totalmente limitato} se $\forall \varepsilon > 0:X$ è ricoperto da un numero finito di palle di raggio $\varepsilon$.
\end{defn}

\begin{prop}
Sia $(X,\tau _d)$ compatto per successioni, allora $(X,d)$, una metrica inducente, è totalmente limitato
\end{prop}
\begin{proof}
Per assurdo: $\exists \varepsilon > 0 \text{ t.c. } X$ non è ricoperto da un numero finito di palle di raggio $\varepsilon$.\\
Allora dato $x_0 \in X$ si ha che $\exists x_1 \in X \text{ t.c. } x_1 \not \in \ball[\varepsilon]{x_0}$ e $\exists x_2 \in X \text{ t.c. } x_2 \not \in \ball[\varepsilon]{x_0} \cup \ball[\varepsilon]{x_1}$, \dots .\\
Dunque per induzione costruisco una successione $x_n$ che non ammette nessuna sottosuccessione convergente, infatti i punti distano fra loro almeno $\varepsilon$, da cui l'assurdo.
\end{proof}

\begin{prop}
$(X, \tau _d)$ compatto per successioni $\implies X$ 2-numerabile
\end{prop}
\begin{proof}
$\forall n \in \N$ esiste una famiglia finita $\mathcal{F}_n$ di palle aperte di raggio $2^{-n}$ che ricopre $X$, per totale limitatezza.\\
Sia allora $\mathcal{B} \is \union _{n \in \N} \mathcal F_n$. Si ha che $\mathcal{B}$ è sicuramente una famiglia numerabile, per verificare che è anche una base di $\tau _d$ si verificano i seguenti due punti:
\begin{itemize}
\item Se $\mathcal{A}$ è un ricoprimento di $X \implies \exists \bar{n} \text{ t.c. }$ ogni palla di $\mathcal{F}_n$ è contenuta in uno degli aperti di $\mathcal{A}$;
\item Si applica il punto precedente $\forall A \in \tau _d , \forall x \in A$, considerando come ricoprimento $\mathcal{A} \is \left\{ {A, X\setminus \left\{ {x}\right\}}\right\}$
\end{itemize}
\end{proof}

\begin{cor}
$(X,\tau _d)$ metrizzabile $\implies$ (compatto $\iff$ compatto per successioni)
\end{cor}

\begin{prop}
$(\R^n, \tau _E), K\subset \R^n\text{ compatto} \iff\text{chiuso e limitato}$
\end{prop}
\begin{proof}
	Data l'equivalenza data dal corollario precedente si dimostrano le due implicazioni nel caso di compatto per successioni:
	\begin{description}
		\item[\proofrightarrow]
		Per assurdo posso supporre che l'insieme sia illimitato, ma prendendo una sottosuccessione che tende a $\infty$ si ha che nessuna sottosuccessione estratta può convergere, perché anch'essa tende a $\infty$, mentre negando la chiusura posso costruire una successione che tenda a un punto di accumulazione che non appartiene all'insieme, e per stabilità del limite nel passaggio a sottosuccessioni si ha che nessuna sottosuccessione può convergere a un punto dell'insieme.
		\item[\proofleftarrow]
		Si ottiene eseguendo estrazioni successive mediante il teorema di Bolzano-Weierstrass (compattezza per i chiusi e limitati di $\R$), il che è reso possibile dalla stabilità del limite nel passaggio a sottosuccessioni e dalla dimensione finita di $\R^n$.
	\end{description}
\end{proof}

\begin{teo}
Sia $(X,\tau _d)$ compatto per successioni, e $(X,d)$ una metrica inducente, allora ($(X,\tau _d)$ compatto $\iff (X,d)$ è totalmente limitato e completo)
\end{teo}
\begin{proof}
La dimostrazione è lasciata per esercizio al lettore, ma non è richiesta nel corso
\end{proof}
