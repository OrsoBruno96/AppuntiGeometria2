% questo documento è in ogni sua parte opera originale del sottoscritto
% Alberto Bordin

\titlet{Cenni alle dimostrazioni dei teoremi di trasversalità}

\begin{teo}[Teorema 1 di $\pitchfork$]
Siano $X$, $Y$ e $A\subseteq Y$ varietà differenziabili, ($X$ compatta, $Y$ e $A$ chiuse) e $\fundef[f]{X}{Y}$  tale che $f\pitchfork A$. Allora $Z := f^{-1}(A)$ è una sottovarietà di $X$ e $\codim_X{Z} = \codim_Y{A}$.
\end{teo}

\begin{proof}
Per semplicità supponiamo $\boundary X = \nullset$.\\
\begin{itemize}
\item Caso particolare: $A = \{ y_0 \}$ è composto da un solo punto.

L'idea è che possiamo ``localizzare'' il problema per ricondurci al caso in cui $X = U$ aperto di $\R^n$, $Y = \R^m$, $A = \{ 0 \}$ e applicare il teorema della funzione implicita (versione suriettiva). Quindi, poichè $f\pitchfork A \iff y_0$ è un valore regolare per $f$, abbiamo che $\forall x \in f^{-1}(y_0)$  $\fundef[D_xf]{T_xX}{Y}$ è suriettiva.
\item Riduzione del caso generale al caso particolare.

L'idea è che possiamo localizzare in arrivo: \[(Y,A) = (U \times V, U \times \{ 0 \}) \subseteq (\R^p \times \R^q, \R^p \times \{ 0 \})\]
e (eventualmente restringendo $\pi$ all'immagine di $f$)
\[ \Fundef{X}{\Fundef[\pi]{U \times V}{V}}\]
\begin{center}
  \input{fig11.pdf_tex}
\end{center}
Sia $g = \pi \circ f$, mostrare per esercizio che $f \pitchfork A \iff g \pitchfork \{ 0 \}$.
\end{itemize}

La dimostrazione nel caso $\boundary X \neq \nullset$ è sostanzialmente analoga, con qualche complicazione: l'idea è considerare la varietà \emph{Doppio}
\[D(X) := \quoset{X \sqcup X}{(\fundef[\id]{\boundary X}{\boundary X})}\]
\begin{center}
  \input{fig12.pdf_tex}
\end{center}
\[\fundef[D(f)]{D(X)}{Y \supseteq A}\]
Sarà quindi $D(Z) = (D(f))^{-1}(A)$ una sottovarietà di $D(X)$ e allora $Z = f^{-1}(A) = D(Z) \cap X$ sarà una sottovarietà di $X$. Basta verificare che $D(Z) \pitchfork \boundary X$ in $D(X)$.
\end{proof}

Ora ci prepariamo ad affrontare la dimostrazione del (Teorema 2). Faremo un po' di nomenclatura e poi caleremo dal cielo senza dimostrazione qualche cannone come il teorema di Brown. Infatti se non bariamo un po' non riusciamo ad arrivare in fondo.

\begin{defn} Sia $\fundef{X}{Y}$ una funzione liscia. Ricordiamo che un punto $x \in X$ si dice \emph{critico} per $f$ se $\fundef[D_xf]{T_xX}{T_{f(x)}Y}$ non è suriettiva. Chiameremo $C(f)$ l'insieme dei punti critici di $f$.
\end{defn}

\begin{oss}
$Y \setminus f(C(f))$ è l'insieme dei \emph{valori regolari} di $f$.
\end{oss}

\begin{teo}[Brown]
$Y \setminus f(C(f))$ è denso in $Y$.
\end{teo}

\begin{proof}
È un corollario del Teorema di Morse-Sard\footnote{Baro: \emph{gioca questa carta insieme ad un oggetto, puoi possedere ed usare l'oggetto anche se è contro le regole.}} (cannone di analisi).
\end{proof}

\begin{teo}[Teorema 2 di $\pitchfork$]
$\pitchfork(X,Y,A) \is \setdef[\fundef{X}{Y}]{f \pitchfork A}$ è un aperto denso in $\mathcal{E}(X,Y)$.
\end{teo}

\begin{proof}~
\begin{description}
\item [Apertura]
Idea: l'algebra lineare ci da dei controlli locali (teorema del rango massimo) e la compattezza ci permette di estendere i controlli a livello globale.
\item [Densità]
$X$ è compatto $\implies$ $f(X) \subseteq Y$ è compatto, quindi posso ricoprire $f(X)$ con un numero finito di carte locali di $Y$.

Localmente vedo $Y$ come $\R^n$: $Y = U \subseteq \R^p \times \R^q$ e $A = \R^p \times \{ 0 \}$. Per ogni punto prendo una palla aperta $B$ e la sua controimmagine $K := f^{-1}(B) \subseteq X$; poiché i $K$ ricoprono $X$, che è compatta, allora posso estrarre un sottoricoprimento finito.

Sia $\pitchfork_K(X,Y,A) := \setdef[\fundef{X}{Y}]{f \pitchfork A \text{ lungo } K}$. Dimostriamo che $\forall K$ $\pitchfork_K(X,Y,A)$ è (un aperto) denso. 
\begin{center}
  \input{fig13.pdf_tex}
\end{center}
La condizione di trasversalità si traduce quindi nel richiedere che, dato $f(x) \in A$, $x$ non sia un punto critico per $\pi \circ f$. Per il teorema di Brown esiste $\{ y_n \} \in \R^p \times \R^q$ tale che $y_n \convarrow 0$ e $(\pi \circ f)(y_n)$ è un valore regolare per $\pi \circ f$. Otteniamo $\{ f_n(x) \} = \{ f(x) - y_n \}$ che è una successione di funzioni trasverse ad $A$ e che converge uniformemente a $f$  in $K$.

Abbiamo ottenuto che $\pitchfork_K(X,Y,A)$ è (un aperto) denso, dobbiamo estendere questa conquista a tutto $X$, ma questo non è difficile perché 
\begin{enumerate}
\item per estendere le $f_n$ basta  usare le funzioni a foruncolo;
\item basta notare che $\pitchfork(X,Y,A) = \union_K \pitchfork_K(X,Y,A)$ e poichè l'intersezione è finita ottengo che anche $\pitchfork(X,Y,A)$ è denso.
\end{enumerate}
\end{description}

Per il caso $\boundary X \neq \nullset$ si usa anche qui il trucco del \emph{Doppio}.
\end{proof}

Fra le potenti applicazioni di questi teoremi di trasversalità c'è il teorema di immersione (vedi lezione successiva).

\begin{teo}[Cannone buffo]
Sia $i(n) \is\min\setdef[N]{\forall X^n\,\exists\,\varphi: X \looparrowright \R^N}$, allora $i(n) = n + (\text{numero di ``1'' nella scrittura in base 2 di $n$})$, mentre per gli embedding è ancora un problema aperto.
\end{teo}
