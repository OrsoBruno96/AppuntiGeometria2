%autore: Alessandro Candido

\marginpar{Candido mi pare sia l'unico che usa la terza persona, bisognerebbe uniformarlo ma non mi va}

\titlet{Fibrato tangente}

\subtitlet{Modello locale}

Si definisce di seguito il fibrato tangente di un aperto di $\R ^n$
e altre nozioni ad esso connesse.
Queste definizioni non saranno generali,
poiché la teoria che si intende affrontare non si limita a sottovarietà di $\R^ n$,
esse però costituiranno il modello locale delle strutture più generali sulle varietà.

\begin{defn}[Spazio totale]
	Si definisce \emph{spazio totale del fibrato tangente} di un aperto $U$ di $\R^ n$
	l'insieme $T(U) \is U \times \R^ n$.
	Ogni $x\in U$ è detto \emph{punto} di $U$
	mentre ogni $v\in \R^ n$ è detto \emph{vettore} di $\R ^n$.
\end{defn}

Si nota che, fissato un $x$ punto di $U$, la traslazione
$v \mapsto v + x$~~$\forall v\in \R^ n$
consente di identificare $\R ^n$
con lo spazio dei vettori tangenti a $U$ nel punto $x$.

\begin{figure}
	\centering
	\input{figura6.pdf_tex}
	\caption{Esempio di fibrato tangente.}
\end{figure}

\begin{defn}[Proiezione naturale]
	Si definisce \emph{proiezione naturale}
	$\fundef[\pi _U]{T(U)~}{U}$
	la mappa $(x,v) \mapsto x$.
\end{defn}

Nella precedente definizione
l'aggettivo \emph{naturale} serve solamente a specificare
che la mappa cui ci si riferisce è quella indotta dalla definizione come prodotto.
In seguito ci si riferirà ad essa anche solo come proiezione.

\begin{defn}[Fibrato tangente]
	La coppia $\big(T(U),\pi_U\big)$ è il \emph{fibrato tangente} con \emph{spazio di base} $U$.
\end{defn}

\begin{defn}[Fibra]
	Si definisce fibra la controimmagine di un punto tramite la proiezione naturale,
	e si scrive $T_x U = \pi ^{-1} _U (x)$.
\end{defn}

Alla luce di quest'ultima definizione si identificano:
\begin{itemize}
	\item $T_x U = \set{x} \times \R ^n = \setdef[(x,v)\in T(U)]{v \in \R^n}$
	\item $T(U) = \djunion_{x \in U} T_x U$
\end{itemize}
Cioè si identifica la fibra come lo spazio tangente del punto $x$,
quindi lo spazio dei vettori applicati,
mentre l'intero fibrato come unione disgiunta delle fibre.

\begin{figure}
	\centering
	\input{figura7.pdf_tex}
	\caption{Esempio di proiezione naturale di $U$.}
\end{figure}

\begin{defn}[Applicazione tangente]
	Data un'applicazione liscia tra aperti $\fundef[f]{U}{W}$
	si definisce l'\emph{applicazione tangente} di $f$
	la funzione $\fundef[Df]{T(U)}{T(W)}$
	che manda $(x,v) \mapsto \big(f(x), \de_xf(v)\big)$.
\end{defn}

\begin{oss}
	L'applicazione tangente fa commutare il seguente diagramma:
	\begin{center}
		\begin{tikzcd}
			T(U) \arrow[r, dashrightarrow, "Df"] \arrow[d, "\pi_U"]
				& T(W) \arrow[d, "\pi_W"] \\
			U \arrow[r, "f"]
				& W
		\end{tikzcd}
	\end{center}
\end{oss}

\begin{oss}
	L'applicazione tangente manda fibre in fibre in modo lineare,
	poiché $\forall x\in U\ \ \fundef[\de_xf]{T_x U}{T_x W}$,
	e il differenziale è un'applicazione lineare per definizione.
	L'applicazione tangente gode inoltre delle seguenti proprietà:
	\begin{itemize}
		\item
			è un'applicazione liscia in tutte le sue variabili
			(infatti $f$ è liscia e $\de_xf$ è liscia perché lineare);
		\item è functoriale: $D(g\circ f) = Dg\circ Df$, $D(\id) = \id$;
		\item
			se $f$ è un diffeomorfismo anche $Df$ è un diffeomorfismo
			e manda fibra in fibra attraverso isomorfismi lineari.
	\end{itemize}
\end{oss}

\begin{defn}[Applicazioni fibrate]
	Un'applicazione che ha la proprietà di $Df$ di mandare fibra in fibra in modo lineare
	è detta applicazione \emph{fibrata}.
\end{defn}

Se $f$ è un diffeomorfismo si ha che $Df$ è quindi un diffeomorfismo fibrato
(come si è già enunciato tra le proprietà di $Df$).

\subtitlet{Caratterizzazione differenziale di $T_x U$}

Di seguito sarà esposta una caratterizzazione più intrinseca delle definizioni che sono state nella sezione precedente.

\begin{defn}[Spazio dei germi]
	Dato un punto $x\in U$ aperto di $\R ^n$,
	si definisce \emph{spazio dei germi} $\Sge_x$
	l'insieme delle funzioni lisce da un intorno aperto di $x$ in $U$ a valori in $\R$,
	quozientato con la seguente relazione di equivalenza:
	\begin{equation*}
		(\fundef[f_1]{W_1}\R) \sim (\fundef[f_2]{W_2}\R) \means
		\exists W_3 \subseteq W_1 \cap W_2:
		\begin{cases}
			\text{$W_3$ intorno aperto di $x$} \\
			f_1|_{W_3}=f_2|_{W_3}
		\end{cases}
	\end{equation*}
\end{defn}

\begin{figure}[h]
	\centering
	\input{figura8.pdf_tex}
	\caption{Un germe è definito da ciò che accade in un intorno del punto.}
\end{figure}

Lo spazio dei germi $\Sge_x$ ha una  naturale struttura di $\R$-algebra, cioè:
\begin{itemize}
	\item è un $\R$-spazio vettoriale;
	\item è munito di un prodotto (in questo caso commutativo).
\end{itemize}
Per verificarlo è sufficiente mostrare che
la somma e il prodotto di due germi è un germe fissato,
indipendentemente dai rappresentanti scelti.

\begin{oss}
	Si ha che lo spazio $\Sge_x$ non dipende dall'aperto $U$,
	che costituisce in questo caso la struttura locale,
	infatti è caratterizzato dalle proprietà locali.
	\marginpar{Cosa vuol dire <<che costituisce la struttura locale>>?}
\end{oss}

\begin{defn}[Derivata direzionale]
	Dato $v\in T_x U$ la \emph{derivata direzionale} lungo il vettore $v$ nel punto $x$
	è un'applicazione lineare sullo spazio dei germi definita nel modo seguente:
	\begin{align*}
		\delta_v: \Sge_x &\funarrow \R \\
		[f] &\mapsto \sum_j v^j \pderiv{f}{x^j}(x)
	\end{align*}
	Dove i $v^j$ sono le componenti del vettore $v$ nella base canonica di $\R ^n$.
\end{defn}

\begin{oss}[Proprietà della derivata direzionale]
	Si verifica facilmente che la derivata direzionale, in un punto $x$ fissato, ha le seguenti proprietà:
	\begin{itemize}
		\item è ben definita (cioè non dipende dal rappresentante del germe);
		\item è $\R$-lineare;
		\item
			ha la proprietà di Leibniz:
			\[ \delta_v(f\circ g) = f(x)\delta_v(g) + g(x)\delta_v(f) \]
	\end{itemize}
	Che si riassumono dicendo che $\delta_v$ è una \emph{derivazione} su $\Sge_x$ (questa definizione corrisponde agli ultimi due punti).
\end{oss}

Si ha dunque che, sempre fissato il punto $x$, è definita la funzione $\fundef[\delta]{T_x U}{\Der(\Sge_x)}$ che manda $v\mapsto\delta_v$, dove $\Der(\Sge_x)$ è lo spazio delle derivazioni su $\Sge_x$.

\begin{lemma}
L'applicazione $\delta$ è un isomorfismo di spazi vettoriali.
\end{lemma}
\begin{proof}
Si verificano le proprietà di un isomorfismo singolarmente:
\begin{description}
\item[Linearità] È evidente dalla definizione.
\item[Iniettività] Il nucleo di $\delta$ è $\set{0}$, infatti la derivazione nulla è quella in cui $v^j = 0~ \forall j$.
\item[Suriettività] Si deve verificare che $\forall d\in \Der(\Sge_x)\ \exists v\in T_x U\ \text{t.c.}\ d = \delta_v$. Dato $f\in \Sge_x$ sia $h(y) \is f(y) -f(x)$. Poiché $f$ è un germe si può restringere a piacere il dominio fino a renderlo convesso. Si applica dunque ad $h$ il \autoref{th:lemprec}: $h(y) = \sum_j g_j(y)(y-x)^j$ con le $g_j$ lisce e tali che $g_j(x) = \pderiv{f}{x^j}(x)$. Allora si ha:
\begin{equation*}
d(f) = d(h) = \sum_jg_j(x)d(y\mapsto y^j) 
\end{equation*}
e quindi $v$ è definito per componenti $v^j = d(y\mapsto y^j)$.
\qedhere
\end{description}
\end{proof}

\begin{oss}
Sia $\fundef[\varphi]{U}{W}$ un'applicazione liscia, e $\fundef[f]{W}{\R}$ un'altra applicazione liscia. Allora $\fundef[\delta_v]{\Sge_x}{\R}$ ha la seguente proprietà:
\begin{equation*}
\delta_v(f\circ \varphi) = \delta_{\de_x\varphi (v)}(f)
\end{equation*}
\end{oss}

\subtitlet{Costruzione del fibrato tangente su varietà}

Si vogliono ora estendere le definizioni date a generiche varietà.
Sia dunque $M$ una $n$-varietà con atlante massimale $\set{(U_i, \phi _i)}_{i\in I}$.

\begin{figure}
\centering
\input{figura9.pdf_tex}
\caption{Cambiamento di carte su cui si definisce il cociclo.}
\end{figure}

\begin{oss}
Dato un cambiamento di carte $\phi _j \circ \phi _i ^{-1}$ si ha che esso è una funzione tra aperti di $\R ^n$ ed è perciò definita l'applicazione tangente:
\begin{equation*}
D(\phi _j \circ \phi _i ^{-1}): (x,v) \mapsto \big(\phi _j \circ \phi _i ^{-1}(x), \de_x(\phi _j \circ \phi _i ^{-1})(v)\big)
\end{equation*}
\end{oss}

Si considerano ora le applicazioni $\mu_{ji}$ definite come segue:
\begin{align*}
\mu_{ji} : U_i\cap U_j &\funarrow \GL_n \\
x &\mapsto \de_{\phi_i(x)}(\phi _j \circ \phi _i ^{-1})
\end{align*}

\begin{oss}[Proprietà delle $\mu_{ji}$]
Si ha che le funzioni ora definite hanno le seguenti proprietà:
\begin{itemize}
\item ogni $\mu_{ji}$ è liscia;
\item $\forall i$ si ha che $\fundef[\mu_{ii}]{U_i}{\GL_n}$ ha come immagine $\set{\id}$;
\item $\mu_{ji}(x) = \mu_{ij}(x)^{-1}$;
\item $\forall x \in U_i\cap U_j\cap U_k$ si ha che $\mu_{ij}(x)\mu_{jk}(x)\mu_{ki}(x) = \id$.
\end{itemize}
Una famiglia di funzioni che verifica le ultime tre proprietà è detta \emph{cociclo}.
\end{oss}


\begin{epigraphs}	
	\qitem{Un cociclo è una \emph{colla} che incolla le fibre.}{R.Benedetti}
\end{epigraphs}
