%fede

\titlet{Classificazione spazi vettoriali su $\quoset\Z{2\Z}$}

Daremo una classificazione completa a meno di isometrie degli spazi vettoriali $V$ su campo $\quoset\Z{2\Z}$ di dimensione 
$\dim(V) = n$. Lo spazio vettoriale è dotato di un prodotto scalare non degenere.
%non sono sicuro della parola isometria
Decomporremo lo spazio vettoriale in somme ortogonali (rispetto al prodotto scalare dato) di spazi vettoriali isometrici
a $U$, $H$ di dimensioni rispettivamente 1 e 2. Questi spazi si scrivono in qualche base rispettivamente come: $\begin{bmatrix} 1 \end{bmatrix}$ e $\begin{bsmallmatrix} 0 & 1 \\ 1 & 0 \end{bsmallmatrix}$.
$H$ è totalmente isotropo ma non è identicamente nullo (ricorda che sono in caratteristica 2!).

Reminder: Un prodotto scalare totalmente isotropo è uno per cui $(v,v) = 0,\, \forall v \in V$.

%non mi è chiaro se quetso teorema lo abbiamo dimostrato o lo abbiamo solo detto%
\begin{teo}
 Posso scrivere $V$ in due e due sole forme:
 \begin{itemize}
  \item $V = U \perp U \perp \dots \perp U$.
  $V$ è somma di $n$ copie di $U$ e in questo caso V ha una base ortogonale e $\dim(V) = n$.
  \item $V = H \perp H \perp \dots \perp H$.
  $V$ è somma di $n$ copie di $H$, il prodotto scalare è totalmente isotropo ma non è nullo. $\dim(V) = 2n$.
 \end{itemize}
\end{teo}

Il caso intermedio in cui:  $V = (H \perp H \perp \dots \perp H) \perp (U \perp U \perp \dots \perp U)$ si può eliminare grazie al lemma seguente:

%verifica che isometrico è il termine corretto.
\begin{lemma}
 $U \perp H \simeq 3U$, cioè $U \perp H$ è isometrico a tre copie di $U$.
\end{lemma}

\begin{proof}
 $U \perp H = \left[\begin{smallmatrix} 1 & 0 & 0 \\ 0 & 0 & 1 \\ 0 & 1 & 0 \end{smallmatrix}\right]$. Trovo $W$ di dimesione 2 tale che in una certa base la sua matrice sia:
 $ \left[\begin{smallmatrix} 1 & 0 \\ 0 & 1 \end{smallmatrix}\right]$. Chiamati $u, v, z$ rispettivamente i tre vettori della base in cui ho scritto $U \perp H$ considero $W = \Span(u+v, u+z)$.
 Si verifica che si può scrivere $W$ come somma di due sottospazi vettoriali di dimesione 1 generati da $u+v$ e $u+z$ rispettivamente. Di conseguenza $V = W \perp U$ dove $U$ è uno spazio di dimensione 1  (non degenere per dimensione!)
\end{proof}

Dunque $V$ è caratterizzato totalmente dall'avere una base ortogonale o dall'essere totalmente isotropo rispetto al prodotto scalare associato.

Sia $X$ una superficie compatta, chiusa, connessa. $V = \eta _1(x)$, il prodotto scalre non degenere è la forma di intersezione precedentemente definita. Dato un diffeomorfismo tra due varietà:
$\fundef[f]{S_1}{S_2}$ questo induce un isomorfismo tra gli spazi vettoriali. Questo isomorfismo è in realtà un'isometria poiché conserva la forma di intersezione definita come sopra.
Dunque se due varietà sono diffeomorfe allora gli spazi vettoriali dotati di prodotto scalare $(\eta _1(S_1), \beta_1)$ e $(\eta _1(S_2), \beta_2)$ sono isometrici. In particolare se gli spazi non sono isometrici le varietà non possono essere diffeomorfe.

Per ogni classe di isometria degli spazi vettoriali in caratteristica due esiste  una varietà $X$ che realizza la classe di isometria data tramite lo spazio $(\eta(S), \beta)$.

\subtitlet{$\eta(S^{2})$}

Studio $\eta(S^{2})$, dimostro che esso ha $\dim \eta(S^{2}) = 0$, in particolare contiene una sola classe di equivalenza. Sia $\fundef[f]{S^1}{S^2}$ per quanto detto in precedenza per caratterizzare
gli elementi di $\eta(S^{2})$ è sufficiente considerare le funzioni da una 1-varietà connessa a $S^2$. $f$ è liscia, $S^{2}/f(S^1)$ è aperto denso, dunque $f$ non è surgettiva. Considero 
$\Fundef[f]{S^{1}}{S^{2}}$ e $\Fundef[p]{S^{1}}{\R^n}$, dove $p$ è la proiezione stereografica dunque ho una proiezione $\Fundef[p \circ f]{S^{1}}{\R^n}$. 
La proiezione stereografica deve essere presa rispetto ad un certo punto $P \in f(S^1 0)$. Considero la retrazione su $\R^2$ del laccio proiettato: $\fundef[r_{t}]{x}{tx}$ che manda il laccio nell'origine, questo corrisponde a retrarre
il laccio nel punto $P$. Dunque tutti i lacci sono equvalenti tra di loro.

\begin{figure}
	\centering 
	\input{figura48.pdf_tex}
	\caption{Proiezione Stereografica}
\end{figure}

\subtitlet{$\eta(T_{2}) = H$}

\begin{figure}
	\centering 
	\input{figura49.pdf_tex}
	\caption{Due lacci su un toro}
\end{figure}

Studio $\eta(S^{1} \times S^{1} = T_{2})$, sappiamo che $\eta(T_{2}) = \quoset\Z{2\Z} \times \quoset\Z{2\Z}$. Voglio mostrare che questo spazio è un modello per
$H = \begin{bsmallmatrix} 0 & 1 \\ 1 & 0 \end{bsmallmatrix}$. Considero due lacci sul toro che si intersecano in un punto ma non si autointersecano, questi generano un sottospazio di $\eta$ 
in cui la restrizione di $\beta$ è chiaramante $H$. Voglio dimostrare che $\quoset{\R^2}{\Z^2}$. Apro il toro lungo i due lacci e ottengo un quadrato. 

\begin{figure}
	\centering 
	\input{figura50.pdf_tex}
	\caption{Toro tagliato lungo due lacci}
\end{figure}

Metto sul piano reale la relazione di equivalenza: $(x, y) \sim (x', y') \Leftrightarrow \exists (n,m) \in \Z^2$ tale che $ (x',y')-(x,y) = (n, m)$.

%non so se identificazioni è la parola giusta

Dunque divido il piano in celle (dominio fondamentale del toro) attraverso il reticolo $\Z^2$. ottengo le identificazioni 
$\quoset{\R^2}{\Z^2} \Leftarrow S^{1} \times S^{1} \subseteq \mathbb{C} \times \mathbb{C}$, $(x, y) \Leftarrow (e^{2 \pi i x}, e^{2 \pi i y})$.
Dunque posso srotolare un laccio sul toro come in figura e si vede che ogni laccio è omotopo al segmento tra l'origine e il punto $(p, q) \in \Z^2$, dove $p$ e $q$ indicano il
numero di volte in cui il laccio si avvolge orizzontalmente e verticalmente sul toro. 


\begin{figure}
	\centering 
	\input{figura51.pdf_tex}
	\caption{Sviluppo del dominio fondamentale del toro nel piano xy}
\end{figure}


\begin{teo} 
 $(p, q) =1 \iff $  il laccio non ha autointersezioni
\end{teo}

Se tolgo un punto al toro posso retrarre tutto il laccio nel punto (come si vede nel disegno).

\begin{figure}
	\centering 
	\input{figura52.pdf_tex}
	\caption{Retrazione dei lacci dopo aver tolto un punto al toro}
\end{figure}

Si può rappresentare un laccio sul toro come un ``bucket a due petali'' cioè due circonferenze $S^{1}$ attaccate per un punto che vengono percorse tante volte quanti sono i giri
orizzontali e verticali che il laccio compie.

\begin{figure}
	\centering 
	\input{disegno7.pdf_tex}
	\caption{Bucket a due petali}
\end{figure}

Quello di cui dovrei convincermi è che non è assolutamente chiaro e che due lacci disgiunti che compiono un giro verticale e uno orizzontale generano tutto. Se questo è vero allora $\eta(T_{2})$ 
è una realizzazione di $H$.

\subtitlet{$\eta(\P^{2}) = U$}

Studio $\eta(\P^{2})$, cioè $\eta$ del piano proiettivo. Considero la calotta sferica di $S^2$. I punti non del bordo della calotta rappresentano un solo punto del piano proiettivo mentre
identifico i punti del bordo. La calotta sferica $S^{2}/s$ mi da un dominio fondamentale del proiettivo.

\begin{figure}
	\centering 
	\input{figura53.pdf_tex}
	\caption{Identificazione dei punti antipodali}
\end{figure}

Come si vede nella figura costruisco un $S^1$ identificando i punti antipodali ($\quoset{S^1}=$), si vede che il bordo della calotta ricopre due volte $\quoset{S^1}=$. Il collare che ha per bordo questi due anelli
è chiamato nastro di Möbius e si indica con $\mathcal{M}$. 
L'anima del nostro di Möbius è il proiettivo $\mathbb{P}^1 \subseteq \mathbb{P}^2 $, si vede inoltre che il bordo $\boundary \mathcal{M}$ ha una sola componente connessa.
Sempre dalle figure è chiaro che $\mathbb{P}^2 = \mathcal{M} \cup D^{2}$ incollati lungo il bordo.

\begin{figure}
	\centering 
	\input{figura54.pdf_tex}
	\caption{Proiettivo ottenuto come somma di un disco più un nasto di Moebius}
\end{figure}

Se tolgo un punto al disco e prendo $\quoset{\mathbb P^2}{P}$ ottengo una varietà completamente retraibile nell'anima del nastro di Möbius cioè in $\mathbb{P}^1$.

Un laccio sul natro di Möbius è costretto all'autointersezione, dunque $\mathcal{M}$ è un modello per $U$.

\begin{figure}
	\centering 
	\input{figura56.pdf_tex}
	\caption{Autointersezione necessaria per un nastro di Möbius}
\end{figure}

Ricordiamo che $\eta_1(T_{2}) \sim H$ e $\eta_1(P^{2}) \sim U$

\subtitlet{Modelli per spazi di dimensione maggiore}

Un modello per $H \bot H \bot H ... H$ (somma ortogonale di g spazi vettoriali isomorfi a H) si può ottenere considerando g tori separati. Essi infatti sono g componenti connesse e i lacci su tori diversi non si intersecano. Quindi la matrice associata è diagonale e ogni blocco corriponde a $\begin{bsmallmatrix} 0 & 1 \\ 1 & 0 \end{bsmallmatrix}$.
Per quanto detto vale dunque $eta_{2}(T_{2} \sqcup T_{2} \sqcup T_{2} \sqcup ... \sqcup T_{2} = \quoset{\mathbb{Z}}{2\mathbb{Z}} \times \quoset{\mathbb{Z}}{2\mathbb{Z}} \times \quoset{\mathbb{Z}}{2\mathbb{Z}} \times ... \quoset{\mathbb{Z}}{2\mathbb{Z}}$.

Voglio dimostrare che la relazione sopra scritta vale anche nel caso in cui sostituisca l'unione disgiunta di $g$ tori con la somma connessa, cioè: $eta_{2}(T_{2} # T_{2} # T_{2} # ... # T_{2} = \quoset{\mathbb{Z}}{2\mathbb{Z}} \times \quoset{\mathbb{Z}}{2\mathbb{Z}} \times \quoset{\mathbb{Z}}{2\mathbb{Z}} \times ... \quoset{\mathbb{Z}}{2\mathbb{Z}}$.

Due lacci diversi sulla somma connessa di più tori si intersecano sempre in un due punti quindi li posso accoppiare a due a due e la relazione rimane vera.

\begin{figure}
	\centering 
	\input{figura57.pdf_tex}
	\caption{Intersezione di due lacci su somma connessa di due tori}
\end{figure}

Si può dimostare che prendere la somma connessa rispetto alla somma disgiunta non cambia la classe di cobordismo. 

\begin{figure}
	\centering 
	\input{figura58.pdf_tex}
	\caption{_}
\end{figure}

\begin{figure}
	\centering 
	\input{figura59.pdf_tex}
	\caption{_}
\end{figure}

\begin{figure}
	\centering 
	\input{figura60.pdf_tex}
	\caption{_}
\end{figure}

\begin{figure}
	\centering 
	\input{figura61.pdf_tex}
	\caption{_}
\end{figure}

