%Bob

\begin{teo}[Lemma di omogeneità]
	Sia $Y$ una varietà connessa, allora $\forall \; y_0, y_1 \in Y, \; \exists f$ diffeomorfismo $\fundef{Y}{Y} \; t.c. \; f(y_0)=y_1$.
	\\ Inoltre posso richiedere che $f$ sia a supporto compatto (ovvero è l'identità fuori da un compatto) e che sia isotopa all'identità (ovvero $\exists \fundef[H]{Y\times [0,1]}{Y} \; t.c. \; H_t \; diffeo \; \forall t, \; H_0=Id, \; H_1=f$)
\end{teo}

\begin{proof}
	Partiamo dal caso particolare in cui $Y=\R^n$.
	
	Prendiamo due punti $x$ e $y$ in $\R^n$ e sia $\wlg x\equiv 0$, (l'origine).
	La funzione cercata potrebbe essere $\fundef[f]{\R^n}{\R^n} \; t.c. \; g(x)=x+y$.
	$f(0)=y$ e $f$ è isotopa all'identità ($H(x,t)=x+ty$), tuttavia $f$ non è a supporto compatto, poiché la traslazione muove tutto $\R^n$. Per rimediare sfruttiamo le funzioni a foruncolo precedentemente definite.
	\\ Sia $\fundef[\lambda]{\R^n}{\R}$ la funzione a foruncolo da $\R^n$ a $\R$. La funzione cercata potrà quindi essere scritta come $g(x)=x+\lambda(x)y$

	Passiamo adesso ad una varietà $Y$ connessa qualunque.

	Definiamo una relazione di equivalenza: $x \sim y \leftrightarrow \exists f$ con le proprietà richieste.
	È facile verificare che si tratta di una relazione di equivalenza. A questo punto se dimostro che le classi di equivalenza sono aperte posso concludere: poiché $Y$ è connesso esiste un'unica classe di equivalenza, tutto $Y$.
	Devo mostrare quindi che $\forall x \in Y \; \exists U_x \subseteq Y \; t.c. \; \forall y \in U_x \; x \sim y$.
	
	Considero una carta locale intorno a $x$. Suppongo $\wlg x =0$. All'interno della carta locale posso considerare la funzione $g$ definita prima.
\end{proof}


