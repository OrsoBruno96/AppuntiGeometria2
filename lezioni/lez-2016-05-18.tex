%Bob

\titlet{Somma connessa}

In questo paragrafo preciseremo la nozione di somma connessa che precedentemente abbiamo usato in maniera intuitiva.

Siano $X_1$ e $X_2$ due $n$-varietà compatte, chiuse e connesse. Si delineano due diversi procedimenti per la definizione di somma connessa, andiamo ad analizzarli e confrontarli.
\\

\textbf{Procedimento 1}

Considero per ciascuna varietà un embedding da un disco $D^n$ nella varietà:
$	\fundef[j_1]{D^n}{X_1}; ~\fundef[j_2]{D^n}{X_2}$. Considero quindi gli insiemi così definiti: $W_1 \is X_1 \setminus \mathring{D}_1; ~ W_2 \is X_2 \setminus \mathring{D}_2$ (ovvero le varietà a cui ho tolto la parte interna del disco).

In questa maniera ho ottenuto $\partial W_1 = \partial D_1 \approx S^{n-1}$. In pratica i bordi coincidono e sono diffeomorfi ad una sfera di dimensione n-1.

Fissiamo un diffeomorfismo $\fundef[\phi]{D_1}{D_2}$. Chiamo $\psi \is \phi_{|\partial D_1}$, la restrizione al bordo di $\fundef[\psi]{\partial D_1}{\partial D_2}$.

Considero adesso $(W_1 \sqcup W_2)_{/\psi}$.

\begin{teo}
Se almeno una tra $X_1$ e $X_2$ è non orientabile, allora $(W_1 \sqcup W_2)_{/\psi}$ è univocamente definito a meno di diffeomorfismi.\\
Se entrambe sono orientate e si inverte l'orientazione dei bordi, allora $(W_1 \sqcup W_2)_{/\psi}$ è orientato e univocamente determinato a meno di diffeomorfismi che preservano l'orientazione.
\end{teo}

\begin{defn}
Dal teorema precedente segue che $(W_1 \sqcup W_2)_{/\psi}$ è ben definito, lo chiameremo somma connessa di $W_1$ e $W_2$ e scriveremo: $W_1\csum W_2 \is (W_1 \sqcup W_2)_{/\psi}$
\end{defn}

\begin{proof}
Diamo qualche idea sulla dimostrazione.

Il punto chiave è mostrare che $W_1 \csum W_2$ non dipenda dalla scelta del disco.

TO DO
\end{proof}

\textbf{Procedura 2}

Prendiamo $X_1, X_2, D_1 \subseteq X_1, D_2 \subseteq X_2, W_1, W_2$ con le stesse definizioni date nella Procedura 1.

Sia data $\fundef[\psi]{S^{n-1}}{S^{n-1}}$. Considero $(W_1 \sqcup W_2)_{/\psi}$.
\\

In cosa differiscono le due procedure?\\
Nella prima prendo un diffeomorfismo $\phi$ su tutto il disco e lo restringo al bordo. Nella seconda procedura prendo direttamente un diffeomorfismo $\psi$ sul bordo.\\
Le due procedure coincidono se $\forall \fundef[\psi]{S^{n-1}}{S^{n-1}}$ posso estendere a $\fundef[\phi ]{D^n}{D^n}$, ma quando questa assunzione è vera? Dipende dalla dimensione $n$ delle varietà $X$ di partenza.

\begin{oss}
Immaginiamo di fare lo stesso discorso con gli omomorfismi al posto che con i diffeomorfismi: estendere un omeomorfismo dal bordo a tutta la varietà è banale, basta estendere lungo i raggi.
\marginpar{non mi è particolarmente chiaro, inserire figura trucco di alexander}
In regime topologico è quindi semplice, non lo è altrettanto in regime differenziale.

Quando si passa ai diffeomorsmi l'estensione è possibile in dimensione 2,3,4,5,6. Oltre la dimensione 7 è in generale falso: è possibile ad esempio trovare una $\fundef[\psi]{S^6}{S^6}$ che non si estende alla varietà $X^7$ di cui è bordo. Esistono quindi 7-varietà che sono omomorfe alla sfera ma non diffeomorfe. Tali sfere sono dette sfere esotiche.
\end{oss}

Da questa osservazione segue che per le superfici le due procedure sono equivalenti. (Per $n=2$ la dimostrazione dell'estensibilità può essere affrontata per esercizio). Quindi $X_1 \csum X_2$ è sempre ben definito per le superfici.

\begin{epigraphs}	
	\qitem{\emph{Possiamo dimenticare la storia delle orientazioni, credeteci...}}{\emph{R.Benedetti}}
\end{epigraphs}

\titlet{Relazione tra somma connessa e caratteristica di Eulero}

In questa sezione ci restringeremo al caso delle superfici ($n$=2). Vogliamo capire qual è l'andamento della caratteristica di Eulero rispetto alla somma connessa. $\chi(X_1 \csum X_2) =~?$.
\marginpar{inserire immagine T+T}

