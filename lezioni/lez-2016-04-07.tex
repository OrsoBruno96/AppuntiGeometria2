% trascrizione Bob

%gnegne

\marginpar{DA FINIRE}

Date due varietà $X$ e $Y$ consideriamo $\mathcal E (X,Y)=\{ \fundef{X}{Y}\text{ liscie} \} $ con la topologia definita precedentemente. Restringiamoci al caso $X$ compatta. Consideriamo alcuni sottoinsiemi:
\begin{itemize}
\item le immersioni $\imm(X,Y) =\{\fundef{X}{Y}\text{ immersioni}\}$\\ovvero $\forall x \in X \; \fundef[Df_x]{T_xX}{T_{f(x)}Y}$ è iniettiva \\(la funzione tangente ristretta alla fibra $x$ che manda lo spazio tangente a $x$ nello spazio tangente a $f(x)$ è iniettiva)
\item gli embedding $\emb(X,Y)=\{\fundef{X}{Y}\text{ embedding}\}$\\ovvero, poiché $X$ compatta $f$ è un'immersione iniettiva.
\item i diffeomorfismi $\diff(X,Y)=\{\fundef{X}{Y}\text{ diffeomorfismi}\}$
\end{itemize}

\begin{oss}
Questi insiemi possono anche essere vuoti.
\end{oss}

\begin{teo}
Questi elencati sono sottoinsiemi aperti di $\mathcal E (X,Y)$.
\end{teo}

Moralmente, se una funzione è ``abbastanza vicina'' a un'immersione (o embedding o diffeom.) è essa stessa un'immersione (o embedding o diffeom.).
\begin{proof}
\noindent
\begin{description}
\item [$\imm(X,Y)$] La condizione di essere immersione è una condizione che si verifica sul comportamento di iniettività delle tangenti che è una condizione aperta. (La formalizzazione di questa idea è lasciata per esercizio)
\item [$\emb(X,Y)$] Suppongo $g$ funzione ``vicina'' ad un embedding $f$. Per il punto precedente $g$ un immersione, resta da mostrare che è iniettiva. 

Supponiamo per assurdo che sia falso. Allora potrò costruire una successione $g_n \convarrow f$ e tali che esistano due successioni di punti distinti $x_n$ e $y_n$ tali che $g_n(x_n) = g_n(y_n)$. 
Poiché $X$ compatto, posso estrarre due sottosuccessioni $x_n \convarrow x_0$ e $y_n \convarrow y_0$. Se per assurdo $x_0 \neq y_0$, valuto $f$ in questi punti e dovrei avere $f(x_0)=f(y_0)$, ch'è assurdo poiché $f$ è iniettiva, ma allora $x_0=y_0$ e entrambe le serie convergono allo stesso punto $x_0$.

Leggendo tutto attraverso una carta intorno a $x_0$ posso pensare $x_0$ e le successioni $x_n$ e $y_n$ in $\R^n$.
Poiché $g_n(y_n)-g_n(x_n)=0$, per il teorema del valor medio $\exists z_n \text{ t.c. } \dd_{z_n}g_n[y_n-z_n]=0$.
\marginpar{questa scrittura con le quadre non mi è chiara}
Ho inoltre che $z_n \convarrow x_0$ perché $z_n \in [x_n,y_n]$.

Considero adesso $v_n \is \frac{y_n-x_n}{\left \|y_n - x_n \right \|} \in S^{n-1}$. Poiché $S^n$ è compatto a meno di estrarre una sottosuccessione $v_n \convarrow v_0 \in S^{n-1}$.
Segue che $\dd_{x_0}f[v_0]=0$,
\marginpar{manco qui}
assurdo perché anche il differenziale di $f$ è iniettivo. \absurd
\item [$\diff(X,Y)$] Per i punti precedenti presa $g$ funzione ``vicina'' a $f$ diffeomorfismo, $g$ è un embedding. Mi resta da mostrare la surgettività.

Poiché $g$ è un embedding $g(x) \subseteq Y$ è aperto, inoltre $X$ è compatta, quindi $g(x)$ è anche chiuso (nelle ipotesi in cui lavoriamo compatto $\iff$ chiuso).

Ma se $X$ e $Y$ sono connesse (ipotesi di comodo) allora $g(x) = Y$, ovvero $g$ è diffeomorfismo. In mancanza dell'ipotesi di comodo mi posso restringere alle singole componenti connesse in partenza e in arrivo.
\qedhere
\end{description}
\end{proof}

\titlet{Orientazione}

Siano $\mathcal{B}$ e $\mathcal{B'}$ due basi di $\R^n$, e sia $M^{\mathcal{B}}_{\mathcal{B'}}$ la matrice di cambiamento di base.
Due basi inducono la stessa orientazione se $\det M^{\mathcal{B}}_{\mathcal{B'}}>0$.

\begin{prop}
Questa è una relazione di equivalenza sulle basi di $\R^n$.
\end{prop}
\begin{proof}
Infatti grazie alle proprietà del determinante e Binet:
\begin{itemize}
\item $\det M^{\mathcal{B}}_{\mathcal{B}} = \det \id = 1$
\item $\det M^{\mathcal{B}}_{\mathcal{B''}} = \det ( M^{\mathcal{B}}_{\mathcal{B'}}  M^{\mathcal{B'}}_{\mathcal{B''}})= \det  M^{\mathcal{B}}_{\mathcal{B'}} \cdot \det  M^{\mathcal{B'}}_{\mathcal{B''}}$
\item $\det M^{\mathcal{B}}_{\mathcal{B''}} = \det  M^{\mathcal{B'}}_{\mathcal{B}}$
\qedhere
\end{itemize}
\end{proof}

\noindent Ci sono quindi due classi di equivalenza.

\begin{defn}[Orientazione]
Una \emph{orientazione} di $\R^n$ è una classe di equivalenza per la relazione prima definita.
\end{defn}

Estendiamo questa definizione alle varietà.

\begin{defn}[Atlante orientato]
Sia $X$ una varietà e $A = \{U_j, \phi_j\}$ un atlante di $X$ (non necessariamente il massimale). Diciamo che tale \emph{atlante} è \emph{orientato} se il cociclo $\{\fundef[\det \lambda_{ji}]{U_i \cap U_j}{\R\setminus\{0\}}\}$ è in effetti a valori in $\R^+$ (è lo stesso cociclo usato per definire il fibrato tangente, ristretto però all'atlante).
\end{defn}

\begin{defn}[Atlanti compatibili]
Due atlanti orientati (ammesso che esistano) sono \emph{compatibili} se la loro unione è un atlante orientato.
\end{defn}

\begin{defn}[Orientazione su varietà]
Un'\emph{orientazione} su una varietà è determinata da un atlante orientato massimale. Una varietà è \emph{orientabile} se ammette un'orientazione, \emph{non orientabile} altrimenti.
\end{defn}

\begin{oss}
Se una varietà è orientabile potrebbe avere più di una orientazione.
\end{oss}

\begin{prop}
Se una varietà è connessa e orientabile, allora ha esattamente due orientazioni.
\end{prop}

DA FINIRE


