% questo documento è in ogni sua parte opera originale del sottoscritto
% Alberto Bordin

\newcommand*\base[1][B]{\mathcal{#1}} % base

\titlet{Fibrati}

\begin{defn}[gruppo di Lie] 
Si dice \emph{gruppo di Lie} un gruppo che è anche una varieta liscia 
e dove sia la moltiplicazione che la funzione cha associa ad 
ogni elemento il suo inverso sono applicazioni lisce (esempi: $\GL(n)$ e $G$ finito con $\tau_D$ sono gruppi di Lie).
\end{defn}

\begin{defn}[automorfismi lisci] Data $F$ una varieta liscia sia
$\Aut(F) := (\setdef[\fundef{F}{F}]{ f\text{ diffeomorfismo}}, \circ)$ il gruppo degli automorfismi lisci di $F$.
\end{defn}

Sia $G \subseteq \Aut(F)$ un sottogruppo che agisce su $F$ mediante l'azione
\begin{align*}
		G \times F &\funarrow F \\
		(g,f) &\funarrow g(f)
\end{align*}

\begin{defn}[fibrato]
Per costruire il \emph{fibrato} ripetiamo parola per parola la costruzione di $T(M)$ sostituendo $F$ a $\R^n$ e $G$ a $\GL(n)$ ottenendo un \emph{fibrato} $\fundef[\pi]{E}{M}$ che viene detto di \emph{gruppo strutturale} $G$ e \emph{fibra (tipica)} $F$.
\end{defn}


\begin{oss}
	$E$ è una varietà di dimensione $\dim(E) = \dim(F) + \dim(M)$.
	$\pi$~è liscia ed è localmente un prodotto di fibra $F$ \marginpar{che cazzo vuol dire?} grazie alla banalizzazione locale
\begin{equation*}
\begin{tikzcd}
	\R^n \times F \supseteq (W_i,F) 
		& \arrow[l, "{(\phi, \id)}"] U_i \times F \arrow[r, "\Psi_i"] \arrow[d, "p_i"]
			& \pi^{-1}(U_i) \arrow[d, "\pi"] \\
		& U_i \arrow[r, hookrightarrow]
			& M
\end{tikzcd}
\end{equation*}
\begin{align*}
	\Psi_j \circ \Psi_i^{-1}  : (U_i \cap U_j) \times F &\funarrow (U_i \cap U_j) \times F \\
	(x,f) &\mapsto (x,\mu_{ji}(x)[f])
\end{align*}
\end{oss}

\begin{defn}
	Se per qualche $k$ ho $G = \GL(k,\R)$ e $F = \R^k$ si ottengono i cosiddetti \emph{fibrati vettoriali reali} di rango $k$.\\
	Se invece $F = G$ e ho $R_g$ (la \emph{rappresentazione regolare}, cioè quella definita dall'azione di $G$ su se stesso mediante la moltiplicazione a destra) si ottengono i \emph{fibrati principali} di gruppo strutturale $G$.
\end{defn}

\titlet{Fatti di algebra lineare}

Sia $V$ uno spazio vettoriale di dimensione finita $n$. \\
Sia $V^* := \Hom(V,\R)$ lo \emph{spazio duale} di $V$ e $V^{**} :=(V^*)^* = \Hom(V^*,\R)$ il \emph{biduale}.\\
Sia $\base = \{v_1, \dots, v_n \}$ base di $V$ e $\base = \{ v^1, \dots, v^n \}$ base di $V^*$ dove $v^j(v_i) = \delta^j_{\hphantom{j}i}$.\\
Poiché esiste un isomorfismo canonico
\begin{align*}
	\varphi: V \funarrow\  &V^{**} \\
		     v \mapsto\  &f_v: V^* \funarrow \R \\
		                 &f_v(\psi) = \psi(v)
\end{align*}
d'ora in poi con un piccolo abuso identificheremo $V$ con $V^{**}$.

\begin{defn}[tensori]
	\[ T_h^k(V) \is V^{\otimes k} \otimes (V^*)^{\otimes h} \is \Mult( (V^*)^k \times V^h, \R) \] Un elemento $t \in T_h^k(V)$ è detto \emph{tensore} di tipo $(k,h)$ su $V$.
\end{defn}

\begin{es}
	$ T_0^0(V) = \R, \ T_1^0(V) = V^*, \ T_0^1(V) = V^{**} = V $. \\
	Un prodotto scalare è un $t \in T_2^0(V) = \Bil(V \times V, \R)$ con $t$ simmetrico.
\end{es}

\begin{oss}
	Esiste un isomorfismo canonico
\begin{align*}
	\varphi: \End(V) &\funarrow T_1^1(V) = \Bil(V^* \times V, \R) \\
				  f &\mapsto \big((\psi,v) \mapsto \psi(f(v))\big)
\end{align*}
\end{oss}

\begin{defn}[basi $\base_h^k$] \marginpar{tutto questo va mergiato attentamente con la lezione successiva}
	Abbiamo visto che una base $\base$ determina una base duale $\base^* = \base_1^0$, vorremmo generalizzare questo fatto ad un qualsiasi $(k,h)$ costruendo una base $\base_h^k$ di $T_h^k(V)$. Cominciamo con $T_2^0(V)$. \\
Si definisce in modo canonico l'applicazione bilineare
\begin{align*}
	\Phi : V^* \times V^*  &\funarrow T_2^0(V) \\
			   (\phi,\psi) &\mapsto \big((v,w) \mapsto \phi(v)\psi(w)\big)
\end{align*}
Utilizzeremo la notazione $\Phi(\phi,\psi) = \phi \otimes \psi$ e chiameremo gli elementi del tipo $\phi \otimes \psi$ \emph{tensori decomponibili} di $T_2^0(V)$.

\noindent Si verifica che la funzione $\Phi$ ha le seguenti proprietà:
\begin{itemize}
\item Data una base e la base duale $\base^* = \{ v^1, \dots, v^n \}$ allora $\base_2^0 = \{ v^i \otimes v^j \}_{i,j = 1, \dots, n}$ è una base di $T_2^0(V)$.
\item \emph{(proprietà universale)} Per ogni applicazione bilineare $\fundef[g]{V^* \times V^*}{Z}$ esiste un'unica applicazione lineare $\fundef[G]{T_2^0(V)}{Z}$ tale che $g = G \circ \Phi$. Infatti $G$ è univocamente determinata dalla relazione $G(v^i \otimes v^j) = g(v^i,v^j)$.
\end{itemize}
Lo stesso argomento si può ripetere per $T_0^2(V)$ e in generale per $T_h^k(V)$ ottenendo le basi \[ \base_h^k = \{ v_{i_1} \otimes \dots \otimes v_{i_k} \otimes v^{j_1} \otimes \dots \otimes v^{j_h} \} \]
\end{defn}