%Costa
%Benedetti ha fatto una buona parte di ricapitolazione se non erro a inizio lezione. Non l'ho inclusa.

\begin{defn}[Varietà parallelizzabile]
 Una varietà $X$ si dice \emph{parallelizzabile} se $TX\simeq X\times\R^n$.
\end{defn}
Tutte le varietà sono parallelizzabili?
\begin{oss}
 Se questo fosse vero, allora su ogni varietà $X$ avremmo l'esistanteza di un campo vettoriale mai nullo: infatti se $TX\simeq X\times\R^n$, abbiamo un diffeomorfismo $\phi$ tra i due insiemi.
 Ma su $X\times\R^n$ posso prendere il campo costante $s(x)=e_1$, e poi portarlo indietro tramite $\phi^{-1}$, ottenendo un campo vettoriale su $X$ mai nullo.
\footnote{Il campo non sarà costante in tutte le carte di $TX$, ma lo sarà carta per carta. Abbiamo però che non si annullerà mai, poiché essendo $\phi$ un diffeomorfismo non può accadere in nessuna presentazione locale che mandi un vettore non nullo in un vettore nullo, dato che il differenziale sarà invertibile.}

\end{oss}
Analizziamo intanto i casi delle sfere $S^n$.
\begin{es}[$S^1$]
 E' facile mostrare che esiste un campo di vettori mai nullo: basta prendere quello tangente in verso antiorario e modulo costante.
 \marginpar{Disegno mulinello}
\end{es}
\begin{es}[$S^2$]
 Mostro che esiste un campo vettoriale che si annulla solo sul polo nord e sul polo sud.
 \marginpar{Non ho capito quale è tale mappa, anche se una idea ce l'avrei...}
 \marginpar{Penso che sia lo stesso di $S^1$ sulle sezioni orizzontali, smorzato mano mano che si allontana dall'equatore e si avvicina ai poli}
 \marginpar{Ricicla vecchi disegnini con sfera e piano tangente}
 Dunque ho due zeri isolati. Per calcolare $\chi(S^2)$ devo dunque sommare $i_N(s)$, $i_S(s)$ con $N= (0,0,1)$, $S=(0,0,-1)$.
 \begin{oss} Per il lemma di Morse, dato che $N$, $S$ sono punti critici della mappa $t$, esiste un sistema di coordinate in cui la funzione è quadratica. 
 In particolare, esiste intorno di $N$ in cui la funzione è nella forma $-(y_1^2+y_2^2)$ e un intorno di $S$ in cui è nella forma $y_1^2+y_2^2$.
 \end{oss}
 Segue che $i(N)=i(S)=1$ (ho due autovalori negativi o due positivi).
 Dunque $\chi(S^2)=2$.
 \marginpar{Disegnino con pozzo e sorgente di linee di campo}
\end{es}
\begin{prop}
 $\chi(S^{2n+1})=0$ e $\chi(S^{2n})=2$.
 \end{prop}
 \begin{proof}
  Basta considerare un campo analogo a quello esibito per $S^{2}$ e concludere tramite Morse.
 \end{proof}

\begin{cor}
 Per $n$ pari non è possibile che esista un campo mai nullo
 ("la sfera non è pettinabile").
\end{cor}
\begin{prop}
 Ci chiediamo se effettivamente per le sfere $S^{2n+1}$ esista un campo di vettori mai nullo. La risposta è affermativa.
\end{prop}
\begin{proof}
La risposta è affermativa: $S^{2n+1}\subseteq\R^{2n+2}$. Considero la mappa 
\begin{align*}
 f : S^{2n+2} &\funarrow {TS^{2n+2}} \\
 (x_1, y_1,\dots, x_{n+1}, y_{n+1}) &\mapsto \big((x_1, y_1,\dots, x_{n+1}, y_{n+1}), (-x_1, y_1,\dots, -x_{n+1}, y_{n+1})\big)
\end{align*}


\marginpar{La notazione usata è forse non corretta: volevo scrivere in modo formale il fatto che la mappa associasse ad un punto un vettore (un elemento del tangente). La componente ``vettoriale'' è solo la seconda dopo il prodotto cartesiano \\
\emph{Risposta: secondo me la notazione formalmente corretta è questa con le parentesi}}
\end{proof}
\begin{oss}
 Questo fatto è vero più in generale: $X$ varietà, $\chi(X)=0\implies \exists$~campo vettoriale mai nullo.
\end{oss}

