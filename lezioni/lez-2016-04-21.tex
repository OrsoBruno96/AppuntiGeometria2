% Valerio

% guardi che signora mia qua non si sa mica eh

\newcommand*\base[1][B]{\mathcal{#1}} % base

\titlet{Immersioni e embedding}

\begin{teo}
	Una varietà compatta chiusa $X$ di dimensione $n$ ammette embedding $X \hookrightarrow \R^{2n+1}$ e immersione $X \looparrowright \R^{2n}$.
\end{teo}
\begin{proof}
Si è già visto che $\exists N \tc \exists \text{ embedding } X \hookrightarrow \R^N $.

Scomponiamo $\R^N$ in $\R^{N-1} \times \R$. Ora, $\forall v \in \R^N \setminus \R^{N-1}: \ \R^N = \Span(v) \oplus \R^{N-1}$ e ad ogni tale $v$ corrisponde una proiezione $\fundef[\pi_v]{\R^N}{\R^{N-1}}$. Proiettiamo dunque $X$ su $\R^{N-1}$ tramite $\pi_v |_X$: vogliamo dunque trovare una condizione su $v$ per cui questa sia un'immersione. Si ha $\ker \pi_v = \Span(v)$, dunque $\pi_v |_X$ è immersione se $\forall x \in X, \forall z \in T_x X: \ z \notin \Span(v)$ (\wlg possiamo supporre $\norm{v} = \norm{z} = 1$, dunque se $z \neq \pm v$).
\marginpar{FIGURA possibilmente 3D con oggetto 3D proiettato sul piano lungo due direzioni diverse, alla peggio si può scendere tutto di una dimensione ma sarebbe molto meno chiaro (Ale)}

Sia $\fundef[\nu]{T(X)}{\R}$ l'applicazione $(x,z) \mapsto \norm{z}$ e sia $T_1(X) \is \nu^{-1}(1)$ (ovvero l'unione delle sfere unitarie di ogni $T_x X$). $T_1(X)$ è dunque sottovarietà di $T(X)$ (per teoremi di trasversalità) di dimensione $2n-1$, e si ha $T_1(X) \subseteq X \times S^{N-1}$. sia ora $\fundef[\rho]{T_1(X)}{S^{N-1}}$ la proiezione $(x, v) \mapsto v$; allora $\pi_v |_X$ è immersione $\leftrightarrow v \in S^{N-1} \setminus \im \rho$.

Per i teoremi di trasversalità avevamo visto che se $\dim X < \dim Y$ e $\fundef{X}{Y}$ liscia, $Y \setminus f(X)$ è denso in $Y$. Dunque se $\dim S^{N-1} > \dim T_1(X)$ (ovvero se $N > 2n$) esiste un insieme denso (aperto per compattezza di $X$) di vettori $v$ che danno $\pi_v |_X$ immersione. Dunque se esiste un embedding in dimensione maggiore di $2n$, esiste un'immersione in una dimensione in meno. \\
\marginpar{a mio parere è più pulito così: dato che nella dimostrazione data usiamo solo il fatto che l'embedding sia una immersione (ossia non stiamo usando l'iniettività), sappiamo che se esiste una immersione in dim N ne esiste una in N-1, a patto che N>2n.}
Per mostrare l'esistenza di un embedding in $\R^{2n+1}$ ragioniamo in modo analogo: a partire da un embedding $X \hookrightarrow \R^N$ cerco $v \tc \fundef[\pi_v |_X]{X}{\R^{N-1}}$ sia un embedding (ovvero, per compattezza di $X$, un'immersione iniettiva).

Consideriamo $X \times X \setminus \Delta$, con $\Delta \is \setdef[(a,b) \in X \times X]{a=b}$; si tratta di una varietà non compatta di dimensione $2n$. Sia ora $\fundef[\alpha]{X \times X \setminus \Delta}{S^{n-1}}$ l'applicazione $(x,y) \mapsto \frac{x-y}{\norm{x-y}} $.

Se $\pi_v$ non è iniettiva, $\exists x,y \in X \tc x-y \in \Span(v)$, ovvero (supponendo \wlg $\norm{v} = 1$) $\pi_v$ è iniettiva se $v \notin \im \alpha$, perché in questo caso non vi sono due punti di $X$ allineati lungo $v$. Dunque, se $\dim S^{N-1} > \dim X \times X \setminus \Delta$ (ovvero se $N > 2n+1$) esiste un insieme denso di vettori $v$ che rendono $\pi_v$ iniettiva; siccome inoltre esiste un denso aperto che la rende immersione, l'intersezione (non nulla per le proprietà di densità) contiene vettori che rendono $\pi_v$ immersione iniettiva, ovvero embedding. Procedendo induttivamente concludiamo che esiste un embedding in $\R^{2n+1}$ e dunque per quanto visto prima un'immersione in $\R^{2n}$.
\end{proof}

\titlet{Versione orientata dei teoremi di trasversalità}

\begin{prop}[Teorema 1 di \pitchfork, versione orientata]
Siano $X, Y \supseteq A$ varietà orientate e $\fundef{X}{Y}, \ f \pitchfork A$.
Allora $Z = f^{-1}(A)$ oltre a essere una sottovarietà di $X$ ammette una procedura di orientazione.
\end{prop}

Prima di esibire questa procedura, usiamo questo fatto per mostrare la versione orientata di un fatto già noto: sappiamo che per \[
\begin{tikzcd}
X_0 \arrow[dr, "f_0"] & \\
W \arrow[r, "F"] & \ Y \supseteq A \\
X_1 \arrow[ur, "f_1"] &
\end{tikzcd} \]
con $(X_0, f_0) \sim_{cob} (X_1, f_1)$ mediante $(W,F)$ e $f_0, f_1 \pitchfork A$, abbiamo per le preimmagini $Z_0 = f_0^{-1}(A), \ Z_1 = f_1^{-1}(A)$ che $(Z_0, f_0 |_{Z_0}) \sim_{cob} (Z_1, f_1 |_{Z_1})$ ed il cobordismo è realizzato da $(U, \tilde{F}|_U)$, dove $\tilde{F}$ è un'applicazione su $W$ vicina ad $F$ e trasversa ad $A$ per cui $\tilde{F}|_{X_0} = f_0, \ \tilde{F}|_{X_1} = f_1$, e $U = \tilde{F}^{-1}(A) \subseteq W$.

Ora, se $(X_0, f_0) \sim_{cob^+} (X_1, f_1)$ con il cobordismo orientato realizzato da $W, F$, applicando la procedura di orientazione alle controimmagini di $A$ otteniamo che $(U, \tilde{F}|_U)$ realizza il cobordismo orientato $(Z_0, f_0 |_{Z_0}) \sim_{cob^+} (Z_1, f_1 |_{Z_1})$.

\begin{proof}
Mostriamo ora la procedura di orientazione, partendo dal caso $A = \set{y_0}, \ \dim X = \dim Y$.

Si ha $f \pitchfork A \leftrightarrow y_0$ è valore regolare per $f$, dunque $f^{-1}(y_0) = \set{x_1, \dots , x_k}$ (finito per compattezza di $X$)\footnotemark è un insieme di punti non critici, e per l'uguaglianza delle dimensioni $\fundef[D_{x_i}f]{T_{x_i} X}{T_{y_0} Y}$ è un isomorfismo.
\footnotetext{Infatti se fossero infiniti avrebbero un punto di accumulazione, ma allora per continuità anche quel punto starebbe nella controimmagine e in quel punto il differenziale non sarebbe suriettivo, perchè in ogni intorno ha dei punti che hanno il suo stesso valore}

Ora, se $Y$, $y_0$ e $X$ sono orientati, fisso in $T_{y_0} Y$ una base $\base_{y_0}$ nella classe di orientazione di $Y$ e assegno a $x_i$ il segno (dunque l'orientazione) $+$ se la controimmagine di questa base per $D_{x_i}f$ è una base nella classe di orientazione di $X$, ed il segno $-$ se è nell'altra classe di orientazione.

Rilassando la condizione sulle dimensioni (dunque passando al caso $\dim X \ge \dim Y$, altrimenti non si potrebbe avere $D_x f$ suriettivo) e ponendo $Z = f^{-1}(y_0)$, per $x \in Z$ abbiamo $T_x X = T_x Z ~{\oplus}_{\perp}~ \nu(v)$, con $\nu(x)$ l'iperpiano ortogonale a $T_x Z$. $\fundef[D_x f]{\nu(x)}{T_{y_0} Y}$ è quindi isomorfismo, dunque $(D_x f)^{-1}(\base_{y_0})$ è base di $\nu(x)$; scelgo pertanto per $Z$ l'orientazione tale per cui una base di $T_x Z$ compatibile con essa completi la base di $\nu(x)$ appena esibita ad una base di $T_x X$ compatibile con l'orientazione di $X$. \marginpar{TODO: disegno di $\nu(x)$ e delle basi}

Il caso $\dim A > 0$ si riconduce a quanto mostrato per $A = \set{pt}$, come si vedrà nella lezione seguente.
\end{proof}
