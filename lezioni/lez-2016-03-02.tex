% Valerio

% anche come fosse antani scappella per tre

\newcommand*\Ps{\mathbb{P}} % Projective space
\newcommand*\Ro[1][n]{\mathbb{R}_0^{#1}} % set of real number vectors excluding origin
\newcommand*\tc{\ \text{t.c.} \ } % tale che

\titlet{Spazio proiettivo (reale)}
Sia $\Ro \is \R^n \setminus \set{0}$ (come sottospazio topologico); introduciamo su $\Ro[n+1]$ la relazione $\sim $ (di equivalenza proiettiva): \[
x, y \in \Ro[n+1], \ x \sim y \means  \exists \lambda  \in  \R \tc x = \lambda y \]

Le relative classi di equivalenza sono date da $[x]_\sim = \Span(x) \setminus \set{0}$.

Inoltre, essendo $S^n \subset \Ro[n+1]$, posso restringere tale relazione di equivalenza a $S^n$, ottenendo $\sim_S$ (rispetto a cui le classi di equivalenza sono date da $[x]_{\sim_S} = \set{x, -x} = [x]_\sim \, \cap \, S^n$).
Consideriamo ora gli insiemi quoziente $X \is \quoset{\Ro[n+1]}{\sim}$ e $Y \is \quoset{S^n}{\sim_S}$, con le rispettive proiezioni $\fundef[\pi]{\Ro[n+1]}{X}$ e $\fundef[\pi_S]{S^n}{Y}$.

\begin{oss}
	L'applicazione $[x]_\sim  \mapsto  \Span(x)$ è una biiezione tra $X$ e \break
	$\setdef[V \subset \R^{n+1}]{V \text{ è SSV\footnotemark{} di dimensione 1}}$.
	\footnotetext{Sottospazio vettoriale}
\end{oss}

\begin{prop}
	La biiezione $[x]_{\sim_S}  \mapsto  [x]_\sim$ (con inversa $[y]_\sim  \mapsto  [y]_\sim \cap S$) è un omeomorfismo tra $X$ e $Y$ (rispetto alle topologie quoziente).
\end{prop}

Si ricorda che gli aperti della topologia quoziente sono le immagini degli aperti saturi rispetto alla proiezione.

\begin{defn}[Cono]
	Un sottoinsieme $C\subseteq\Ro$ è un \emph{cono} se $\Span(C)\setminus\set{0}\subseteq C$ (e dunque, poiché $x\in\Span(x)$, se $\Span(C)\setminus\set{0}=C$).
\end{defn}
\begin{lemma}
	 Valgono le seguenti:
	\begin{itemize}
		\item $A \subseteq \Ro[n+1]$ è $\pi$-saturo $\leftrightarrow$ $A$ è un cono.
		\item $A \subseteq S^n$ è $\pi_S$-saturo $\leftrightarrow \, \forall x \in A \ -x \in A$.
	\end{itemize}
\end{lemma}

Pertanto, la biiezione tra $\setdef[A \subseteq {\Ro[n+1]} ]{A \text{ è $\pi$-saturo}}$ e $\setdef[A_S\subseteq S^n]{A_S \text{ è $\pi_S$-saturo}}$ data semplicemente da $A\mapsto A\cap S^n$, $\Span(A_S)\setminus \set{0} \mapsfrom A_S$ manda aperti in aperti e viceversa, mostrando continuità e apertura di $[x]_{\sim_S}  \mapsto  [x]_\sim$, che dunque è realmente un omeomorfismo; $X$ e $Y$ sono dunque topologicamente equivalenti. Questi quozienti sono dunque identificati sotto il nome di \emph{spazio proiettivo reale di dimensione $n$}, indicato con $\Ps^n$.


\begin{prop} $\Ps^n$ è una varietà (compatta). \end{prop}
È infatti $T_2$, 2-numerabile e localmente omeomorfo a $\R^n$.

\begin{proof}
Siano $[x], [y] \in \Ps^n$ distinti. Tornando alle controimmagini in $S^n$ abbiamo: \[
\pi_S^{-1}([x]) = \set{x, -x}; \
\pi_S^{-1}([y]) = \set{y, -y}; \
x \neq \pm y \]

Allora possiamo prendere $U_x, U_{-x}, U_y, U_{-y}$ intorni aperti (rispettivamente di $x, -x, y, -y$) in modo che non si intersechino e che $U_x = -U_{-x}, U_y = -U_{-y}$; le unioni $U_x \cup U_{-x}$ e $U_y \cup U_{-y}$ sono duque $\pi_S$-sature e disgiunte, e passando alle loro immagini otteniamo intorni aperti disgiunti di $[x]$ e $[y]$, dunque $\Ps^n$ è $T_2$. Similmente si può mostrare che è anche 2-numerabile. Inoltre, dalla continuità di $\pi_S$ e dalla compattezza di $S^n$ segue che $\Ps^n $ è compatto.

Esibiamo ora un atlante: vedendolo come quoziente di $\Ro[n+1]$, sia $\Pi_i$ il piano affine: \[
\setdef[(x_1, \dots, x_{n+1})\in {\Ro[n+1]}]{x_i = 1} \]
Allora la restrizione di $\pi$
\begin{gather*}
\fundef[\phi_i]{\Pi_i}{\Ps^n} \\
\phi_i : x \mapsto [x]
\end{gather*}
è iniettiva e continua. Con $U_i \is \phi_i(\Pi_i) \subset \Ps^n$ e $\fundef[\tau_i]{\Pi_i}{\R^n}$ l'isomorfismo banale, ho $\fundef[f_i \is \tau_i \circ \phi_i^{-1}]{U_i}{\R^n}$ biiettiva e continua, dunque se $U_i$ è aperto, $(U_i, f_i)$ è una carta per $\Ps^n$. Ma posto \[
G_i \is \setdef[(x_1, \dots, x_{n+1})\in {\Ro[n+1]}]{x_i = 0} \]
Ho $U_i = \pi(\Ro[n+1] \setminus G_i)$ e $\Ro[n+1] \setminus G_i$ è un aperto $\pi$-saturo, ergo $U_i$ è aperto.

$\set{(U_i, f_i)}_{1 \leq i \leq n+1}$ è dunque un atlante per $\Ps^n$.
\end{proof}

\begin{prop}
	$\fundef[\pi_S]{S^n}{\Ps^n}$ è un omeomorfismo locale, ovvero
	\begin{gather*}
	\forall x \in S^n \ \exists U \text{ intorno aperto di } x \tc \pi_S(U) \\
	\text{è aperto in } \Ps^n \ e \ \fundef[\pi_S | _{_U}]{U}{\pi_S(U)} \text{ è omeomorfismo}
	\end{gather*}
\end{prop}
La dimostrazione è lasciata per esercizio al lettore.

\titlet{Filtrazione di $\Ps^n $}
Poiché si ha $\pi^{-1}(\Ps^n \setminus U_i) = G_i$, il quale è chiaramente identificabile con $\R^n \setminus \set{0} = \Ro$, e inoltre vale evidentemente $\pi | _{\Ro} (\Ro) = \Ps^{n-1}$, valgono:
\begin{itemize}
	\item $\Ps^n = U_i \sqcup \Ps^{n-1}$
	\item $\Ps^n \setminus \Ps^{n-1}$ è omeomorfo a $\R^n$
\end{itemize}


