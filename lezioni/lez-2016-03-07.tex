%Trascrizione Bob

%gnegne

\begin{prop}
	Lo spazio proiettivo $\Ps^n$ è una $n$-varietà differenziabile.
\end{prop}

\begin{proof}
	Mostriamo che l'atlante topologico
	$\{(U_1,f_1),\dots,(U_{n+1},f_{n+1})\}$
	fornito precedentemente è in effetti un atlante differenziabile.

	Ricordiamo che $U_j$ è l'immagine tramite $f_j^{-1}$ di $\R^n$
	identificato con il piano $\Pi_{j}= \{x_{j}=1 \}$,
	e che $\fundef[f_j]{U_j}{\R^n}$ ha quest'espressione:
	\[f_j([x_1,\dots,x_{n+1}])=
	\left(\frac{x_1}{x_j},\dots,\frac{x_{j-1}}{x_j},\frac{x_{j+1}}{x_j},\dots,\frac{x_{n+1}}{x_j}\right)\]
	dove infatti si ha $x_j\neq 0$ per $[x]\in U_j$.
	Conseguentemente avrò $\fundef[f_i^{-1}]{\R^n}{\mathbb{P}^n}$ con:
	\[f_i^{-1}(y_1,\dots,y_n)=[y_1,\dots,y_{i-1},1,y_{i+1}\dots,y_n]\]
	Se adesso mi restringo all'intersezione $U_j \cap U_i$ e considero la composizione:
	\begin{align*}
		f_j \circ f_i^{-1}(\xi_1,\dots,\xi_n) &=
		f_j([\xi_1,\dots,\xi_{i-1},1,\xi_{i+1}\dots,\xi_n]) = \\
		&= \left( \frac{\xi_1}{\xi_j},\dots,
		\frac{\xi_{i-1}}{\xi_j},\frac{1}{\xi_j},\frac{\xi_{i+1}}{\xi_j},\dots,
		\frac{\xi_{j-1}}{\xi_j},\frac{\xi_{j+1}}{\xi_j},\dots,
		\frac{\xi_n}{\xi_j} \right)
	\end{align*}
	questa funzione è in effetti liscia.
\end{proof}

\begin{ex}
	Si verifichi che anche $S^n$ munito dell'atlante
	$\{(S^n\setminus \set N,f_N),(S^n\setminus\set S,f_S)\}$,
	citato precedentemente, è una $n$-varietà differenziabile.
\end{ex}

\begin{defn}[Carte $f$-adattate]
	Siano $M$, $N$ varietà e $\fundef MN$.
	Due carte $(U,\phi)$ e $(W,\psi)$ rispettivamente di $M$ e $N$
	sono \emph{$f$-adattate} se $f(U)\subseteq W$.
\end{defn}

\begin{defn}[$C^\infty$ su varietà]
	Con la stessa notazione della definizione precedente,
	$f$ si dice \emph{$C^\infty$} o \emph{liscia} se
	$\forall x \in M \, \exists (U_x,\phi), (W,\psi)$ $f$-adattate tali che
	$\psi \circ f \circ \phi^{-1}$ è liscia.
	La funzione $\psi \circ f \circ \phi^{-1}$ è detta \emph{rappresentazione locale di $f$}.
\end{defn}

\begin{ex}
	Se $f$ è $C^\infty$ lo sono anche tutte le sue rappresentazioni locali.
\end{ex}

\begin{ex}
	$C^\infty$ su varietà $\implies$ continua
\end{ex}

\begin{defn}[Diffeomorfismo su varietà]
	Una funzione tra varietà è un \emph{diffeomorfismo}
	se è $C^\infty$, invertibile e con inversa $C^\infty$.
	Se tra due varietà esiste un diffeomorfismo, si dicono \emph{diffeomorfe}.
\end{defn}

\begin{oss}
	Essere diffeomorfe è una relazione più debole della compatibilità tra atlanti.
\end{oss}

\begin{es}
	Considero lo spazio topologico $M \is (\R,\tau_E)$ e due funzioni da~$M$ a~$\R$:
	\[\phi_0\is \id; \qquad
	\phi_1(x)\is
	\begin{cases}
		x & x\le 0 \\
		2x & x>0
	\end{cases}\]
	$\{(M,\phi_0)\}$ e $\{(M,\phi_1)\}$ sono due atlanti differenziabili
	(banalmente perchè costituiti da una sola carta),
	ma non sono compatibili:
	infatti $\{(M,\phi_0),(M,\phi_1)\}$ \emph{non} è un atlante differenziabile
	poichè $\phi_1 \circ \phi_0^{-1} = \phi_1$ che non è differenziabile in~0.
	Tuttavia le due varietà differenziabili sono diffeomorfe attraverso $\phi_1$,
	infatti $\phi_0 \circ \phi_1 \circ \phi_1^{-1} = \id$,
	che evidentemente è differenziabile.
\end{es}

\begin{es}
	Più in generale,
	sia $\fundef M{M_0}$ omeomorfismo
	e $\{(U_i,\phi_i)\}$ atlante differenziabile di~$M_0$.
	Allora $\{(f^{-1}(U_i),\phi_i\circ f)\}$ è un atlante differenziabile su~$M$
	che rende $f$ un diffeomorfismo.
\end{es}

\titlet{Linearizzazioni di applicazioni lisce tra spazi euclidei}

\begin{fat}[Teorema della funzione inversa]
	Sia $\fundef{\Omega\subseteq\R^n}{\R^n}$ liscia con $\Omega$ aperto
	e per un certo $x\in\Omega$ sia $\de_xf$ invertibile.
	Allora esiste un intorno aperto $U$ di $x$
	tale che $f(U)$ è aperto e $\fundef[f|_U]U{f(U)}$ è un diffeomorfismo.
\end{fat}

In altre parole,
restringendo $f$ a un intorno aperto opportuno $U$ di~$x$ in~$\Omega$,
$(U,f|_U)$ è una carta locale di~$\Omega$ intorno a~$x$.

È interessante notare come questo teorema,
a partire da un'informazione locale (in un punto),
ci restituisce informazioni ``localmente globali''.

Vediamo i teoremi della funzione implicita,
che sono corollari di questo teorema.
Definisco un \emph{modello locale lineare},
ovvero una proiezione $\fundef[\pi]{\R^n}{\R^m}$ con $n \ge m$
tale che $\pi(x_1,\dots,x_n)=(x_1,\dots,x_m)$.
Noto che $\forall x \in \R^n$, $\de_x\pi$ è surgettivo
(essendo $\de_x\pi=\pi$ per linearità di $\pi$).

\begin{fat}[Teorema della funzione implicita, versione surgettiva]
	\label{th:funimpsurg}
	Sia $\fundef{U\subseteq\R^n}{\R^m}$ liscia con $U$ aperto
	e per un certo $x\in U$ sia $\de_xf$ surgettivo.
	Allora esiste un diffeomorfismo $\fundef[\phi]{U'\subseteq\R^n}U$
	con $U'$ e $\phi(U')$ aperti
	tale che $f\circ\phi=\pi|_{U'}$, cioè che fa commutare il diagramma:
	\begin{center}
		\input{figura4.pdf_tex}
	\end{center}
\end{fat}

In termini di varietà, $\phi$ è una parametrizzazione locale di $U$ intorno a $x$.

Similmente a prima definisco un modello locale lineare,
stavolta una inclusione $\fundef[j]{\R^n}{\R^m}$ con $n \le m$
tale che $j(x_1,\dots,x_n)=(x_1,\dots,x_n,0,0,\dots)$.
Noto che $\forall x \in \R^n$, $\de_xj$ è iniettivo
(essendo $\de_xj=j$ per linearità di $j$).

\begin{fat}[Teorema della funzione implicita, versione iniettiva]
	Sia $\fundef{U\subseteq\R^n}{\R^m}$ liscia con $U$ aperto
	e per un certo $x\in U$ sia $\de_xf$ iniettivo.
	Allora esistono un intorno aperto $U'$ di $x$ con $f(U')$ aperto
	e un diffeomorfismo $\fundef[\psi]{f(U')}{W\subseteq\R^m}$ con $W$ aperto
	tale che $\psi\circ f=j|_{U'}$, cioè che fa commutare il diagramma:
	\begin{center}
		\input{figura5.pdf_tex}
	\end{center}
\end{fat}

Ancora in termini di varietà, $\big(f(U'),\psi\big)$ è una carta locale di $f(U)$ intorno a~$f(x)$.

\begin{prop}
	Sia $\fundef{\R^n}{\R^n}$ differenziabile con $\de_xf$ invertibile ovunque.
	La composizione del differenziale $\fundef[\de f]{\R^n}{\GL_n}$,
	del determinante $\fundef[\det]{\GL_n}{\Ro[]}$
	e della funzione segno $\fundef[\sgn]{\Ro[]}{\{\pm1\}}$
	è una funzione costante.
\end{prop}

\begin{proof}
	$\R^n$ è connesso
	e ${\sgn} \circ {\det} \circ \de f$ è composizione di funzioni continue,
	quindi continua.
\end{proof}

\begin{oss}
	I diffeomorfismi hanno differenziale invertibile ovunque.
\end{oss}

\begin{defn}
	Se $\forall x \in \R^n$, $ \det \de_xf > 0$,
	si dice che $f$ \emph{preserva l'orientazione}.
\end{defn}

\begin{oss}
	Per la proposizione precedente basta valutare il segno di $\det (\de_xf)$ in un punto.
	Inoltre se $f$ non preserva l'orientazione basterà comporla con
	\[\tau:(x_1,\dots,x_n)\mapsto (x_1,\dots,-x_n)\]
	per ottenere una funzione che la preserva.
	Ci si potrà quindi restringere allo studio delle funzioni che preservano l'orientazione.
\end{oss}

% sta nella lezione successiva
% \begin{teo}
% Sia $\fundef{\R^n}{\R^n}$ un diffeomorfismo che preserva l'orientazione $\implies$ $f$ è \emph{diff-isotopo}, ovvero:
% $\exists \; \fundef[F]{\R^n \times [0,1]}{\R^n}$ con $f_t = F|_{\R^n \times \{t\}}$, tale che:
% \begin{itemize}
% \item $F$ è differenziabile
% \item $f_0=f$
% \item $f_1 = \id$
% \item $f_t$ è un diffeomorfismo $\forall t \in [0,1]$
% \end{itemize}
% \end{teo}
