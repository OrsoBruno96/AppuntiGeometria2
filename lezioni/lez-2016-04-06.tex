% Federico
%cusu

\newcommand*\tc{\ \text{t.c.} \ } % tale che
\newcommand*\dual{{^\ast}} % dual
\newcommand*\base[1][B]{\mathcal{#1}} % base

\titlet{Richiami della lezione scorsa}

Sia $M$ una varietà differenziabile con atlante massimale $\set{(U_i, \phi_i)}$ e sia $\set{ \fundef[\mu_{ij}]{U_i \cap U_j}{G} }$ un cociclo a valori in un gruppo $G \subseteq \Aut(F)$ (con $F$ varietà liscia). Ripetendo la costruzione usata per realizzare il fibrato tangente $T(M)$ (per cui s'era usato $G = \GL(n, \R)$ e $F=\R^n$, $n = \dim M$), ottengo un fibrato $\Fundef[\pi]{E}{M}$ di fibra $F$ e gruppo strutturale $G$.

\begin{defn}[Equivalenza tra fibrati in termini di cocicli]
Siano dati due fibrati su $M$ con lo stesso gruppo di struttura e la stesa fibra: $\fundef[\pi_1]{E_1}{M}$ e  $\fundef[\pi_2]{E_2}{M}$.
Siano $\set{ \fundef[\mu_{ij}]{U_i \cap U_j}{G} }$ e $\set{ \fundef[\lambda_{ij}]{U_i \cap U_j}{G} }$ rispettivamente coclici di $E_1$ e $E_2$.
$E_1$ ed $E_2$ sono fibrati equivalenti (nel senso dei coclici) se $\exists \fundef[\gamma_{i}]{U_i}{G}$ tale che $\forall x U_i \cup U_j$ dati $\lambda_{ij}$ e $\mu_{ij}$
sia $\lambda_{ji} = {\gamma_j}^{-1} \times \mu_{ji} \times  {\gamma_i}^{-1}$.  (Quest'ultima è la moltiplicazione tra elementi del gruppo $G$)
\end{defn}

\begin{defn}[Immersione liscia]
Sia $\fundef[f]{X}{Y}$, $f$ è una immersione se $\forall x \in X \fundef[Df_{x}]{T_{x}X}{T_{f(x)} X}$ (restrizione dell'applicazione tangente alla fibra) è iniettiva.
\end{defn}

\begin{defn}[Embedding] 
 Una immersione $f$ è un embedding se $\fundef[f]{X}{f(X)}$  è un omeomorfismo (funzione continua fra spazi topologici tale che ha inversa continua).
\end{defn}

%Nn mi è chiaro cosa volesse dire con questa frase...
%Se $X$ è compatta e $f$ è iniettiva e suriettiva allora $f$ è un embedding.

\titlet{Teorema di Embedding}

\begin{teo}[Di embedding elementare di Whitney]
 Se $X$ è una varietà compatta allora esiste un $N$ abbastanza grande tale che $\exists$ un embedding $\fundef[f]{X}{\mathbb{R}^{n}}$
\end{teo}


\begin{proof}
 Poiché $X$ è compatta esiste un atlante finito $\set{(U_i, \phi_i)}_{i = 1, ..., m}$ tale che 
 \begin{itemize}
  \item $\ball[2]0 \subseteq {\phi}_{i}(U_{i})$
  \item ${\phi_{i}^{-1}(\ball[1]0)}$ ricopre tutto $X$
  \end{itemize}
 Sia $\fundef[\lambda]{\R^N}{[0, 1]}$ la funzione a foruncolo relativa alle palle $B_{1}$ e $B_{2}$. Definisco la funzione $\fundef[\lambda_i]{U_i\subseteq X}{[0,1]}$
 come $\lambda \circ \phi_{i}$ in $U_{i}$ e $0$ in $X \setminus U_{i}$.
 
 ${B_{i}}$ ricopre $X$ e $B_{i} = {\lambda}^{-1} \subseteq U_{i}$. Poiché $\lambda_i$ vale 1 quando sono su ${\phi_{i}}^{-1} (\ball[1]0)$ ho un ricoprimento.


\begin{figure}
    \centering % per centrare l'immagine
    \input{figura41.pdf_tex}
    \caption{Palle $B_{1}$ e $B_{2}$}
\end{figure}
 
 Definisco $\fundef[f]{X}{\R^N}$ come $\lambda_{i}  \phi_{i}$ in $U_{i}$ e $0$ in $X \setminus U_{i}$. Considero le funzioni $\fundef[g_{i}]{X}{\mathbb{R}^{N}}$
 definita come $x \mapsto (f_{i}(x), \lambda_{i}(x))$. Definisco ancora la funzione $g$ che: $x \mapsto (g_1, g_2, \dots, g_m)$.
 
 \marginpar{da rivedere - fede}
 
 Dico che $g$ così definita è un embedding, devo dunque dimostrare che è una immersione (cioè che il suo differenziale è iniettivo) e che la funzione stessa $g$ è omomorfismo nella sua immagine
 cioè ha funzione inversa continua.
 
 \begin {itemize}
  \item è diffeomorfismo per costruzione
  \item tutte le $g_{i}$ per come son state costruite sono immersioni ogniuna nel suo $\psi^{-1}(B_1(0))$
  \item g è iniettiva: se ho $x \neq y$  ho due possibilità: $x, y \in B_{i}$ allora la $f_i$ coincide con la $\phi_i$ sulla palla $B_{i}$, che è iniettiva sulla palla. 
  Oppure $x, y$ stanno in due palle diverse in particolare $y \in B_{i}$ allora $\lambda_{i}(y) = 1$ e $\lambda_{i}(x) = 0$. Dunque ogni funzione $g_i$ è iniettiva e quindi lo è anche la funzione
  $g$.
  \qedhere
 \end {itemize}
\end{proof}

Se ho una varietà $X$ con bordo $\boundary X \neq 0$ si può rafforzare la costruzione in modo che $(X, \boundary X)$ sia una sottovarietà del semipiano 
$(\mathbb{H}^{N}, \boundary \mathbb{H}^{N})$ con $\boundary X = X \cup \boundary \mathbb{H}^{N}$ e su $\boundary X$ ho $X \perp \boundary \mathbb{H}^{N}$.

\begin{figure}
    \centering % per centrare l'immagine
    \input{figura42.pdf_tex}
    \caption{Varietà con bordo}
\end{figure}

Inoltre se $x_{1}, x_{2}, x_{3}, \dots, x_{k}$ sono punti della varietà con bordo non nullo si può rafforzare la costruzione in modo che un intorno di $x_{j}$ va in un piano di $\mathbb{R}^{N}$.

\begin{defn}[Metrica Riemanniana] %a caso
 Una metrica riemanniana su una varietà $X$ è un campo di tensori di tipo (0,2) su $X$ simmetrici e definiti positivi. Cioè è una sezione del 
 fibrato tangente tale che ogni tensore associato ad un punto della varietà è simmetrico e definito positivo: $\fundef[R]{TX}{{T_{2}}^{0}}$ 
\end{defn}

\begin{teo}
Ogni $X$ compatta ha una metrica Riemanniana.
\end{teo}

\begin{proof}
  Esistendo un embedding, per il teorema sopra, posso vedere la varietà come immersa in un $\mathbb{R}^{n}$ dotato del suo prodotto scalare canonico $g_{0}$. $X \hookrightarrow \mathbb{R}^{n}$.
  Posso definire la metrica riemanniana sulla varietà $X$ come $g_{x} = g_{0}|_{T_x X}$.
 Ad ogni punto $x$ prendo dunque come metrica il tensore associato al prodotto canonico ristretto all'immagine del $T_x X$ attraverso l'embedding.
\end{proof}


\titlet{Topologia sullo spazio delle applicazione lisce tra varietà - fede}

\marginpar{Sii più specifico su come entra in gioco il compatto in questa definizione}

\titlet{Spazi di applicazioni lisce}
\begin{defn}[Topologia sulle applicazioni lisce tra varietà in $\mathbb{R}^{N}$]
Definisco $\mathcal{E} \is \setdef[\fundef{X}{Y}]{f \, \text{liscia}}$, con $X$ e $Y$ varietà lisce. Voglio munire $\mathcal{E} \is \setdef[\fundef{X}{Y}]{f \, \text{liscia}}$ di una topologia. Se $X \subseteq \mathbb{R}^{N}$ e $Y = \mathbb{R}^{N}$ considero
la base di intorni data $U_{r, k, \epsilon}$ e $r \in \mathbb{N}$ e $K \subseteq X$  è un compatto e $\epsilon > 0$. Questa base di intorni è definita da 
\[U_{r, K, \epsilon} \is \Setdef[{\fundef[g]{X}{\mathbb{R}^{N}}}]{\left\|\frac{\partial f}{\partial x_{i_{1}} \dots \partial x_{i_{k}}} - \frac{\partial g}{\partial x_{i_{1}} \dots \partial x_{i_{k}}}\right\| \eqslantless \epsilon}\]
Posso considerare un ricoprimento compatto finito,  ${K_i}$ (che trovo se la varietà $X$ è compatta) e posso prendere l'intersezione dei $U_{r,K_i, \epsilon}$, in questo modo ottengo
dei $U_{r, \epsilon}$ che sono indipendenti dal compatto $K$, questo è ora un intorno della funzione $\fundef[f]{X}{Y}$. 
Unendo gli intorni per tutte le funzioni ottengo effettivamente una base di intorni per lo spazio delle funzioni lisce $\mathcal{E}$.
\end{defn}

\marginpar{perchè ti restringi (solo in arrivo mi sembra di capire) a varietà reali? - non ho capito la critica - fede}

\begin{defn}
Per definire una topologia tra varietà generiche mi riduco al caso di varietà reali.
Definisco una base di intorni $U_{r, K, \epsilon, (U, \phi), (U', \phi')} = \set{\fundef[g]{X}{\mathbb{R}^{N}}}$ tale che $g(U) \subseteq U'$ e valga 
$\phi \circ g \circ \phi^{-1} \in U_{r, K, \epsilon}(\phi \circ g \circ \phi^{-1})$ cioè che $\phi \circ g \circ \phi^{-1}$ sia intorno di una funzione da $\mathbb{R}^{N}$ a $\mathbb{R}^{N}$.
Posso considerare un ricoprimento compatto finito,  ${K_i}$ (che trovo se la varietà X è compatta) e posso prendere l'intersezione dei $U_{r,K_i, (U_i, \psi_i)(U'_i, \psi'_j)}$, in questo modo ottengo
dei $U_{r, \epsilon, (U, \phi), (U', \phi')}$ che sono indipendenti dal compatto $K$, questo è ora un intorno della funzione $\fundef[f]{X}{Y}$. Si procede similmente a quanto fatto precedentemente 
per rimuovere la dipendenza dal compatto $K$.
Unendo gli intorni per tutte le funzioni ottengo effettivamente una base di intorni per lo spazio delle funzioni lisce $\mathcal{E}$.

\end{defn}

\begin{oss}
 La topologia che ottengo è metrizzabile.
\end{oss}







