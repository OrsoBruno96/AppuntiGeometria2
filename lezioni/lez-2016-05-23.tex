%Trascrizione iniziale: Costa

Fissata una triade $(W, V_0, V_1)$ e una funzione di Morse $f:W \rightarrow \left [0,1\right ]  $, posso ottenere una decomposizione in manici della suddetta triade.

\begin{example}
 Sia $W$ una superficie compatta, chiusa, connessa.
 \marginpar{fai disegni alberto}
 \chi(W) = $\card$ $0$-manici - $\card$ $1$-manici + $\card$ $2$-manici.
\end{example}

Per contarli, vorrei potermi dimenticare dei massimi e minimi e selle. 
Pertanto, vogliamo trovare una serie di mosse che semplifichino la superficie lasciandone invariate le proprietà cercate.

\begin{enumerate}
 \item Cancellazione di una coppia ($0$-manico, $1$-manico) in posizione complementare.
 \begin{figure}
    \centering
    \input{mossa1.pdf_tex}
    \caption{retrazione 1-manico}
\end{figure}
  Infatti nel conteggio di $\chi (W)$, conteggiando con il segno giusto i manici da semplificare, si ottiene $1-1 = 0$
  \begin{teo}
    Posso cancellare $0$-manici e $1$-manici fino a restare con un solo $0$-manico
  \end{teo}
  \begin{proof}
    Siccome la superficie è connessa, posso cancellare a coppie per induzione i manici. Se parto infatti con una collezione di $0$-manici, questi possono essere connessi solo tramite rete di $1$-manici.
    Li retraggo a 2 a 2 e resto con un solo $0$-manico.
  \end{proof}
  \begin{cor}
  Esiste una funzione di Morse con un solo punto di minimo.
  \end{cor}
  \begin{oss}
  Per un ragionamento duale ($f\rightarrow -f$) ho anche un solo $2$-manico (e dunque esiste una funzione di Morse con un solo massimo).
  \end{oss}
\item Scivolamento.
\begin{figure}
    \centering
    \input{scivolamento.pdf_tex}
    \caption{procedura di scivolamento}
\end{figure}
Posso assumere che tutti gli $1$-manici si attachino solamente lungo il bordo dello $0$-manico, che deve essere uno solo perchè lo devo poter tappare con un solo $2$-manico.
\end{enumerate}

Segue che ogni superficie compatta connessa ammette una decomposizione in manici del seguente tipo
\begin{enumerate}
 \item Un solo $0$-manico $D^2$.
 \item Si attaccano solamante $k$ $1$-manici lungo $\partial D^2$. Da questo punto e dal precedente risulta una $S$ connessa con $\boundary S \sim S^1$
 \item Si tappa $\boundary S$ con un unico $2$-manico.
\end{enumerate}
\begin{oss}
 $\chi(W) = 2-k$, ossia, più volgarmente, ci dice che la $\chi$ conta i buchi.
\end{oss}
\begin{es}
 Per $k=0$ si ha $S^2$. Per $k=1$ si ha il $\mathbb P^2$ (che è un nastro di Moebius); un toro non può essere tappato con un $2$-manico.
 Per $k=2$ vanno bene solo la superficie disegnata, che tappata con un $2$-manico diviene un toro $T_2$, e $\mathbb P^2 \csum \mathbb P ^2$.
 \marginpar{figura strana}
\end{es}

Siccome la $\chi$ è troppo difficile da calcolare, usiamo la forma di intersezione. 
Supponiamo di avere una superficie $W$ e una forma di intersezione $\beta$ totalmente isotropa.
\begin{oss}
 esiste una curva semplice $\left [ c\right ]\neq 0$, $c\cdot c = 0$. Dato che $c$ non sconnette $W$, $c$ ha un intorno a nastro diffeomorfo a $c \times \left[ 0 \right ]$
\end{oss}
Segue che esiste $c'$ tale che $c'\cdot c = 1$ e $\left [ c' \right ]\neq 0$. Ma dato che $\beta$ è completamente isotropo, $\c' \cdot c'= 0$.
Allora abbiamo ottenuto un toro: $W = T_2\csum Z$, dove $Z$ non la conosco, ma so che ha un numero di manici inferiore (scatta l'induzione!).
\begin{oss}
 $\beta$ è totalmente isotropo anche ristretta a $Z$
\end{oss}
\begin{oss}
 $Z$ ha una decomposizione in manici con 2 manici in meno rispetto a $W$.
\end{oss}
\begin{teo}
 Se $W$ ha una $\beta$ totalmente isotropa, allora $W\simeq \underbrace {T_2\csum \dots \csum T_2}_{g addendi}$ e $\chi (W) = 2 - 2g$.
\end{teo}

Supponiamo ora che $W\neq S^2$ e $\beta$ non totalmente isotropa, ossia esiste $c$ semplice su $W$ tale che $c\cdot c = 1$.
\begin{oss}
 Allora $c$ ha un intorno che è diffeomorfo ad un nastro di Moebius, ossia $W=\mathbbP ^2 \csum Z$
\end{oss}
Anche qui adesso vogliamo far scattare l'induzione come sopra! Abbiamo due casi:
\begin{enumerate}
 \item $W\simeq \underbrace {\mathbb P^2\csum \cdot \mathbb P^2}_{k addendi}$ e $\chi (W) = 2 - k$
 \item $W\simeq\underbrace{\mathbb P^2\csum \dots \csum \mathbb P^2}_{r volte}\csum Z$ con $Z$ totalmente isotropo.
 Tuttavia se $Z$ è totalmente isotropo, possiamo applicare i ragionamenti fatti in precdedenza. Così otteniamo la scomposizione
 $W \simeq \underbrace{\mathbb P^2\csum \dots \csum \mathbb P^2}_{r volte}\csum \underbrace{T_2\csum \dots \csum T_2}_{g volte}$.
 Segue che $\chi(W) = 2-r-2g$.
\end{enumerate}
\begin{lemma}
 da farselo spiegare.
\end{lemma}



