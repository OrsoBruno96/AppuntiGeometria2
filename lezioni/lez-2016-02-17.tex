% trascrizione: Petrillo

% questo lemma non lo ricordo ma mi serviva dopo ed è stato usato a lezione
\begin{lemma}\label{th:apertointorni}
	Un insieme è aperto se ogni punto ha un intorno contenuto nell'insieme:
	\[A\text{ aperto}\iff
	\forall x\in A\,\exists U_x\subseteq A\]
\end{lemma}

\begin{proof}
	L'implicazione verso destra è banale perché $A$ è un intorno dei suoi punti\footnote{Se è banale, perché l'abbiamo spiegata?}.
	Mostriamo l'altra: per definizione di intorno possiamo prendere gli $U_x$ aperti. Sia $A'\is\union_{x\in A}U_x$. $A'$ è aperto perché unione di aperti ed è contenuto in $A$, però contiene tutti i punti di $A$ quindi $A=A'$.
\end{proof}

% anche questa parte sui sottospazi non la ricordo ma dopo è data per scontata
% forse è perché quel giorno sono arrivato tardi a lezione
\titlet{Sottospazi}

Si può definire una topologia sui sottoinsiemi di uno spazio topologico:

\begin{prop}[Topologia dei sottospazi]
	Sia $(X,\tau)$ spazio topologico e $\chi\subseteq X$, l'insieme
	$\tau_\chi\is\setdef[A\cap\chi]{A\in\tau}$
	è una topologia su $\chi$.
\end{prop}

\begin{proof}
	Verifichiamo le tre proprietà della topologia:
	\begin{itemize}
		\item Il vuoto c'è perché $\nullset\cap\chi=\nullset$;
		$\chi$ c'è perché $X\cap\chi=\chi$.
		\item Siano $a_1,\ldots,a_n\in\tau_\chi$.
		Allora $\forall k\,\exists A_k\in\tau:a_k=A_k\cap\chi$.
		Quindi:
		\[\inters a_k=\inters(A_k\cap\chi)=\left(\inters A_k\right)\cap\chi\]
		Ma $\inters A_k$ è aperto in $(X,\tau)$.
		\item Sia $\set{a_k}_{k\in I}\subseteq\tau_\chi$.
		Definiti come sopra gli $A_k$, abbiamo:
		\[\union a_k=\union(A_k\cap\chi)=\left(\union A_k\right)\cap\chi \qedhere\]
	\end{itemize}
\end{proof}

\begin{defn}[Sottospazio]
	Chiamiamo $(\chi,\tau_\chi)$ \emph{sottospazio} di $(X,\tau)$.
\end{defn}

È interessante che la topologia dello spazio di partenza si colleghi direttamente alla topologia di sottospazio di ogni aperto:

\begin{prop}
	$A$ aperto $\iff$ tutti gli aperti in $A$ come sottospazio sono aperti
\end{prop}

\begin{proof}
	Mostriamo le due implicazioni:
	\begin{description}
		\item[\proofrightarrow]
		Infatti ogni aperto in $A$ è intersezione di due aperti.
		\item[\proofleftarrow]
		Infatti $A$ è aperto in $A$. \qedhere
	\end{description}
\end{proof}

\titlet{Chiusura, parte interna, frontiera}

In generale i sottoinsiemi di uno spazio topologico non sono né aperti né chiusi. Tuttavia possiamo associare a ognuno in modo naturale un aperto e un chiuso.

\begin{defn}[Chiusura]
	La \emph{chiusura} di un insieme $Y$ è l'intersezione di tutti i chiusi che lo contengono:
	\[\clos Y\is\inters_{\substack{C\text{ chiuso}\\C\supseteq Y}}C\]
\end{defn}

\begin{prop}
	La chiusura di $Y$ è il più piccolo chiuso che contiene $Y$, cioè:
	\[\begin{rcases}
		Y\subseteq C\subseteq\clos Y\\
		C\text{ chiuso}
	\end{rcases}\implies C=\clos Y\]
\end{prop}

\begin{proof}
	Infatti dalla definizione di $\clos Y$ segue $\clos Y=\clos Y\cap C$.
\end{proof}

\begin{defn}[Parte interna]
	La \emph{parte interna} di un insieme $Y$ è l'unione di tutti gli aperti contenuti in $Y$:
	\[\inter Y\is\union_{\substack{A\text{ aperto}\\A\subseteq Y}}A\]
\end{defn}

\begin{prop}
	La parte interna di $Y$ è il più grande aperto contenuto in $Y$, cioè:
	\[\begin{rcases}
		\inter Y\subseteq A\subseteq Y\\
		A\text{ aperto}
	\end{rcases}\implies A=\inter Y\]
\end{prop}

\begin{proof}
	Infatti dalla definizione di $\inter Y$ segue $\inter Y=\inter Y\cup A$.
\end{proof}

\begin{oss}
	La parte interna e la chiusura sono rispettivamente aperta e chiusa.
\end{oss}

Definiamo ora alcune proprietà che collegano singoli punti di un insieme alle nozioni di chiusura e parte interna.

\begin{defn}[Punto interno]
	Gli elementi della parte interna di $Y$ si chiamano \emph{punti interni di $Y$}:
	\[x\text{ interno a }Y\means x\in\inter Y\]
\end{defn}

\begin{defn}[Punto di accumulazione]
	Un elemento $x$ (non necessariamente in $Y$) è \emph{punto di accumulazione per $Y$} se ogni intorno di $x$ contiene punti di $Y$ diversi da $x$:
	\[x\text{ accumulazione per }Y\means\forall\,U_x:(U\setminus\set{x})\cap Y\neq\nullset\]
\end{defn}

\begin{defn}[Punto isolato]
	I punti di $Y$ non di accumulazione sono \emph{isolati}:
	\[x\text{ isolato in }Y\means
	\begin{cases}
		x\in Y\\
		\neg(x\text{ accumulazione per }Y)
	\end{cases}\]
\end{defn}

\begin{lemma}
	\label{th:chiusuraacc}
	La chiusura di un insieme è l'insieme stesso unito ai suoi punti di accumulazione:
	\[\clos Y=Y\cup\setdef{x\text{ accumulazione per }Y}\]
\end{lemma}

\begin{proof}
	Mostriamo le due inclusioni:
	\begin{description}
		\item[\proofsubseteq]
		Ci basta verificare che $y\in\clos Y\setminus Y\implies y\text{ accumulazione per }Y$.
		Per assurdo:
		\[y\text{ non di accumulazione}\so
		\exists A\text{ aperto}:
		\begin{cases}
			y\in A\\
			A\cap Y=\nullset
		\end{cases}\text{cioè }
		\begin{cases}
			y\not\in\comp A\\
			Y\subseteq\comp A
		\end{cases}\]
		Ma $\comp A$ è chiuso, quindi avremmo $y\not\in\clos Y\absurd$.
		\item[\proofsupseteq]
		Ci basta verificare che $y\not\in Y\et\, y\text{ accumulazione per }Y\implies y\in\clos Y$.
		Per assurdo, $y\not\in\clos Y\so y\in\comp{\clos Y}$. Ma $\comp{\clos Y}$ è aperto, quindi $\comp{\clos Y}$ sarebbe un intorno di $y$ disgiunto da $Y\absurd$. \qedhere
	\end{description}
\end{proof}

Vogliamo ora formalizzare il concetto di ``bordo'' di un insieme, cioè la famiglia dei punti che lo ``separa'' dal suo complementare. Diamo questa definizione:

\begin{defn}[Frontiera]
	La famiglia dei punti della chiusura di $Y$ esterni a $Y$ è la \emph{frontiera} di $Y$:
	\[\bound Y\is\clos Y\setminus\inter Y\]
\end{defn}

\begin{prop}
	La frontiera consta dei punti esterni isolati o di accumulazione:
	\[\bound Y=(\setdef{x\text{ isolato in }Y}\cup\setdef{x\text{ accumulazione per }Y})\setminus\inter Y\]
\end{prop}

\begin{proof}
	Basta usare il \autoref{th:chiusuraacc} e osservare che $Y\cup\setdef{x\text{ accumulazione}}=\setdef{x\text{ isolato}}\cup\setdef{x\text{ accumulazione}}$.
\end{proof}

\begin{prop}
	La frontiera di un insieme è anche frontiera del complementare:
	\[\bound Y=\bound{\comp Y}\]
\end{prop}

\begin{proof}
	Passando al complementare le unioni diventano intersezioni quindi parte interna e chiusura si scambiano:
	\[\inter{(\comp Y)}=
	\union_{\substack{A\text{ aperto}\\A\subseteq\comp Y}}A=
	\comp{\left(\inters_{\substack{\comp A\text{ chiuso}\\\comp A\supseteq Y}}\comp A\right)}=
	\comp{\clos Y}\]
	Un conto analogo mostra che $\clos{\comp Y}=\comp{\inter Y}$.
	Quindi:
	\[\bound{\comp Y}=\clos{\comp Y}\setminus\inter{(\comp Y)}=\comp{\inter Y}\setminus\comp{\clos Y}=\clos Y\setminus\inter Y=\bound Y \qedhere\]
\end{proof}

% questa cosa l'aveva detta ma poi ha detto che è sbagliata il 24 febbraio
% \begin{oss}
% 	$x\text{ interno a }Y\implies x\text{ accumulazione per }Y$
% \end{oss}
% sposto qui la cosa del 24, con qualche modifica:
\begin{oss}
	Sia $X$ l'insieme ambiente e $x\in Y\subseteq X$.
	Si ha che:
	\[\text{$x$ interno a $Y$} \notimplies \text{$x$ di accumulazione per $Y$}\]
	Ad esempio per $(X, \tau_\text{disc})$ preso $Y\is\set{x}$ allora $x$ è interno a $Y$, poiché quest'ultimo è aperto, ma $\set{x}\setminus\set{x} = \nullset$ quindi non è di accumulazione.
	Tuttavia, basta supporre che i singoletti non siano aperti affinché invece l'implicazione sia valida.
\end{oss}

\titlet{Morfismi degli spazi topologici}

In generale, dati degli spazi con certe proprietà, i \emph{morfismi} sono funzioni tra gli spazi che mantengono in qualche modo le proprietà. Ad esempio i morfismi degli spazi vettoriali sono le funzioni lineari, che mandano 0 in 0, sottospazi in sottospazi, ecc.

In particolare gli \emph{isomorfismi} mettono in corrispondenza due spazi in modo che siano completamente indistinguibili relativamente alle proprietà considerate. Data una famiglia di spazi, si può definire su di essa un'equivalenza per isomorfismo e considerare le sottofamiglie di spazi \emph{isomorfi} come singoli elementi.

Vediamo ora che i naturali morfismi degli spazi topologici sono le funzioni \emph{continue}. Siano da qui in poi $X$, $Y$ spazi topologici\footnote{Identificheremo, quando non ci siano ambiguità, lo spazio topologico $(X,\tau)$ con $X$ o con $\tau$. Si noti che comunque $\tau$ contiene tutta l'informazione perché $X=\union\tau$.} e $\fundef XY$.

\begin{defn}[Continuità]
	$f$ si dice \emph{continua} se le controimmagini di aperti sono aperte:
	\[f\text{ continua}\means
	\forall A\subseteq Y,\,A\text{ aperto}:f^{-1}(A)\text{ aperto}\]
\end{defn}

Possiamo definire una forma \emph{locale} di continuità e collegarla alla nozione globale:

\begin{defn}[Continuità locale]
	$f$ è \emph{continua in $x\in X$} se ogni intorno di $f(x)$ contiene l'immagine di un intorno di $x$:
	\[f\text{ continua in }x\means
	\forall\, U_{f(x)}\,\exists\, U_x:f(U_x)\subseteq U_{f(x)}\]
\end{defn}

\begin{prop}
	$f\text{ continua}\iff\forall x\in X:f\text{ continua in }x$
\end{prop}

\begin{proof}
	Mostriamo le due implicazioni:
	\begin{description}
		\item[\proofrightarrow]
			$\forall x\in X\,\forall U_{f(x)}$ definiamo $U_x\is f^{-1}(U_{f(x)})$.
			Poiché $f^{-1}$ manda aperti in aperti, $U_x$ è un intorno di $x$,
			e ovviamente $f(U_x)=U_{f(x)}$.
		\item[\proofleftarrow]
			Sia  $A\subseteq Y$, $A$ aperto.
			Poiché $A$ è un intorno dei suoi punti, abbiamo che
			$\forall x\in f^{-1}(A)\,\exists U_x:f(U_x)\subseteq A$ cioè
			$U_x\subseteq f^{-1}(A)$.
			Il \autoref{th:apertointorni} conclude.
			\qedhere
	\end{description}
\end{proof}

Definiamo gli isomorfismi topologici, che chiameremo \emph{omeomorfismi}:

\begin{defn}[Omeomorfismo]
	$f$ è un \emph{omeomorfismo} se è continua, bigettiva e con inversa continua:
	\[f\text{ omeomorfismo}\means
	\begin{cases}
		f\text{ bigettiva }\\
		f, f^{-1}\text{ continue}
	\end{cases}\]
\end{defn}

È necessario specificare che $f^{-1}$ sia continua perché la continuità di una funzione invertibile non implica la continuità dell'inversa:

\begin{es}
	Consideriamo due spazi topologici sullo stesso insieme $(X,\tau_1)$, $(X,\tau_2)$.
	Sia $\tau_1$ più fine di $\tau_2$, cioè $\tau_1\supset\tau_2$.
	Sia $\fundef[\id]XX$ l'identità.
	$\id$ è invertibile e continua rispetto a $\tau_1\funarrow\tau_2$,
	ma non continua rispetto a $\tau_2\funarrow\tau_1$.
\end{es}

\titlet{Separazione}

Per adesso gli spazi topologici sono un concetto molto generale. Cominciamo a definire delle proprietà che ci permettano di avere un'idea più intuitiva di come sono fatti certi spazi topologici.

\begin{defn}[Proprietà di separazione]
	Uno spazio topologico si dice \emph{di Hausdorff}\footnote{Preferiremo la notazione $T_2$ perché ci da fastidio dire che uno spazio topologico sia ``di Hausdorff''.} o \emph{separato} o \emph{$T_2$} se per ogni coppia di punti distinti esistono due intorni disgiunti:
	\[X\text{ è }T_2\means
	\forall x\neq y\,\exists U_x,U_y:U_x\cap U_y=\nullset\]
\end{defn}

\begin{prop}
	Gli spazi metrizzabili sono separati:
	\[X\text{ metrizzabile}\implies X\text{ è }T_2\]
\end{prop}

\begin{proof}
	$x\neq y\so r\is d(x,y)\neq 0\so\ball[r/3]x\cap\ball[r/3]y=\nullset$
\end{proof}

\begin{prop}
	Negli spazi separati i singoletti sono chiusi:
	\[X\text{ è }T_2\implies\forall x\in X:\set{x}\text{ chiuso}\]
\end{prop}

\begin{proof}
	Abbiamo che $\forall y\in\comp{\set{x}}\,\exists U_y,U_x:U_y\cap U_x=\nullset$
	cioè in particolare $U_y\subseteq\comp{\set x}$.
	Quindi $\comp{\set x}$ è aperto.
\end{proof}

Vediamo che non vale il viceversa:

\begin{es}[Topologia di Zariski]
	Sia $\tau_z$ una topologia su $\R$ i cui aperti sono i complementari di insiemi finiti:
	\[\tau_z\is\Setdef[A\subset\R]{A=\nullset\vel(A=\comp F\et\card F\in\N)}\]
	Gli aperti non vuoti non sono disgiunti, quindi $(\R,\tau_z)$ non è $T_2$.
	Però i singoletti sono chiusi.
\end{es}

\begin{oss}
La topologia euclidea è più fine di $\tau_z$.
\end{oss}

\begin{prop}
	La proprietà $T_2$ passa ai sottospazi.
\end{prop}

\begin{proof}
	Sia $\chi\subseteq X$.
	$\forall x\neq y\in X\,\exists U_x,U_y:U_x\cap U_y=\nullset$,
	per $x,y\in\chi$ prendiamo come intorni $U_x\cap\chi$ e $U_y\cap\chi$.
\end{proof}

\begin{prop}
	La proprietà $T_2$ è invariante per omeomorfismo.
\end{prop}

\begin{proof}
	Sia $\fundef XY$ un omeomorfismo. Mostriamo che $Y\text{ è }T_2\implies X\text{ è }T_2$:
	\begin{align*}
		&\forall x_1\neq x_2\in X:\\
		&y_1\is f(x_1),\ y_2\is f(x_2)\\
		f\text{ iniettiva}\so&y_1\neq y_2\\
		Y\text{ è }T_2\so&\exists U_{y_1},U_{y_2}\text{ aperti}:U_{y_1}\cap U_{y_2}=\nullset\\
		&U_{x_1}\is f^{-1}(U_{y_1}),\ U_{x_2}\is f^{-1}(U_{y_2})\\
		U_{y_1}\cap U_{y_2}=\nullset\so&U_{x_1}\cap U_{x_2}=\nullset\\
		f\text{ continua}\so&U_{x_1}, U_{x_2}\text{ sono aperti}
	\end{align*}
	Applicando lo stesso ragionamento a $f^{-1}$ si mostra che $X\text{ è }T_2\implies Y\text{ è }T_2$.
\end{proof}

\begin{oss}
	In generale le proprietà che dipendono solo dalla topologia sono invarianti per omeomorfismo, perché un omeomorfismo induce una bigezione tra le topologie tale che due aperti corrispondenti sono messi in bigezione dall'omeomorfismo.
\end{oss}

\begin{es}
	Sia $\fundef[\id]\R\R$ l'identità e $\tau_E$ la topologia euclidea.
	$\id$ è continua rispetto a $\tau_E\funarrow\tau_z$ perché $\tau_z\subseteq\tau_E$,
	però non nell'altro verso.
	Infatti $(\R,\tau_E)$ è $T_2$ ma $(\R,\tau_z)$ no,
	quindi non possono essere omeomorfi.
\end{es}

\titlet{Numerabilità}

Di solito con insieme numerabile si intende un insieme con la cardinalità dei numeri naturali. Per brevità di notazione chiameremo numerabili anche gli insiemi finiti:

\begin{defn}[Numerabilità]
	$S\text{ numerabile}\means\card S\leq\card\N$
\end{defn}

\begin{defn}[Proprietà di numerabilità]
	Uno spazio topologico è \emph{1-numerabile} se ogni punto ha una base di intorni numerabile:
	\[X\text{ 1-numerabile}\means
	\forall x\in X\,\exists\,\text{base di intorni di $x$ numerabile}\]
\end{defn}

\begin{prop}
	$X$ metrizzabile $\implies X$ 1-numerabile
\end{prop}

\begin{proof}
	Avevamo già osservato che le palle aperte centrate in un punto formano una base di intorni per il punto.
	Ci basta scegliere la sottofamiglia numerabile $\setdef[{\ball[1/n]x}]{n\in\N}$,
	che è ancora una base perché $\forall r\,\exists n:\ball[1/n]x\subseteq\ball x$.
\end{proof}

\begin{defn}[Proprietà di numerabilità]
	Uno spazio topologico è \emph{2-numerabile} se ha una base di aperti numerabile:
	\[X\text{ 2-numerabile}\means\exists\,\text{base di aperti di $X$ numerabile}\]
\end{defn}
