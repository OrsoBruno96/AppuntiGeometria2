%autore: Andrea


% questa parte si può anche dare per nota, la commento
% In questa sezione si userà la nozione di immagine e di preimmagine di un sottoinsieme. In generale siano dati $X$ e $Y$ insiemi, $\fundef[f]{X}{Y}$ una funzione fra essi.\\
% Dato un sottoinsieme $X'$ di $X$ definisco immagine di $X'$ secondo $f$: \marginpar{come lo uso qui l'ambiente \texttt{defn}?}
% \begin{equation*}
% \img_f(X') \is f(X') \is \setdef[y \in Y]{\exists x \in X' \et f(x) = y}
% \end{equation*}
% Dualmente sia $Y'\subseteq Y$ definisco preimmagine di $Y'$ : \marginpar{dovrò ricordarmi di non saltare tra preimmagine e controimmagine}
% \begin{equation*}
% \img_f^{-1}(Y') \is f^{-1}(Y') \is \setdef[x\in X]{f(x)\in Y'}
% \end{equation*}
% Si noti che la definizione è coerente anche non chiedendo che $f$ sia invertibile!
% % \vspace{.8cm} ma non andrebbero messi questi spazi, se ti serve una spaziatura grossa per ragioni logiche forse non stai fittando il contenuto con la struttura titoli-paragrafi-teoremi-formule, ma vabbé non ho voglia di discuterne in realtà

\subtitlet{Compattezza e funzioni continue}

\begin{lemma}
	L'immagine continua di un compatto è compatta.\footnotemark
	\footnotetext{Non stiamo dimostrando che l'immagine continua di $T_2$ è $T_2$.}
\end{lemma}

\begin{proof}
	Sia $X$ compatto e $\fundef XY$ continua. Vogliamo dimostrare che $f(X)$ con la topologia di sottospazio è compatto.

	Prendiamo dunque un ricoprimento aperto $\mathcal A$ di $f(X)$.
	Consideriamo $\mathcal A_X\is\setdef[f^{-1}(A)]{A\in\mathcal A}$.
	Per definizione di continuità, gli elementi di $\mathcal A_X$ sono aperti, mentre per ragioni insiemistiche ricoprono $X$.
	Allora da $\mathcal A_X$ posso estrarre un sottoricoprimento finito $\mathcal A_X^F$.
	Tornando indietro, l'insieme $\setdef[f(A)]{A\in\mathcal A_X^F}$ è un sottoricoprimento finito di $\mathcal A$.
\end{proof}

\begin{lemma}
	L'immagine continua di un compatto per successioni è compatta per successioni.
\end{lemma}

\begin{proof}
	Siano $X$ compatto per successioni e $\fundef XY$ continua, vogliamo dimostrare che $f(X)$ è compatto per successioni.
	
	Sia $(y_n)_{n\in N}$ una successione in $Y$. Per ognuno degli $y_n$ scelgo un $x_n\in f^{-1}(y_n)$, definendo così una successione in $X$.
	Estraggo da essa una sottosuccessione convergente $(x_{n_k})$ (esiste poiché $X$ è compatto per successioni).
	Tornando indietro, $(f(x_{n_k}))$ è una sottosuccessione di $(y_n)$ e converge per continuità di~$f$.
\end{proof}

\begin{lemma}
	Se $\fundef X\R$ è continua e $X$ è compatto allora $f$ ammette massimo e minimo assoluti.
\end{lemma}

\begin{proof}
	L'immagine continua di compatti è compatta, dunque $f(X)$ è compatto. Ma in $\R$ compatto equivale a chiuso e limitato. Dalla limitatezza segue che esiste, in $\R$, estremo superiore e inferiore, dalla chiusura che essi sono in $f(X)$.
\end{proof}

\begin{lemma}
	Se $X$ è compatto e $\fundef XY$ è continua e invertibile, allora $f^{-1}$ è continua (corollario: è un omeomorfismo).
\end{lemma}

\begin{proof}
	Per prima cosa mostriamo che $f$ è chiusa (cioè manda chiusi in chiusi).
	Sia $C\subseteq X$ chiuso.
	$X$ è compatto quindi $C$ è compatto.
	L'immagine continua di compatti è compatta quindi $f(C)$ e $Y=f(X)$ sono compatti.
	$f(C)$ è un sottospazio compatto di $Y$ quindi è chiuso.
	
	Sia ora $A\subseteq X$ aperto.
	Il suo complementare $\comp A$ è un chiuso di $X$, allora la sua immagine $f(\comp A)$ è un chiuso di $Y$ ovvero $\comp{f(\comp A)}$ è aperto.
	Per invertibilità, $\comp{f(\comp A)}=f(A)$.
	Abbiamo quindi che $f^{-1}$ è continua.
\end{proof}

\titlet{Topologia prodotto e topologia quoziente}

Definiamo ora delle operazioni fra spazi topologici che ci permettano di costruirne altri in maniera ``naturale''.

\begin{defn}[Topologia prodotto]
	Siano $X$, $Y$ spazi topologici.
	La \emph{topologia prodotto} è la topologia meno fine su $X\times Y$ tale che le proiezioni canoniche $(x,y)\mapsto x$ e $(x, y)\mapsto y$	sono continue.
\end{defn}

\begin{oss}
	Si vede come la condizione sulle proiezioni non sia impossibile da soddisfare. Infatti la topologia discreta sul prodotto cartesiano soddisfa banalmente.
\end{oss}

\begin{fat}
	La topologia prodotto esiste ed è generata dalle controimmagini secondo le proiezioni canoniche di aperti di $X$ e di $Y$,
	ovvero dalla base di aperti:
	\[\setdef[A_X\times A_Y]{\text{$A_X$ aperto di $X$}\et\text{$A_Y$ aperto di $Y$}}\]
\end{fat}

\begin{fat}
	Proprietà comuni di compattezza, connessione, $T_2$, 1- o 2-numerabilità si propagano allo spazio prodotto.
\end{fat}

\begin{oss}
	Data una funzione surgettiva $\fundef XY$, la relazione $a\sim b\means f(a)=f(b)$ è di equivalenza, e l'insieme quoziente si identifica con $Y$ perché $\quoset X\sim=\setdef[f^{-1}(\set{y})]{y\in Y}$.
	
	Viceversa, data una relazione di equivalenza $\sim$ su $X$, la proiezione che manda un elemento nella sua classe di equivalenza
	\[f\is\fundef[{(x\mapsto [x]_\sim)}]X{\quoset X\sim}\]
	ha la proprietà che $a\sim b\iff f(a)=f(b)$.
\end{oss}

\begin{defn}[Topologia quoziente]
	Siano $X$ uno spazio topologico e $\sim$ una relazione di equivalenza su $X$.
	La \emph{topologia quoziente} è la topologia più fine su $\quoset X\sim$ tale che la proiezione $x\mapsto[x]_\sim$ è continua.
\end{defn}

\begin{oss}
	Anche questa volta, se l'esistenza di topologie che rendano continua l'immersione è palese (basti pensare alla topologia banale), non è affatto scontato che esista una ``topologia più fine possibile''.
\end{oss}

\begin{defn}[Insieme $f$-saturo]
	Data una funzione $\fundef{X}{Y}$ si dice $f$-saturo un $X'\subseteq X$ tale che $f^{-1}(f(X'))=X'$.
\end{defn}

\begin{fat}
	La topologia quoziente su $\quoset X\sim$ esiste ed è data dalle immagini attraverso $f\is(x\mapsto[x]_\sim)$ di aperti $f$-saturi di $X$.
\end{fat}

\begin{fat}
	Le proprietà di connessione e compattezza si propagano al quoziente.
	\marginpar{Non si propaga anche $T_2$?}
\end{fat}

% questo esercizio non c'era a lezione però poi viene implicitamente usato questo fatto
% e si dimostra facilmente partendo dalla def. con gli f-saturi
\begin{ex}
	La topologia quoziente è data dagli insiemi la cui controimmagine attraverso la proiezione è aperta.
\end{ex}

% questa parte è un'anticipazione della lezione successiva
%
% \titlet{Varietà topologiche}
%
%
% Ora anticipiamo un po' di quello che faremo la prossima volta (non so, forse lo ha fatto solo per rendere un po' meno organiche queste note!)!!
%
% \begin{defn}[Varietà topologica $n$-dimensionale]
% Fissato un intero $n$ una varietà topologica $n$-dimensionale è uno spazio topologico $V$ $T_2$ localmente omeomorfo ad aperti di $R^n$.
%
% Più propriamente si definisce varietà topologica la coppia $(V, M)$ dove $V$ è lo spazio topologico e $M$ è l'insieme degli omeomorfismi.
% \end{defn}
