%Marco Costa

\marginpar{DA FINIRE}

\begin{oss}
 $(g\circ h )_* = g_*\circ h_*$,
 $\id_* = \id$
\end{oss}
$g_*\left(\left[\Fundef[f]X{Y_1} \right]\right) = \left[\Fundef[g\circ f]X {Y_2} \right]$
\begin{oss}
 Se $\fundef[g]{Y_1}{Y_2}$ è diffeomorfismo, $\exists \fundef[g^{-1}]{Y_2}{Y_1}$ liscia $\implies g_*$ è isomorfismo di spazi vettoriali.
\end{oss}
\begin{defn}[Omotopia]
 $\fundef[g_0]{Y_1}{Y_2}$, $\fundef[g_1]{Y_1}{Y_2}$ sono \emph{omotope} se $\exists \fundef[G]{Y_1\times\left[ 0,1\right]}{Y_2}$ tale che,
 detta $G_t\is  G|_{Y_1\times \set t}, G_0=g_0, G_1=g_1$.
\end{defn}
 \begin{oss}
  L'omotopia è un particolare cobordismo (basta considerare il cilindro associato!), in cui le varietà che costituiscono il bordo sono la stessa, ma la funzione verso la varietà obiettivo è differente.
 \end{oss}
\begin{prop}
 $\fundef[g_0, g_1]{Y_1}{Y_2}$ omotope$\implies g_{0*}=g_{1*}$.
\end{prop}
\begin{proof}
 \begin{figure}
  \centering
  \input{homotopy.pdf_tex}
  \caption{dimostrazione teorema}
 \end{figure}
 Per dimostrare che le due funzioni sono uguali verifico che valutate in ogni punto sono uguali, dove un punto di $\eta_n(Y_1)$ è una varietà con la sua funzione a valori in $Y_1$, e l'uguaglianza è ai sensi del cobordismo, quindi se in ogni punto sono cobordanti.

 Costruisco ora il cobordismo. Poiché le valuto nello stesso punto, in partenza avrò la coppia $(X,f)$ con $X$ varietà e $\fundef[f]{X}{Y_1}$ liscia.

 Ho già un cobordismo tra $(Y_1, g_1)$ e $(Y_1, g_2)$, realizzato dal cilindro di base $Y_1$ e come funzione sul cilindro l'omotopia $G$ che congiunge $g_1$ a $g_2$. Considero inoltre, in partenza, il cilindro di base $X$, e lo mando nell'altro cilindro mediante la funzione definita da $\forall t\in \left[ 0,1\right], F(x,t)=(f(x,t), t)$. Componendo le funzioni sui due cilindri ottengo un cobordismo fra $g_{1*}((X,f))$ e $g_{1*}((X,f))$, valido per ogni coppia $(X,f) \in \eta_n(Y_1)$
  \begin{figure}
  \centering
  \input{homotopy2.pdf_tex}
  \caption{$G\circ F (W) = Y_2$}
 \end{figure}
\end{proof}
\begin{oss}
 $\eta_S (\R ^n) = \eta (punto)_S = \eta_S $
\end{oss}


\begin{defn}[Cobordismo orientato]
 $X_0\sim_{cob^+}X_1$ se $\exists$ triade $(W, Z_0, Z_1)$ con $W, \boundary W$ orientata e isomorfismi $\fundef[\phi_0]{X_0}{Z_0}, \fundef[\phi_1]{X_1}{Z_1}$
 che preservano l'orientazione.
\end{defn}
\begin{defn}
 $\Omega _n(punto)=\Omega _n = \set{ X \mbox{$n$-varietà compatta, chiusa, orientata}} /\sim_{cob^+}$
\end{defn}

\begin{oss}
 Anche in questo caso possiamo equipaggiare $\Omega _n$, con la stessa operazione definita nella lezione precedente.
 In particolare valgono:
 \begin{itemize}
  \item $\left [ X \right ] + \left [ Y \right ] = \left [ X \djcup Y \right ]$
  \item 0 = $\left [ \nullset \right ]$ ossia i bordi orientati.
  \item $-\left[ X\right]=\left[-X\right]$. (Viene scambiata l'orientazione di tutte le componenti connesse).
 \end{itemize}
 Così $(\Omega _n, +)$ assuma la struttura di gruppo abeliano.
\end{oss}
Possiamo adesso ripercorrere lo schema precedente e cambiare la varietà obiettivo.
\begin{defn}
 $\Omega_n (Y) = \set{\mbox{$X$ $n$-varietà liscia tale che} \exists \fundef XY \mbox{liscia}}$
\end{defn}
\begin{oss}
 Non ho chiesto nessuna condizione di orientabilità sulla varietà $Y$ in arrivo.
\end{oss}
Adesso ripetiamo la costruzione e creiamo degli omomorfismi di gruppi $\fundef[g_*]{\Omega _n (Y_1)}{\Omega _n(Y_2)}$ tale che $(g\circ h )_* = g_*\circ h*$,
 $\id_* = \id$.
\begin{teo}
 Vale la proprietà di omotopia
\end{teo}
\marginpar{Non so cosa significhi ma sugli appunti c'è scritto così}
Questi si chiamano oggetti topologico-algebrici e servono a fare i conti. Un esempio di questi è il seguente.
\titlet{Gruppo fondamentale}
$y_0\in Y$ varietà connessa per archi. Considero la coppia $(Y, y_0)$
\begin{defn}[Omotopia puntata]
Una omotopia $F$ è detta \emph{puntata} (o \emph{o.p.}) se $F(P, t) = y_0 \forall t$
 \begin{figure}
  \centering
  \input{puntata.pdf_tex}
  \caption{Esempio nel caso si sia preso $(S^1, P).$}
 \end{figure}

\end{defn}
Considero ora l'insieme $\set{\fundef {(S^1, P)}{(Y, y_0)}, f(P) = y_0}/\sim_{o.p.}$, ossia considero i cammini chiuso a meno di omotopia puntata.
Lo doto di una operazione in modo da renderlo un gruppo.
\begin{defn}
 $f_0 * f_1:(S^1, P)\rightarrow (Y, y_0)\begin{cases} f_1(2t), & \mbox{se } t\in \left [0, 1/2\right ] \\ f_0(2t-1), & \mbox{se }t\in \left [1/2, 1\right ]\end{cases}$.
 In pratica è percorrere i 2 cammini in successione a velocità doppia.
 E' ben definita e $\left [ f_0 \right] *\left [ f_1 \right] = \left [ f_0 * f_1 \right]$
 \marginpar{Questaa definizione di concatenazione l'ho presa da wiki, anche se mi pare Benedetti l'abbia fatta identica a lezione}
\end{defn}
\begin{oss}
 Moralmente il partire da $S^1$ sente se $\R ^2$ ha buchi (stiamo "lanciando" nella varietà di arrivo un laccio chiuso).
 Se partivo da un punto, avrei sentito solo la connessione per archi, non la semplice connessione.
\end{oss}
\begin{defn}[Gruppo fondamentale]
 L'insieme $\pi_1(Y, y_0)=\set{\fundef{(S^1, P)}{(Y, y_0)}, f(P) = y_0}/\sim_{o.p.}$ è detto \emph{gruppo fondamentale} della varietà $Y$.
\end{defn}

