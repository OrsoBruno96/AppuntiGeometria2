%appunti trascritti la sera del venerdì 4 Marzo 2016 da Belliardo Federico.

\titlet{Ripasso e complementi sul proiettivo}

\begin{defn}[Spazio proiettivo]
Dalla scorsa lezione sappiamo che posso definire lo spazio proiettivo 
$\mathbb{P}^n$ dalle funzioni $\fundef[\pi]{\R^{n+1}}{\quoset {\R^{n+1}}{\sim}}$ o $\fundef[\pi_S]{S^n \subseteq \R^{n+1}}{\quoset {S^{n}}{\sim}}$. Dove gli insiemi quozienti sono omeomorfi (come spazi topologici) e quindi definiscono entrambi lo stesso spazio topologico $\mathbb{P}^{n}$.
\end{defn}
\begin{oss}
Intuitivamente gli aperti della prima topologia sono i coni di centro nell'origine e gli aperti della seconda sono le intersezioni di questi coni con la sfera unitaria. Dunque abbiamo una proiezione che è corrispondenza biunivoca tra gli spazi topologici che induce una corrispondenza biunivoca tra gli aperti.
\end{oss}
\begin{oss}
$\mathbb{P}^{n}$ è $T_{2}$ e 2-numerabile, queste proprietà passano al quoziente perché la proiezione $\pi$ è una mappa aperta (manda aperti in aperti).
\marginpar{Sarebbe bello dimostrarlo}
\end{oss}
\begin{oss}
$\mathbb{P}^{n}$ è compatto perché è immagine di $S^{n}$ tramite una mappa continua.
\end{oss}

\titlet{Sfera come $n$-varietà}

\begin{prop}
Si può costruire un atlante $\mathcal{A} = \set{{(U_1, f_1), (U_2, f_2), \dots, (U_{n+1}, f_{n+1})}}$.
\end{prop}
\begin{proof}
Considero il piano $x_{n+1} = 1$. La retta passante per l'origine degli assi e per un punto della sfera appartenente alla calotta con $x_{n+1} > 0$ interseca il piano suddetto in un punto. Il piano è isomorfo a $\R^n$ e questa procedura stabilisce una biezione tra i punti della calotta superiore e i punti del piano. Ripeto la costruzione per la calotta inferiore. 
Devo costruire una carta locale anche per l'equatore ma esso è isomorfo a una sfera $S^{n-1}$ dunque posso procedere iterativamente, finché non arrivo a $S^{0}$. 
\marginpar{e poi come faccio a costruire una mappa per $S^{0}$?}
\end{proof}
Dunque si riesce a dotare $S^{n}$ di un atlante, essa è poi 2-numerabile e $T_{2}$ poiché queste proprietà passano ai sottospazi topologici.

\titlet{Varietà differenziabili}

\begin{prop}[Prodotto di varietà topologiche]
Date $M_{1}$ e $M_2$ $n_1$-varietà e $n_2$-varietà topologiche $M_1\times M_2$ è una \emph{$(n_1+n_2)$-varietà}.
\end{prop}
\begin{proof}
Devo esibire un atlante. Se $\mathcal{A}_{1} = {(U_i, f_i)}$ è un atlante per $M_1$ e $\mathcal{A}_{2} = {(U_i, f_i)}$ è un atlante per $M_2$ allora il prodotto $\mathcal{A}_{1} \times \mathcal{A}_{2} = {(U_i \times U_j, f_i \times f_j)}$ è un atlante per il prodotto. Il prodotto di funzioni è definito come segue: sia $\fundef[f_i]{U_i}{A_i}$ e $\fundef[f_j]{W_j}{B_j}$ allora $\fundef[f_i\times f_j]{U_{i} \times W_{j}}{A_{i} \times B_{j}}$ è la funzione tale che $(p, q) \mapsto f_i(p) \times f_j(q)$  .
$T_2$ e 2-numerabilità sono proprietà che passano al prodotto topologico.
\end{proof}

\begin{oss}
La topologia sul prodotto cartesiano degli insiemi è quella meno fine che rende le proiezioni $\fundef[f]{M_1 \times M_2}{M_1}$ e $\fundef[f]{M_1 \times M_2}{M_2}$ continue.
\end{oss}
\begin{oss}
$S^p \times S^q$ è una $(p+q)$-varietà, proprio come $S^1 \times S^1 \times S^n \times \ldots S^1$ è chiamato $n$-toro ed è anche una $n$-varietà.
\end{oss}
\begin{defn}
$\fundef[f]{M_1}{M_2}$ con $f$ continua si dice morfismo tra spazi topologici.
\end{defn}
L'obbiettivo è studiare gli spazi topologici a meno di applicazioni continue.
\begin{oss} 
$S^n$ è omeomorfo a $S^m$? $S^n$ e $\mathbb{P}^n$ sono omeomorfi? 
Sono domande di non facile risposta\dots
\end{oss}
\begin{es}[Curva di Peano]
Definisco una funzione $\fundef[f]{[0,1]}{[0,1]\times[0,1]}$ continua e surgettiva. Dunque i due insiemi di partenza e di arrivo con la topologia euclidea sono omeomorfi.
\end{es}
\begin{defn}[Differenziabilità in un punto]
$\fundef[f]{\R^n}{\R^k}$ è differenziabile in $x_0$ se esiste un applicazione lineare $\fundef[L]{\R^n}{\R^k}$ tale che $\lim_{h\to 0}{\frac{f(x_0+h)-f(x_0)-L(h)}{\Vert h \Vert}}=0$
Se $L$ esiste è necessariamente unica.
\end{defn}
\begin{defn}
$\fundef[f]{\Omega \subseteq \R^n}{\R^m}$ è differenziabile in $\Omega$ se è differenziabile in ogni suo punto.
\end{defn}
\begin{defn}[Funzione differenziale]
$\fundef[\dd f]{\Omega}{\Hom(\R^n, \R^m)} \approx \R^{nm}$ è chiamata differenziale della funzione $\fundef[f]{\Omega \subseteq \R^n}{\R^m}$.
\end{defn}
\begin{oss}
Posso costruire induttivamente $\dd(\dd f) = \dd^2f$, $\dd(\dd^2f) = \dd^3f$ e così via.
\end{oss}
\begin{oss}
$f$ è $n$-differenziabile e posso costruire induttivamente le funzioni sopra mostrate fino alla $n$-esima.
\end{oss}
\begin{defn}
$f$ è $C^{\infty}$ se è $k$-differenziabile $\forall k > 0$.
\end{defn}
\begin{defn}
Dall'analisi reale sappiamo che $f$ è $C^{\infty}$ se tutte le derivate parziali esistono e sono continue
\end{defn}
\begin{defn}[Diffeomorfismo]
$\fundef[f]{\Omega \subseteq \R^n}{\Omega' \subseteq \R^n}$ è un diffeomorfismo se è $C^\infty$ ed è invertibile tale che $f^{-1}$ è $C^{\infty}$.
\end{defn}
\begin{prop}
Se $\fundef[f]{\R^n}{\R^m}$ è diffeomorfismo allora $n = m$, cioè la dimensione è un invariante per diffeomorfismo.
\end{prop}
\begin{proof}
Sia $g$ la funzione inversa di $f$. allora $g \circ f = \id$ e $f \circ g = \id$, in particolare vale che $\dd(f \circ g) = \dd f \circ \dd g = \id$ (dall'analisi reale). Dunque valutando il differenziale in un punto ottengo che le due matrici che sono i differenziali della funzione $f$ e della sua inversa sono una l'inversa dell'altra ma per avere un'inversa la matrice deve essere quadrata, dunque il diffeomorfismo conserva le dimensioni.
\end{proof}
\begin{defn}[$n$-varietà topologica differenziabile]
Sia un atlante $\mathcal{A} = {(U_i, f_i)}$. Considero gli omeomorfismi
$\fundef[f_i]{U_i}{A_i \subseteq \R^n }$ e $\fundef[f_j]{W_j}{B_j\subseteq \R^n}$ e l'intersezione 
$U_i \cap U_j$. Dico che $\mathcal{A}$ è un atlante differenziabile se ogni funzione $\fundef[f_k \circ f_j^{-1}]{f_j(U_i \cap U_j)}{f_i(U_i \cap U_j)}$ è $C^{\infty}$. Essa è evidentemente una funzione da $\R^n$ a $\R^n$ ed è anche invertibile infatti $f_j \circ f_k^{-1}$ è la sua inversa.
\end{defn}
\begin{defn} 
Data una $n$-varietà M (topologica) due atlanti differenziabili si dicono compatibili se $\mathcal{A}_1 \cup \mathcal{A}_2$ è ancora un atlante differenziabile.
\end{defn}
\begin{oss}
Se le carte non si intersecano l'atlante è differenziabile
\end{oss}
\begin{oss}
Potrei avere più di un atlante differenziabile, l'unione di tutti gli atlanti differenziabili è l'atlante massimale per quella struttura di $n$-varietà liscia.
\end{oss}
\begin{oss}
Data una $n$-varietà topologica esiste sempre un atlante differenziabile massimale?
\end{oss}