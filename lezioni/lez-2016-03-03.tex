%appunti trascritti la sera del venerdì 4 Marzo 2016 da Belliardo Federico.

% questa parte si sovrappone totalmente alla lezione precedente, la commento
% \titlet{Ripasso e complementi sul proiettivo}
%
% \begin{defn}[Spazio proiettivo]
% Dalla scorsa lezione sappiamo che posso definire lo spazio proiettivo
% $\mathbb{P}^n$ dalle funzioni $\fundef[\pi]{\R^{n+1}}{\quoset {\R^{n+1}}{\sim}}$ o $\fundef[\pi_S]{S^n \subseteq \R^{n+1}}{\quoset {S^{n}}{\sim}}$. Dove gli insiemi quozienti sono omeomorfi (come spazi topologici) e quindi definiscono entrambi lo stesso spazio topologico $\mathbb{P}^{n}$.
% \end{defn}
% \begin{oss}
% Intuitivamente gli aperti della prima topologia sono i coni di centro nell'origine e gli aperti della seconda sono le intersezioni di questi coni con la sfera unitaria. Dunque abbiamo una proiezione che è corrispondenza biunivoca tra gli spazi topologici che induce una corrispondenza biunivoca tra gli aperti.
% \end{oss}
% \begin{oss}
% $\mathbb{P}^{n}$ è $T_{2}$ e 2-numerabile, queste proprietà passano al quoziente perché la proiezione $\pi$ è una mappa aperta (manda aperti in aperti).
% \marginpar{Sarebbe bello dimostrarlo}
% \end{oss}
% \begin{oss}
% $\mathbb{P}^{n}$ è compatto perché è immagine di $S^{n}$ tramite una mappa continua.
% \end{oss}

% questo atlante per la sfera mi sembra sbagliato, lo commento
% cosa mi sembra sbagliato: quando devo costruire una carta intorno all'equatore (che è chiuso),
% uso le carte dell'equatore come S^{n-1}, quindi andrebbero in qualche modo estese ad aperti di S^n.
% comunque la figura è un poccio
% \titlet{Sfera come $n$-varietà}
%
% \begin{prop}
% Si può costruire un atlante $\mathcal{A} = \set{{(U_1, f_1), (U_2, f_2), \dots, (U_{n+1}, f_{n+1})}}$.
% \end{prop}
% \begin{proof}
% Considero il piano $x_{n+1} = 1$. La retta passante per l'origine degli assi e per un punto della sfera appartenente alla calotta con $x_{n+1} > 0$ interseca il piano suddetto in un punto. Il piano è isomorfo a $\R^n$ e questa procedura stabilisce una biezione tra i punti della calotta superiore e i punti del piano. Ripeto la costruzione per la calotta inferiore.
% Devo costruire una carta locale anche per l'equatore ma esso è isomorfo a una sfera $S^{n-1}$ dunque posso procedere iterativamente, finché non arrivo a $S^{0}$.
% \marginpar{e poi come faccio a costruire una mappa per $S^{0}$?}
% \end{proof}
%
% \begin{figure}
%     \centering % per centrare l'immagine
%     \input{figura40.pdf_tex}
%     \caption{Proiezione stereografica della sfera}
% \end{figure}
%
% Dunque si riesce a dotare $S^{n}$ di un atlante, essa è poi 2-numerabile e $T_{2}$ poiché queste proprietà passano ai sottospazi topologici.

\titlet{Varietà prodotto}

\begin{prop}[Prodotto di varietà]
	Date $M_{1}$ e $M_2$ $n_1$-varietà e $n_2$-varietà,
	$M_1\times M_2$ è una $(n_1+n_2)$-varietà.
\end{prop}

\begin{proof}
	Devo esibire un atlante.
	Se $\set{(U_i, f_i)}$ è un atlante per $M_1$ e $\set{(W_j, g_j)}$ è un atlante per $M_2$ 
	allora $\set{(U_i \times W_j, f_i \times g_j)}$ è un atlante per il prodotto.
	Il prodotto di funzioni è definito come segue: sia $\fundef UA$ e $\fundef[g]WB$,
	allora $\fundef[f\times g]{U\times W}{A\times B}$ è la funzione $(p, q) \mapsto \big(f(p), g(q)\big)$.
	$T_2$ e 2-numerabilità sono proprietà che passano al prodotto topologico.
\end{proof}

% già detto
% \begin{oss}
% La topologia sul prodotto cartesiano degli insiemi è quella meno fine che rende le proiezioni $\fundef[f]{M_1 \times M_2}{M_1}$ e $\fundef[f]{M_1 \times M_2}{M_2}$ continue.
% \end{oss}

\begin{es}~
	\begin{itemize}
		\item $S^p \times S^q$ è una $(p+q)$-varietà.
		\item
			$\underbrace{S^1 \times \dots \times S^1}_\text{$n$ volte}$
			è chiamato \emph{$n$-toro} ed è una $n$-varietà.
	\end{itemize}
\end{es}

% già detto
% \begin{defn}
% $\fundef[f]{M_1}{M_2}$ con $f$ continua si dice morfismo tra spazi topologici.
% \end{defn}
% L'obbiettivo è studiare gli spazi topologici a meno di applicazioni continue.

\titlet{Varietà differenziabili}

\begin{oss}
	Gli omeomorfismi sono isomorfismi anche per le varietà.
\end{oss}

\begin{fat} 
	$S^n$ è omeomorfo a $S^m$ se e solo se $n=m$,
	e $S^n$ non è omeomorfo a $\mathbb{P}^n$,
	ma non è facile dimostrarlo.
\end{fat}

Gli omeomorfismi e in generale le funzioni continue possono essere molto lontani dalla nostra intuizione di ``continuo'':

\begin{es}[Curva di Peano]
	Esiste una funzione $\fundef[f]{[0,1]}{[0,1]\times[0,1]}$ continua e surgettiva.\footnotemark
	\footnotetext{Ma non un omeomorfismo.}
\end{es}

Dunque vogliamo definire degli oggetti che siano ``più continui''.

\begin{defn}[Differenziabilità in un punto]
	$\fundef[f]{\R^n}{\R^k}$ è differenziabile in $x$ se esiste un'applicazione lineare
	$\fundef[L]{\R^n}{\R^k}$ tale che:
	\[\lim_{h\to 0}{\frac{f(x+h)-f(x)-L(h)}{|h|}}=0\]
\end{defn}

\begin{fat}
	Se $L$ esiste è necessariamente unica.
\end{fat}

\begin{defn}[Differenziabilità in un insieme]
	$\fundef[f]{\Omega \subseteq \R^n}{\R^k}$ è differenziabile in $\Omega$
	se è differenziabile in ogni suo punto.
\end{defn}

\begin{defn}[Funzione differenziale]
	Data una funzione differenziabile $\fundef[f]{\Omega \subseteq \R^n}{\R^k}$,
	la funzione $\fundef[\de f]\Omega{\Hom(\R^n,\R^k)\simeq\R^{nk}}$
	che manda $x\mapsto L$
	è il \emph{differenziale} di $f$.
	Poiché il valore è una funzione, l'argomento si indica a pedice: $\de_x f$.
\end{defn}

% questa cosa in realtà stava negli appunti all'inizio della lezione successiva (7 marzo)
\begin{oss}
	Il differenziale è \emph{functoriale}, ovvero:
	\begin{itemize}
		\item $\de_x\id=\id$;
		\item $\de_x(f\circ g)=\de_{g(x)}f\circ\de_xg$;
	\end{itemize}
	le quali implicano $\de_x(f^{-1})=\de_{f(x)}f$.
\end{oss}

\begin{oss}
	Se il differenziale è differenziabile,
	posso costruire induttivamente $\de(\de f) = \de^2f$,
	e analogamente $\de(\de^2f) = \de^3f$ e così via.
\end{oss}

\begin{defn}
	$f$ è $n$-differenziabile se esiste $\de^n f$.
\end{defn}

\begin{defn}[Funzione liscia]
	$f$ è $C^{\infty}$ o \emph{liscia} se è $k$-differenziabile $\forall k > 0$.
\end{defn}

\begin{fat}
	Dall'analisi reale sappiamo che $f$ è $C^{\infty}$
	se tutte le derivate parziali esistono e sono continue.
\end{fat}

\begin{defn}[Diffeomorfismo]
	$\fundef[f]{\Omega \subseteq \R^n}{\Omega' \subseteq \R^k}$ è un \emph{diffeomorfismo}
	se è $C^\infty$ ed è invertibile con inversa $C^{\infty}$.
\end{defn}

\begin{prop}
	Se $\fundef[f]{\R^n}{\R^k}$ è diffeomorfismo allora $n = k$,
	cioè la dimensione è un invariante per diffeomorfismo.
\end{prop}

\begin{proof}
	Sia $g$ la funzione inversa di $f$.
	Allora $g \circ f = \id$ e $f \circ g = \id$,
	in particolare vale che $\de(f \circ g) = \de f \circ \de g = \id$ (dall'analisi reale).
	Dunque valutando il differenziale in un punto ottengo che le due matrici
	che sono i differenziali della funzione $f$ e della sua inversa
	sono una l'inversa dell'altra ma per avere un'inversa la matrice deve essere quadrata,
	dunque il diffeomorfismo conserva le dimensioni.
\end{proof}

\begin{defn}[Atlante differenziabile]
	Un atlante è \emph{differenziabile} se,
	per ogni coppia di carte $(U_i,\phi_i)$, $(U_k,\phi_k)$, si ha che la funzione
	$\phi_k\circ\phi_i^{-1}|_{\phi(U_i\cap U_k)}$ è liscia.\footnotemark
	\footnotetext{Per simmetria, è in effetti un diffeomorfismo.}
	\begin{figure}
    \centering
    \input{figura39.pdf_tex}
    \caption{Un atlante è differenziabile se il passaggio da una rappresentazione a un'altra è liscio.}
	\end{figure}
\end{defn}

\begin{oss}
	Se le carte non si intersecano l'atlante è differenziabile.
\end{oss}

\begin{defn}[Atlanti compatibili]
	Due atlanti differenziabili sulla stessa varietà sono \emph{compatibili}
	se la loro unione è ancora un atlante differenziabile.
\end{defn}

\begin{defn}[Atlante massimale]
	Per una certa classe di atlanti compatibili,
	l'unione di tutti gli atlanti possibili è l'\emph{atlante massimale}.
\end{defn}

\begin{defn}[Varietà differenziabile]
	Una varietà dotata di atlante differenziabile è una \emph{varietà differenziabile} o \emph{liscia}.
\end{defn}

\begin{oss}
	La struttura di varietà differenziabile si identifica con l'atlante massimale.
\end{oss}

\begin{fat}
	Il prodotto di varietà differenziabili è differenziabile.
\end{fat}

D'ora in poi chiameremo brevemente ``varietà'' le varietà lisce
e quando vorremo specificare che non è liscia la chiameremo ``varietà topologica''.

Data una varietà topologica esiste sempre un atlante differenziabile massimale? È unico?
Sono domande a cui non risponderemo.
