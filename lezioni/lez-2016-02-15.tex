% trascrizione Bob

%gnegne

\titlet{Definizioni di base}

\begin{defn}[Topologia]
Sia dato un insieme $X$ e una famiglia $\tau$ di sottoinsiemi di $X$. $\tau$ è detta \emph{topologia} se soddisfa le seguenti proprietà:
\begin{itemize}
\item $\nullset,X\in\tau$ 
\item$\tau$ è chiusa per intersezioni finite, ovvero: $\forall A_1, A_2 \in \tau \implies A_1 \cap A_2 \in \tau$
\item $\tau$ è chiusa per unioni arbitrarie, ovvero: $\forall \mathcal B \subseteq \tau \implies \bigcup_{A\in\mathcal B} A \in \tau$.
\end{itemize}
Gli elementi di $\tau$ sono detti \emph{aperti} per la topologia $\tau$.
\end{defn}

\begin{defn}[Spazio topologico]
Sia $X \neq \nullset$ un insieme munito di una topologia $\tau$. La coppia $(X,\tau)$ è detta \emph{spazio topologico}.
\end{defn}
\begin{defn}[Chiuso]
Un sottoinsieme $Y \subseteq X$ è detto \emph{chiuso} se $\comp Y$ è aperto.
\end{defn}

\begin{oss}
$\nullset$ e $X$ sono sia aperti che chiusi.
\end{oss}

\begin{prop}
Sia $F_\tau$ la famiglia dei sottoinsiemi chiusi di $X$. Questa gode delle seguenti proprietà:
\begin{itemize}
\item $\nullset ,X \in F_\tau$ 
\item $F_\tau$ è chiusa pe unioni finite
\item $F_\tau$ è chiusa per intersezioni arbitrarie.
\end{itemize}
\end{prop}

\begin{oss}
Un insieme $X\neq\nullset$ ammette sempre due topologie degeneri:
\begin{itemize}
\item $\tau_B = \{\nullset,X\}$ detta topologia banale
\item $\tau_D = \pset(X)$ detta topologia discreta.
\end{itemize}
\end{oss}

\begin{defn}[Finezza]
Date due topologie $\tau$ e $\tau'$ su $X$, si dice che $\tau$ è \emph{più fine} di $\tau'$ se $\tau'\subset\tau$, ovvero tutti gli aperti di $\tau'$ sono anche aperti di $\tau$, ma esistono aperti in $\tau$ non presenti in $\tau'$.
\end{defn}

\begin{oss}
	La topologia banale è quella meno fine, mentre quella discreta è la più fine.
\end{oss}

\titlet{Spazi metrici e metrizzabili}

\begin{defn}[Spazio metrico]
Sia dato un insieme $X\neq\nullset$ e una funzione $\fundef[d]{X \times X}\R$ tale che:
\begin{itemize}
\item $\forall x,y \in X:  d(x,y)\ge0 \et (d(x,y)=0 \iff x=y)$
\item $d(x,y) = d(y,x)$
\item $d(x,z) \le d(x,y) + d(y,z)$
\end{itemize}
La funzione $d$ è detta \emph{distanza} e la coppia $(X,d)$ è detta \emph{spazio metrico}.
\end{defn}

\begin{es}
	\label{es:spazimetrici}
Esempi di spazi metrici sono:
\begin{itemize}
\item  $\R^n$ munito della distanza euclidea $d_E$
%$\tau_E \is \sqrt { \sum_i { ( x_ i - y_i )^2 } = \sqrt { \left \langle x-y,x-y \right \rangle } $
\item Un insieme $X$ qualsiasi munito della distanza discreta $d_D= \begin{cases} 0 & x=y \\ 1 & x \neq y \end{cases}$
\end{itemize}
\end{es}

\begin{defn}[Palla aperta]
Sia $(X,d)$ uno spazio metrico, definiamo la \emph{palla aperta centrata in $x$ di raggio $r$} come:
\[\ball[r]x \is \setdef[y \in X]{d(x,y) < r}\]
\end{defn}

\begin{prop}[Topologia indotta]
Sia $(X, d)$ uno spazio metrico e $\tau_d$ una famiglia di sottoinsiemi così definita:
\[\tau_d \is \setdef[A \subseteq  X]{\forall x \in A \;\exists \ball[r]x \subseteq A}\]
Affermo che $\tau_d$ è una topologia su $X$ (la chiamerò \emph{topologia indotta}).
\end{prop}

\begin{proof}
Verifico che si tratti di una topologia:
\begin{itemize}
\item $\nullset \in \tau_d$ per vacuità, $X \in \tau_d$ perché le palle aperte sono a valori in $X$ per definizione.
\item Siano $A_1,\ldots,A_n$ sottoinsiemi di $X$. Considero la loro intersezione $\inters A_k$. Per ogni punto $x$ nell'intersezione considero tutti gli $A_k$ tali che $x \in A_k$. Prendo quindi una palla aperta $B^k(x)$ per ognuno di questi (posso farlo per definizione di $\tau_d$).
Tra queste prendo quella con raggio minore: questa è contenuta in $\inters A_k$, segue che $\tau_d$ è chiuso per intersezioni finite.
\item se $x \in \union A_j$ allora $x$ appartiene almeno ad uno degli $A_j$. Poiché questi sono aperti segue che  $\exists \ball[r_i]x \subseteq A_i \subseteq \union A_j$. \qedhere
\end{itemize}
\end{proof}

\begin{defn}
Uno spazio topologico $(X,\tau)$ si dice \emph{metrizzabile} se esiste una distanza $d$ tale che $\tau=\tau_d$, ovvero la topologia sia uguale alla topologia indotta dalla distanza $d$.
\end{defn}

\begin{es}
Esempi di topologie indotte sono:
\begin{itemize}
\item $(\R^n,d_E)$  induce $(\R^n,\tau_E)$.
\item Lo spazio topologico indotto da $(X,d_D)$ (dove $d_D$ è la distanza discreta dell'\autoref{es:spazimetrici}) è $\tau_D$ (la topologia discreta). Infatti $\ball[1/2]x = \{x\}$, quindi ogni punto è un aperto.
\end{itemize}
\end{es}

\begin{prop}
Sia $(X,d)$ uno spazio metrico e $(X,\tau_d)$ lo spazio topologico indotto. Si  verifica che:
\begin{itemize}
\item le apelle aperte sono aperti per la topologia indotta
\item ogni $A \in \tau_d$ è unione di palle aperte.
\end{itemize}
\end{prop}

\begin{proof}\noindent
\begin{itemize}
\item Sia $y \in \ball[r]x$, devo trovare un $r'$ tale che $\ball[r']x \subseteq \ball[r]x$. Prendo $r' < r-d(x,y)$. Segue:
\[\forall z \in \ball[r']y: d(x,z) \le d(x,y) + d(y,z) < d(x,y)  + r - d(x,y) < r\]
\item Sia $A$ un qualsiasi aperto per la topologia indotta. $A$ è unione dei suoi punti e per ogni punto $x$ ho una palla aperta centrata in $x$ e contenuta in $A$. Segue che $A$ è unione di palle aperte.
\qedhere
\end{itemize}
\end{proof}

\begin{defn}
$(X,d)$ e $(X,d')$ si dicono \emph{topologicamente equivalenti} se le topologie indotte coincidono.
\end{defn}

\begin{oss}
È chiaro che uno spazio metrico ha una ``struttura'' più rigida di uno spazio topologico e conseguentemente spazi metrici diversi possono indurre la stessa topologia.
\end{oss}

\begin{teo}
Tutte le distanze indotte da un prodotto scalare sono topologicamente equivalenti.
\end{teo}

\begin{proof} \marginpar{DA FARE}
to do
\end{proof}

\begin{es}
to do
\end{es}

\begin{defn}[Intorno]
Dato $x \in (X,\tau)$, si dice \emph{intorno} di $x$ (rispetto a $\tau$) un insieme $U_x$ che contiene un aperto $A$ tale che $x \in A \subseteq U_x$.
\end{defn}

\begin{defn}[Base di aperti]
Dato uno spazio topologico $(X,\tau)$ si dice \emph{base di aperti} una famiglia di aperti $\mathcal B$ tale che ogni elemento di $\tau$ è ottenibile come unione di elementi di $\mathcal B$.
\end{defn}

\begin{defn}[Base di intorni]
Una \emph{base di intorni} è un insieme di intorni di un punto $x$ tale che qualsiasi intorno di $x$ contenga almeno uno degli elementi della base.
\end{defn}

\begin{oss}
Le palle aperte sono una base di intorni per una topologia indotta.
\end{oss}