% Valerio

% che insomma se non si mettono in chiaro le cose me lo dica che me ne vado

\newcommand*\tc{\ \text{t.c.} \ } % tale che
\newcommand*\dual{{^\ast}} % dual
\newcommand*\base[1][B]{\mathcal{#1}} % base
\newcommand*\operp{\overset{\perp}{\oplus}} % orthogonal direct sum
\newcommand*\bigoperp{\overset{\perp}{\bigoplus}} % orthogonal direct sum

\titlet{Forme d'intersezione}

\subtitlet{Una costruzione alternativa}

Sia $X$ superficie, su $\eta_1(X)$ abbiamo la forma d'intersezione $\fundef[\beta]{\eta_1(X) \times \eta_1(X)}{\Z_2}$. Consideriamo le curve su $X$ liscie a meno di un numero finito di incroci normali; ogni tale curva $\gamma$ può essere decomposta in circuiti (ovvero immersioni di $S^1$): $\gamma = \gamma_1 \cup \gamma_2 \cup \gamma_3 \dots$.
Siano $\fundef{S^1 \djcup S^1 \djcup S^1 \djcup \dots \djcup S^1}{X}$ immersioni, allora queste curve sono date da $\set{\gamma} = \set{\Im f}$.

Su queste curve posso definire una relazione di equivalenza specificando operazioni rispetto a cui le classi di equivalenza siano chiuse (ovvero, specificando come posso modificare una di queste curve in modo da ottenere una curva equivalente); sia la relazione $\simeq$ generata dalle operazioni seguenti:
\begin{itemize}
	\item Isotopia ambiente (ovvero, $\gamma \simeq H(1, \gamma)$ con $\fundef[H}{[0,1] \times X}{X}$ isotopia e $H |_{ \set{0} \times X} = \id_X}$)
	\item \input{figura36.pdf_tex}
	\item \input{figura37.pdf_tex}
	\item \input{figura38.pdf_tex}
\end{itemize}
Posso allora pensare $\eta_1(X)$ come $\quoset{ \set{\gamma} }{\simeq} $, e la somma $[\gamma_1] + [\gamma_2]$ è data da $[\gamma_1 \cup \gamma_2]$, con $\gamma_1, \gamma_2$ rappresentanti trasversi tra loro (dunque che s'icrociano solo normalmente e in un numero finito di punti), che certamente esistono (ad esempio per isotopia).

Definisco allora $\beta([\gamma_1], [\gamma_2]} = [\gamma_1] \cdot [\gamma_2]$ come la parità del numero di intersezioni di $\gamma_1 \cup \gamma_2$ (avendo sempre cura di scegliere rappresentanti trasversi tra loro), ovvero come $\mod_2( \card\setdef{x}{x \in \gamma_1 \et x \in \gamma_2})$.

Per mostrare che $\beta$ è non degenere, notiamo che con la nostra relazione $\simeq$ possiamo supporre che per $[\gamma] \neq 0$, $\gamma$ sia liscia e connessa e non può sconnettere $X$, dunque ripetendo la costruzione della lezione precedente posso trovare una curva che intersechi trasversalmente $\gamma$ esattamente una volta.

\subtitlet{Caratterizzazione delle classi d'isometria}

Ricordiamo che la classe d'isometria di $(\eta_1(X), \beta)$ è invariante per diffeomorfismi.

Sia in generale $V$ vettoriale su $\Z_2$ e $\beta$ prodotto scalare non degenere su $V$. Il lettore ricorderà che per spazi vettoriali su $\R$ o $\C$ un prodotto scalare non degenere ammette almeno un vettore non isotropo, poiché se ogni vettore è isotropo si ottiene $0 = \scalprod{v+w}{v+w} = 2\scalprod{w}{v} \ \forall v,w$. È chiaro però che su $\Z_2$ (e in generale per campi con caratteristica 2) quest'identità risulta banale e non implica che $\beta$ sia degenere; poiché la non totale isotropia era necessaria con gli spazi su $\R$ o $\C$ per dimostrare (induttivamente) l'esistenza di una base ortonormale, segue che non potremo concludere altrettanto per $V$.

\begin{lemma}[Lemma 1]
	$V = A \operp B$, con $(A, \beta)$ avente una base ortonormale e $(B, \beta)$ totalmente isotropo.
\end{lemma}
\begin{proof}
	Infatti, essendo $\beta$ non degenere vi sono due possibilità: o $V$ è totalmente isotropo, oppure $\exists v \in V$ non isotropo. Nel primo caso il lemma è banalmente vero, nel secondo ho $V = \Span(v) \operp \set{v}^\perp$ e ripetendo induttivamente il ragionamento su $\set{v}^\perp$ ottengo $V = U \operp U \operp \dots \operp U \operp B$, con $B$ totalmente isotropo e $U$ sottospazio di dimensione 1 tale che $\beta = (1)$ in $U$ (ovvero, $U$ generati da vettori ortonormali).
\end{proof}

Consideriamo ora un sottospazio $H \subseteq V$ totalmente isotropo (con $\beta|_H$ non degenere). Prendiamo una base di $H$ $\base = \set{v_1, v_2, v_3, \dots, v_n}$ e consideriamo la base duale $\base\dual = \set{v_1\dual, v_2\dual, v_3\dual, \dots, v_n\dual}$. Allora dal teorema di rappresentazione sappiamo che 
$v \mapsto (\phi_v : w \mapsto \beta(v,w)$ è isomorfismo, siano dunque $\set{w_1, \dots, w_n}£$ i vettori che rappresentano i covettori di $\base\dual$ (ovvero la controimmagine di $\base\dual$ per l'isomorfismo indicato). Abbiamo quindi $\beta(w_i, v_j) = \delta_{ij}$, e in particolare $\beta(v_1, w_1) = 1$. Pertanto in $\Span(v_1, w_1)$ si ha $\beta = \left(\begin{smallmatrix} 0 & 1 \\ 1 & 0 \end{smallmatrix}\right)$ e dunque $H$ si decompone come $\Span(v_1, w_1) \operp \set{v_1, w_1}^\perp$, e continuando per induzione ottengo una base di $H$ rispetto a cui \[
\beta = \left( \begin{matrix} 0 & 1 \\
							  1 & 0 \\
								&   & 0 & 1 \\
								&   & 1 & 0 \\
								&   &   &   & 0 & 1 \\
								&   &   &   & 1 & 0 \\
								&   &   &   &   &   & \ddots \\
							 \end{matrix} \right)
\]

\begin{defn}[Piano iperbolico]
Chiameremo uno spazio vettoriale di dimensione 2 con prodotto scalare $\beta = \left(\begin{smallmatrix} 0 & 1 \\ 1 & 0 \end{smallmatrix}\right)$ un \emph{piano iperbolico}
\end{defn}
Notiamo che su $\Z_2$ un tale prodotto scalare è totalmente isotropo e non degenere.

Da quanto abbiamo appena visto segue il seguente lemma:
\begin{lemma}[Lemma 2]
	Se $\beta$ su $V$ (vettoriale su $\Z_2$) è non degenere e totalmente isotropo, $\dim V$ è pari e $V$ è somma diretta ortogonale di piani iperbolici.
\end{lemma}

Dunque in generale $V$ vettoriale su $\Z_2$ con $\beta$ non degenere si decompone come $V = \bigoperp U_i \operp \bigoperp H_i$, con $\beta|_{U_i} = (1)$ e $\beta|_{H_i} = \left(\begin{smallmatrix} 0 & 1 \\ 1 & 0 \end{smallmatrix}\right)$.

\begin{lemma}[Lemma 3]
	$U \operp H = U \operp U \operp U$ (con $H$ piano iperbolico e $U$ generato da un vettore non isotropo).
\end{lemma}

La dimostrazione segue nella lezione successiva; i tre lemmi visti ci permettono di dimostrare quanto segue:
\begin{teo}[Teorema di classificazione]
	Dato un prodotto scalare $\beta$ non degenere su $V$ di dimensione $n$, si ha a meno di isometrie:
\begin{itemize}
	\item $V = \bigoperp \limit_{i=1}^n U_i$ \\
	oppure
	\item $ V = \bigoperp \limit_{i=1}^n{ \sfrac{n}{2} } H_i$
\end{itemize}
\end{teo}
