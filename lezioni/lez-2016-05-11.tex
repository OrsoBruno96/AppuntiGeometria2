% questo documento è in ogni sua parte opera originale del sottoscritto
% Alberto Bordin

\titlet{Superfici}

In questa sezione $S$ sarà una superficie, cioè una 2-varietà compatta, chiusa, connessa. L'obiettivo sarà studiare le superfici a meno di diffeomorfismi, come al solito andremo a caccia di invarianti.

\emph{Attenzione:} $S_1, S_2, \dots$ saranno qui superfici, mentre $S^1, S^2, \dots$ saranno circonferenza, sfera, \dots

\begin{es} Seguono alcuni esempi di superfici
\begin{description}
\item [Sfera] $S^2$
\item [Toro] $T := T_1 := S^1 \times S^1$
\item [Tori siamesi] $T_2 :=$ 2 tori incollati con un tubicino
\begin{center}
  \input{fig21.pdf_tex}
\end{center}
\item [Toro-catena] $T_k := k$ tori incollati con $k-1$ tubicini
\begin{center}
  \input{fig22.pdf_tex}
\end{center}
\end{description}
\end{es}

\begin{oss}
Gli esempi precedenti non sono tra loro diffeomorfi, infatti sono distinti dalla $\chi$ di Eulero.

$\chi(S^2) = 2$. Per calcolare la $\chi(T_k)$ usiamo la funzione di Morse \emph{``altezza''} definita come in figura.
\begin{center}
  \input{fig23.pdf_tex}
\end{center}
Notiamo che ci sono $2k + 2$ punti critici di cui
\begin{itemize}
\item $1$ massimo: $\lambda = 1$
\item $2k$ punti di sella (2 per ogni buco): $\lambda = -1$
\item $1$ minimo: $\lambda = 1$
\end{itemize}
Pertanto $\chi(T_k) = 2 - 2k$. Un modo alternativo per calcolarla è mostrare solamente che $\chi(T) = 0$ (notiamo in particolare che è facile trovare per $T$ un campo di vettori mai nullo, ad esempio il campo di vettori tangenti al toro che ``girano'' in senso antiorario) e poi usare la proprietà di additività della $\chi$.
\end{oss}

\begin{ex}
	Sia $\mathbb{P}_2^2$ la varietà ottenuta incollando due $\mathbb{P}^2$ analogamente a quanto fatto per i \emph{tori siamesi}. Dimostrare che $\chi(\mathbb{P}_2^2) = 0$. \emph{Hint:} usare l'additività di $\chi$.
\end{ex}

\begin{oss}
Notiamo, grazie all'esercizio, che la $\chi$ non distingue tra il toro $T$ e $\mathbb{P}_2^2$, tuttavia osserviamo che queste non sono diffeomorfe perché $\mathbb{P}_2^2$ non è orientabile. Urge un invariante più fine: la forma di intersezione.
\end{oss}

\titlet{La forma di intersezione su $\eta_1(S)$}

Ricordiamo che $\eta_1(S) = \quoset{ \{ \fundef{M}{S} \} }{\sim_\text{cob}}$ dove $M = S^1 \sqcup \dots \sqcup S^1$ è una 1-varietà compatta, chiusa, che $\eta_1(S)$ è uno $\quoset\Z{2\Z}$-spazio vettoriale e che dato un diffeomorfismo $\fundef{S_1}{S_2}$ questo induce un isomorfismo $\fundef[f_*]{\eta_1(S_1)}{\eta_1(S_2)}$ (vedi lezione dell'11 aprile).

\begin{teo}
\marginpar{io farei una macro per i "fatti" alias "atti di fede" alias "cannoni" e gli darei una numerazione propria \\
\emph{Risposta: ci sono già i teoremi senza dimostrazione}}
$\dim \eta_1(S)$ è finita (per dimostrarlo si userebbe la compattezza di~$S$).
\end{teo}

\begin{cor}
	$\dim \eta_1(S)$ è un invariante per diffeomorfismi.
\end{cor}

\begin{defn}[forma di intersezione $\beta$] $\beta$ sarà una \[ \fundef[\beta]{\eta_1(S) \times \eta_1(S)}{\quoset\Z{2\Z}} \] Cerchiamo un modo di definire $\beta([M_1,f_1],[M_2,f_2])$.
\[ f_1 \times f_2: M_1 \times M_2 \funarrow S \times S \hookleftarrow \Delta \text{ diagonale} \]
Dai teoremi di trasversalità sappiamo trovare una $\tilde{f_1}$ e una $\tilde{f_2}$ omotope rispettivamente a $f_1$ e $f_2$ e tali che $\tilde{f_1} \pitchfork \tilde{f_2}$ e, sempre per i teoremi di trasversalità, abbiamo che $(\tilde{f_1} \times \tilde{f_2})^{-1}(\Delta)$ è una sottovarietà di $M_1 \times M_2$, che ha dimensione 2; inoltre $\codim{(\tilde{f_1} \times \tilde{f_2})^{-1}(\Delta)} = 2$ pertanto $(\tilde{f_1} \times \tilde{f_2})^{-1}(\Delta)$ ha dimensione 0, ovvero è un insieme finito di $k$ punti $\{x_1, \dots , x_k \}$.

Ha dunque senso definire \[ \beta([M_1,f_1],[M_2,f_2]) := k \mod 2 \]
\end{defn}

\begin{ex}
	$\beta$ è ben definita, bilineare e simmetrica.
\end{ex}

La forma d'intersezione $\beta$ misura la parità del numero di intersezioni tra le immagini di due curve su una superficie. Grazie ai teoremi di trasversalità (grazie anche alla dimensione finita e al teorema di immersione di Whitney) possiamo supporre che le $f_1$ e $f_2$ siano fin da subito immersioni e che i punti in comune alle due immagini su $S$ non siano dei punti di tangenza, ma siano degli incroci ``normali'', infatti se le $f_1$ e $f_2$ non avessero tali requisiti basterebbe perturbarle un poco per trovare delle $\tilde{f_1}$ e $\tilde{f_2}$, omotope alle precedenti, che abbiano i suddetti requisiti.

\paragraph{Caratterizzazione di $\eta_1(S)$}
Cercheremo ora di capire come sono fatte le classi di equivalenza $[M,f]$. Con lo stesso spirito di quando caliamo i cannoni per ``perturbare un poco'' le funzioni e ricondurci a studiare casi semplici, il nostro intento sarà introdurre alcune ``mosse'' che pur perturbando una curva non ne cambiano la classe. In questo modo potremo scegliere, senza perdita di generalità, un rappresentante particolarmente semplice: scopriremo che possiamo prendere un rappresentante connesso.

Per quanto già discusso prendiamo una $\fundef{M}{S}$ che sia fin da subito un'immersione
\begin{center}
  \input{fig24.pdf_tex}
\end{center}
faremo ora due semplificazioni:

\begin{enumerate}
\item Eliminare i punti doppi
\begin{center}
  \input{fig25.pdf_tex}
\end{center}
Notiamo che, non avendo orientazione, la scelta è indifferente.

Quindi da una $\fundef{M}{S}$ abbiamo ottenuto un embedding $\fundef[i]{M'}{S}$ senza punti doppi.

Esercizio: mostrare che $[M,f] = [M',i]$ in $\eta_1(S)$.

\item Connettere l'immagine
\begin{center}
	\def\svgwidth{11cm}
  \input{fig26.pdf_tex}
\end{center}
Dove il cobordismo è ``esplicitato'' dalla figura seguente: i cerchi interni vanno immaginati come sollevati di qualche centimetro rispetto al piano del foglio e la parte colorata vista come una bolla di sapone.
\begin{center}
	\def\svgwidth{11cm}
  \input{fig27.pdf_tex}
\end{center}
\end{enumerate}

Pertanto, grazie alla semplificazione (1), ogni classe di equivalenza  $\alpha \in \eta_1(S)$ può essere rappresentata per mezzo di una 1-sottovarietà di $S$ e, grazie alla (2), $\alpha$ può essere rappresentata da una curva connessa. D'ora in poi le curve che considereremo saranno belle lisce, iniettive e connesse.

\begin{teo}
 La forma di intersezione è non degenere.
\end{teo}

\begin{proof}
Utilizzeremo la notazione $\alpha \cdot \gamma := \beta(\alpha,\gamma)$.

Sia $\alpha \in \eta_1(S)$ tale che $\alpha \cdot \gamma = 0$  $\forall \gamma$, vogliamo mostrare che  $\alpha = 0$.

Supponiamo per assurdo che $\alpha \neq 0$, sia dunque $\alpha = [c]$ con $c$ connessa su $S$.

Mostriamo che $c$ non sconnette $S$. Infatti se $S \setminus c$ fosse sconnesso allora $c$ sarebbe cobordante al vuoto, come si vede dalla figura, e sarebbe dunque $\alpha = 0$.
\begin{center}
  \input{fig28.pdf_tex}
\end{center}
Dunque $c$ non sconnette $S$, prendiamo quindi un piccolo arco $\gamma_1$ trasverso a $c$ di estremi $x_1$ e $x_2$. Poiché $c$ non sconnette $S$ allora esiste un arco $\gamma_2$ che collega $x_1$ e $x_2$. Sia $\gamma = \gamma_1 \cup \gamma_2$, abbiamo dunque costruito un $\gamma$ tale che $\alpha \cdot \gamma = 1 \neq 0$ \absurd.
\end{proof}
\begin{center}
  \input{fig29.pdf_tex}
\end{center}

\begin{oss}
Abbiamo visto che $\beta$ è bilineare, simmetrica e non degenere, dunque $\beta$ è un \emph{prodotto scalare}.

Inoltre $\dim \eta_1(S)$ è finita, allora dato un diffeomorfismo $\fundef{S_1}{S_2}$ questo induce un isomorfismo $\fundef[f_*]{\eta_1(S_1)}{\eta_1(S_2)}$ tale che $\beta_1(\alpha,\gamma) = \beta_2(f_*(\alpha),f_*(\gamma))$ dunque $f_*$ è un'\emph{isometria} tra $(\eta_1(S_1), \beta_1)$ e $(\eta_1(S_2), \beta_2)$.

Abbiamo costruito un nuovo invariante per diffeomorfismi: la classe di isometria di $(\eta_1(S), \beta)$.
\end{oss}