%autore: Alessandro Candido

% non ho inserito una piccola parte sulla caratteristica delle sfere perché di riepilogo, anche se non ancora presente nella lezione precedente
% in alcuni ambienti (teo,prop,cor,oss,defn) mi piacerebbe che il testo venga scritto sotto il titolo e non a fianco, perciò trovate alcune ~ qua e là che realizzano quanto detto; se non siete d'accordo le togliamo, ma proporrei di inserirle nella definizione dell'ambiente (il che significa toglierle comunque)

\begin{teo}
$X$ ammette un campo tangente mai nullo $\iff$ $\chi(X) = 0$
\end{teo}

\begin{proof}
~
%questa ~ non rientra in quanto scritto sopra, l'ho inserita solo al fine di evitare in questo caso un mancato allineamento orizzontale di description
	\begin{description}
		\item[\proofrightarrow]
		già visto		
		\item[\proofleftarrow]
		ci si appoggia al:
		\begin{teo}[Teorema di Hopf]
		$f_0~\text{omotopa}~f_1$ $\iff$ $\grad{f_0} = \grad{f_1}$

		\emph{Caso particolare:} $\fundef[f]{S^n}{S^n}$ è omotopa ad una costante $\iff$ $\grad{f} = 0$
		\end{teo}
%devo trovare il modo di allineare il testo seguente all'item e non al teorema
% risposta: è allineato all'item! È la prima riga dell'item che è allineata a sinistra. Il teorema non ha allineamenti particolari, si becca quello dell'item
		Si ha dunque che $\chi(X) = 0$. Sia $s \pitchfork s_0$, con $x_j$ zero isolato di $s$.
		Applicando il lemma di omogeneità non è restrittivo supporre che $x_1, \dots, x_k \in B$, palla dentro una carta di $X$.
		Inoltre considero per ogni $x_j$ una palletta di centro $x_j$ e raggio abbastanza piccolo da non contenere altri $x_i$.
		Si ha che la figura realizza un cobordismo esplicito.

		\begin{figure}[h]
		\centering
		\input{figura34.pdf_tex}
		\caption{Punti di Morse in carta locale}
		\end{figure}

		Perciò la somma dei gradi interni è pari al grado esterno
		\begin{equation}
		\sum_{j=1}^k i_s(x_j) = 0
		\end{equation}
		Perciò la sezione sulla sfera esterna è omotopa ad una costante, dunque taglio all'interno della palla e sostituisco con il 			vettore costante e in questo modo non  ho più zeri $\so$ $\hat{s}$ si estende ad un'applicazione $\fundef[F]{D^{n+1}}{S^n}$, 			sostituisco $\hat{s}$ definito su $D^{n+1}$, con il campo $F$ che estende $\hat{s}|_{S^n}$.
	\end{description}
\end{proof}

\begin{oss}
~
\begin{itemize}
\item Per semplicità si consideri $X$ connesso, quindi ha 2 orientazioni possibili. $\chi(X) = X \cdot X$ in $TX$, il fibrato tangente, e si ha che non dipende dall'orientazione di $X$ fissata, infatti passando da un orientazione all'altra si ha che cambia anche quella delle fibre, perciò in $TX$ quando si cambia base si ha che nel Jacobiano cambia di segno un blocco di dimensione pari, e perciò il determinante rimane 1;
\item Lo stesso ragionamento vale anche se $X$ non è orientabile, ma solo a livello locale, questo però è sufficiente per far sì che $\chi(X)$ è ben definita anche in questo caso;
\item Sia $\fundef[f]{X}{Y}$ con entrambe le varietà compatte e connesse, $f$ diffeomorfismo locale, e sia $p=\card f^{-1}(y)$. Si ha che $p$ è costante come funzione di $y$.
\begin{ex}
allora $\chi(X) = p~\chi(Y)$.
\end{ex}
\end{itemize}
\end{oss}

\begin{es}
L'ultima osservazione ci permette ad esempio di calcolare la caratteristica di Eulero,
\begin{align*}
	\chi(\mathbb{P}^n)=\begin{cases}
		0&\text{$n$ dispari} \\
		1&\text{$n$ pari}
	\end{cases}
\end{align*}
\end{es}

\begin{ex}
La caratteristica di Eulero è ``moltiplicativa'' $\chi(X_1\varprod X_2) = \chi(X_1)\chi(X_2)$
\end{ex}

\titlet{Funzioni di Morse}

Si consideri la triade $(W,V_0,V_1)$

\begin{center}
\begin{tikzpicture}[
	  tqft/.cd,
	  cobordism/.style={draw},
	  every boundary component/.style={draw,rotate=90}
	]
  \pic [tqft/pair of pants, rotate=90];
\end{tikzpicture}\end{center}

\begin{defn}[Funzioni di Morse]
~
Una funzione sulla triade, $\fundef[f]{W}{[0,1]}$ con $f^{-1}(0)=V_0$ e $f^{-1}(1)=V_1$, è detta di Morse se:
\begin{itemize}
\item non ci sono punti critici di $f$ in un intorno di $\boundary W = V_0\djcup V_1$
\item tutti i punti critici sono non degeneri
\end{itemize}
\end{defn}

Perciò per una funzione di Morse si ha che $\forall x$ critico per $f~\exists$ una carta locale, in cui (ponendo \wlg $x=0$) la funzione coincide con il modello locale di Morse $f(x) = -(x_1^2 + \dots + x_{\lambda}^2) + (x_{\lambda+1}^2 + \dots + x_n^2)$, dove $n$ è la dimensione di $W$ e $\lambda$ l'indice del punto critico.

%Piccolo intermezzo di algebra lineare: se ho un prodotto scalare identifico canonicamente vettori e funzionali

Si ha che l'applicazione tangente di una funzione $\fundef[f]{W}{\R}$ è dunque un campo di funzionali su $TW$, cioè un elemento di $T*W$, il fibrato cotangente.
Fissiamo su $W$ una metrica Riemanniana $g$, cosicché $\nabla _g f$ è il campo di vettori che rappresenta $Df$ tramite $g$.

Si applica ora quest'ultimo risultato alle funzioni di Morse. Fissiamo dunque una metrica che nelle carte di Morse della funzione in oggetto appaia come la metrica standard di $\R^n$.

\begin{figure}[h]
\centering
\input{figura35.pdf_tex}
\caption{Transizione da vuoto a vuoto}
\end{figure}

%figura con W chiusa e punti critici
Sia ora $W$ chiusa, quindi una transizione dal vuoto al vuoto. $\fundef[f]{W}{\R}$, e sia $g$ la metrica scelta. Allora si ha che:
\begin{equation*}
\{\text{gli zeri di}~\nabla_g f\} = \{\text{punti critici di}~f\}
\end{equation*}

In quanto punto critico $x$ ha associato un indice di Morse $\lambda$, mentre come campo di vettori ha associato il segno del Jacobiano, che si legge esplicitamente dal gradiente.

\begin{teo}
$W$ compatta chiusa, $\fundef[f]{W}{\R}$ di Morse allora $\chi(W)=\sum_{\text{$x$ p.to critico}} (-1)^{\lambda}$ con $\lambda$ indice di Morse del punto.
\end{teo}

\begin{teo}[Altro teorema di trasversalità]
$\mathcal{M}(W) = \set{\fundef[f]{W}{\R}~\text{di Morse}}$ è aperto in $\Sge(W,\R)$.
\end{teo}

\begin{oss}
Se $\fundef[f]{W^n}{\R}$ è di Morse anche $-f$ lo è, e calcolando la caratteristica di Eulero mediante queste due funzioni dobbiamo perciò ottenere lo stesso risultato. Da cui si ottiene che:
\begin{equation*}
\chi(W) = \sum_\text{$x_j$~p.to~critico} (-1)^{\lambda _j} = \sum_\text{$x_j$~p.to~critico} (-1)^{n - \lambda _j} 
\end{equation*}
con $\lambda _j$ indice di Morse di $x_j$.
\end{oss}

Dall'ultima osservazione si ottiene:
\begin{prop}
Se $\dim W$ è dispari $\implies$ $\chi(W) = 0$
\end{prop}

\begin{cor}
Ogni varietà di dimensione dispari ammette un campo di vettori mai nullo.
\end{cor}
