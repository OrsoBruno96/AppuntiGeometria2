% trascrizione: Petrillo

\titlet{Sottospazi}

Si può definire una topologia sui sottoinsiemi di uno spazio topologico:

\begin{prop}[Topologia dei sottospazi]
	Sia $(X,\tau)$ spazio topologico e $\chi\subseteq X$, l'insieme
	$\tau_\chi\is\setdef[A\cap\chi]{A\in\tau}$
	è una topologia su $\chi$.
\end{prop}

\begin{proof}
	Verifichiamo le tre proprietà della topologia:
	\begin{itemize}
		\item Il vuoto c'è perché $\nullset\cap\chi=\nullset$;
		$\chi$ c'è perché $X\cap\chi=\chi$.
		\item Siano $a_1,\ldots,a_n\in\tau_\chi$.
		Allora $\forall k\,\exists A_k\in\tau:a_k=A_k\cap\chi$.
		Quindi:
		\[\inters a_k=\inters(A_k\cap\chi)=\left(\inters A_k\right)\cap\chi\]
		Ma $\inters A_k$ è aperto in $(X,\tau)$.
		\item Sia $\set{a_k}_{k\in I}\subseteq\tau_\chi$.
		Definiti come sopra gli $A_k$, abbiamo:
		\[\union a_k=\union(A_k\cap\chi)=\left(\union A_k\right)\cap\chi \qedhere\]
	\end{itemize}
\end{proof}

\begin{defn}[Sottospazio]
	Chiamiamo $(\chi,\tau_\chi)$ \emph{sottospazio} di $(X,\tau)$.
\end{defn}

È interessante che la topologia dello spazio di partenza si colleghi direttamente alla topologia di sottospazio di ogni aperto:

\begin{prop}
	$A$ aperto $\iff$ tutti gli aperti in $A$ come sottospazio sono aperti
\end{prop}

\begin{proof}
	Mostriamo le due implicazioni:
	\begin{description}
		\item[\proofrightarrow]
		Infatti ogni aperto in $A$ è intersezione di due aperti.
		\item[\proofleftarrow]
		Infatti $A$ è aperto in $A$. \qedhere
	\end{description}
\end{proof}

\titlet{Chiusura, parte interna, frontiera}

In generale i sottoinsiemi di uno spazio topologico non sono né aperti né chiusi. Tuttavia possiamo associare a ognuno in modo naturale un aperto e un chiuso.

\begin{defn}[Chiusura]
	La \emph{chiusura} di un insieme $Y$ è l'intersezione di tutti i chiusi che lo contengono:
	\[\clos Y\is\inters_{\substack{C\text{ chiuso}\\C\supseteq Y}}C\]
\end{defn}

\begin{prop}
	La chiusura di $Y$ è il più piccolo chiuso che contiene $Y$, cioè:
	\[\begin{rcases}
		Y\subseteq C\subseteq\clos Y\\
		C\text{ chiuso}
	\end{rcases}\implies C=\clos Y\]
\end{prop}

\begin{proof}
	Infatti dalla definizione di $\clos Y$ segue $\clos Y=\clos Y\cap C$.
\end{proof}

\begin{defn}[Parte interna]
	La \emph{parte interna} di un insieme $Y$ è l'unione di tutti gli aperti contenuti in $Y$:
	\[\inter Y\is\union_{\substack{A\text{ aperto}\\A\subseteq Y}}A\]
\end{defn}

\begin{prop}
	La parte interna di $Y$ è il più grande aperto contenuto in $Y$, cioè:
	\[\begin{rcases}
		\inter Y\subseteq A\subseteq Y\\
		A\text{ aperto}
	\end{rcases}\implies A=\inter Y\]
\end{prop}

\begin{proof}
	Infatti dalla definizione di $\inter Y$ segue $\inter Y=\inter Y\cup A$.
\end{proof}

\begin{oss}
	La parte interna e la chiusura sono rispettivamente aperta e chiusa.
\end{oss}

Definiamo ora alcune proprietà che collegano singoli punti di un insieme alle nozioni di chiusura e parte interna.

\begin{defn}[Punto interno]
	Gli elementi della parte interna di $Y$ si chiamano \emph{punti interni di $Y$}:
	\[x\text{ interno a }Y\means x\in\inter Y\]
\end{defn}

\begin{defn}[Punto di accumulazione]
	Un elemento $x$ (non necessariamente in $Y$) è \emph{punto di accumulazione per $Y$} se ogni intorno di $x$ contiene punti di $Y$ diversi da $x$:
	\[x\text{ accumulazione per }Y\means\forall\,U_x:(U\setminus\set{x})\cap Y\neq\nullset\]
\end{defn}

\begin{defn}[Punto isolato]
	I punti di $Y$ non di accumulazione sono \emph{isolati}:
	\[x\text{ isolato in }Y\means
	\begin{cases}
		x\in Y\\
		\neg(x\text{ accumulazione per }Y)
	\end{cases}\]
\end{defn}

\begin{lemma}
	La chiusura di un insieme è l'insieme stesso unito ai suoi punti di accumulazione:
	\[\clos Y=Y\cup\setdef{x\text{ accumulazione per }Y}\]
\end{lemma}

\begin{proof}
	Mostriamo le due inclusioni:
	\begin{description}
		\item[\proofsubseteq]
		Ci basta verificare che $y\in\clos Y\setminus Y\implies y\text{ accumulazione per }Y$.
		Per assurdo:
		\[y\text{ non di accumulazione}\so
		\exists A\text{ aperto}:
		\begin{cases}
			y\in A\\
			A\cap Y=\nullset
		\end{cases}\text{cioè }
		\begin{cases}
			y\not\in\comp A\\
			Y\subseteq\comp A
		\end{cases}\]
		Ma $\comp A$ è chiuso, quindi avremmo $y\not\in\clos Y\absurd$.
		\item[\proofsupseteq]
		Ci basta verificare che $y\not\in Y\et\, y\text{ accumulazione per }Y\implies y\in\clos Y$.
		Per assurdo, $y\not\in\clos Y\so y\in\comp{\clos Y}$. Ma $\comp{\clos Y}$ è aperto, quindi $\comp{\clos Y}$ sarebbe un intorno di $y$ disgiunto da $Y\absurd$. \qedhere
	\end{description}
\end{proof}

Vogliamo ora formalizzare il concetto di ``bordo'' di un insieme, cioè la famiglia dei punti che lo ``separa'' dal suo complementare. Diamo questa definizione:

\begin{defn}[Frontiera]
	La famiglia dei punti della chiusura di $Y$ esterni a $Y$ è la \emph{frontiera} di $Y$:
	\[\bound Y\is\clos Y\setminus\inter Y\]
\end{defn}

\begin{prop}
	La frontiera consta dei punti isolati e dei punti esterni di accumulazione:
	\[\bound Y=\setdef{x\text{ isolato in }Y}\cup\setdef{x\text{ accumulazione per }Y}\setminus\inter Y\]
\end{prop}

% questa cosa l'aveva detta ma poi ha detto che è sbagliata il 24 febbraio
%\begin{oss}
%$x\text{ interno a }Y\implies x\text{ accumulazione per }Y$
%\end{oss}

\begin{prop}
$\bound Y=\bound\comp Y$
\end{prop}

\titlet{Morfismi degli spazi topologici}

$X$, $Y$ spazi topologici; $\fundef XY$

\begin{defn}[Continuità]
$f\text{ continua}\means\forall A\subseteq Y,\,A\text{ aperto}:f^{-1}(A)\text{ aperto}$
\end{defn}

\begin{defn}[Continuità locale]
$f\text{ continua in }x\means\forall U_Y(f(x))\,\exists U_X(x):f(U_X)\subseteq U_Y$
\end{defn}

\begin{prop}
$f\text{ continua}\iff\forall x\in X:f\text{ continua in }x$
\end{prop}

\begin{defn}
$f\text{ omeomorfismo}\means f\text{ isomorfismo topologico}\means f\text{ bigettiva }\et f, f^{-1}\text{ continue}$
\end{defn}

\begin{es}
Consideriamo due spazi topologici sullo stesso insieme $(X,\tau_1)$, $(X,\tau_2)$. Sia $\tau_1$ più fine di $\tau_2$, cioè $\tau_1\supset\tau_2$. Sia $\fundef[\id]XX$ l'identità. $\id$ è invertibile e continua rispetto a $\tau_1\funarrow\tau_2$, ma non continua rispetto a $\tau_2\funarrow\tau_1$.
\end{es}

\titlet{Separazione e numerabilità}

\begin{defn}[Proprietà di separazione]
$X\text{ di Hausdorff}\means X\text{ è }T_2\means\forall x\neq y\,\exists U(x),U(y):U(x)\cap U(y)=\nullset$
\end{defn}

\begin{prop}
$X\text{ metrizzabile}\implies X\text{ è }T_2$
\end{prop}
\begin{proof}
$x\neq y\so r\is\dis(x,y)\neq 0\so\ball[r/3]x\cap\ball[r/3]y=\nullset$
\end{proof}

\begin{prop}
$X\text{ è }T_2\implies\forall x\in X:\set{x}\text{ chiuso}$
\end{prop}
\begin{proof}
$(\forall y\in\comp\set{x}\,\exists U(y),U(x):U(y)\cap U(x)=\nullset\so U(y)\subseteq\comp\set{x})\so\comp\set{x}\text{ aperto}$
\end{proof}

\begin{es}[Topologia Zariski]
Sia $\tau_z$ una topologia su $\R$:
\[\tau_z\is\setdef[A\subset\R]{A=\nullset\vel(A=\comp V\et\card V\in\N)}\]
Allora $\forall A_1,A_2\in\tau_z:A_1\cap A_2\neq\nullset\so(\R,\tau_z)\text{ non è }T_2$. Però i singoletti sono chiusi.
\end{es}

\begin{oss}
La topologia euclidea è più fine di $\tau_z$.
\end{oss}

\begin{prop}
La proprietà $T_2$ passa ai sottospazi.
\end{prop}

\begin{prop}
La proprietà $T_2$ è invariante per omeomorfismo.
\end{prop}
\begin{proof}
Sia $\fundef XY$ un omeomorfismo. Mostriamo che $Y\text{ è }T_2\implies X\text{ è }T_2$:
\begin{align*}
&\forall x_1\neq x_2\in X:\\
&y_i\is f(x_i),\,i=1,2\\
f\text{ iniettiva}\so&y_1\neq y_2\\
Y\text{ è }T_2\so&\exists U_Y^i(y_i)\text{ aperti}:U_Y^1\cap U_Y^2=\nullset\\
f\text{ continua}\so&U_X^i\is f^{-1}(U_Y^i)\text{ sono aperti}\et U_X^1\cap U_X^2=\nullset
\end{align*}
Applicando lo stesso ragionamento a $f^{-1}$ si mostra che $X\text{ è }T_2\implies Y\text{ è }T_2$.
\end{proof}

\begin{es}
Sia $\fundef[\id]\R\R$ e $\tau_E$ la topologia euclidea. $\id$ è continua rispetto a $\tau_E\funarrow\tau_z$ perché $\tau_z\subseteq\tau_E$, però non nell'altro verso. Infatti $(\R,\tau_E)$ è $T_2$ ma $(\R,\tau_z)$ no, quindi non sono omeomorfi.
\end{es}

\begin{defn}[Numerabilità]
$X\text{ numerabile}\means\card X\leq\card\N$
\end{defn}

\begin{defn}[Proprietà di numerabilità]
$X$ 1-numerabile $\means\forall x\in X\,\exists$ base di intorni di $x$ numerabile
\end{defn}

\begin{prop}
$X$ metrizzabile $\implies X$ 1-numerabile
\end{prop}

\begin{defn}[Proprietà di numerabilità]
$X$ 2-numerabile $\means\exists$ base di aperti di $X$ numerabile
\end{defn}
