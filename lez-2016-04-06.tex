% Federico
%cusu

\newcommand*\tc{\ \text{t.c.} \ } % tale che
\newcommand*\dual{{^\ast}} % dual
\newcommand*\base[1][B]{\mathcal{#1}} % base

\titlet{Bho}
\begin{defn}[Equivalenza tra fibrati in termini di cocicli]
Sia $M$ una varietà differenziabile con atlante massimale $\set{(U_i, \phi_i)}$ e sia $\set{ \fundef[\mu_{ij}]{U_i \cup U_j}{G} }$ un cociclo a valori in un gruppo $G \subseteq \Aut(F)$ (con $F$ varietà liscia). Ripetendo la costruzione usata per realizzare il fibrato tangente $T(M)$ (per cui s'era usato $G = \GL(n, \R)$ e $F=\R^n$, $n = \dim M$), ottengo un fibrato $\Fundef[\pi]{E}{M}$ di fibra $F$ e gruppo strutturale $G$.
Siano dati due fibrati su $M$ con lo stesso gruppo di struttura e la stesa fibra: $\fundef[\pi_1]{E_1}{M}$ e  $\fundef[\pi_2]{E_2}{M}$.
Siano $\set{ \fundef[\mu_{ij}]{U_i \cup U_j}{G} }$ e $\set{ \fundef[\lambda_{ij}]{U_i \cup U_j}{G} }$ rispettivamente coclici di $E_1$ e $E_2$.
$E_1$ ed $E_2$ sono fibrati equivalenti (nel senso dei coclici) se $\exists \fundef[\gamma_{i}]{U_i}{G}$ tale che $\forall x U_i \cup U_j$ dati $\lambda_{ij}$ e $\mu_{ij}$
sia $\lambda_{ji} = {\gamma_j}^{-1} \times \mu_{ji} \times  {\gamma_i}^{-1}$.  (Quest'ultima è la moltiplicazione tra elementi del gruppo $G$)
\end{defn}
\marginpar{
Mi sa proprio che il cociclo $\mu_{ij}$ et similia si definiscono nell'intersezione, non nell'unione!!
}
\begin{defn}[Immersione liscia]
Sia $\fundef[f]{X}{Y}$, $f$ è una immersione se $\forall x \in X \fundef[Df_{x}]{T_{x}X}{T_{f(x)} X}$ (restrizione dell'applicazione tangente alla fibra) è iniettiva.
\end{defn}

\begin{defn}[Embedding] 
 Una immersione $f$ è un embedding se $\fundef[f]{X}{f(X)}$  è un omeomorfismo (funzione continua fra spazi topologici tale che ha inversa continua).
\end{defn}

%Nn mi è chiaro cosa volesse dire con questa frase...
%Se $X$ è compatta e $f$ è iniettiva e suriettiva allora $f$ è un embedding.

\begin{defn}[Metrica Riemanniana] %a caso
 Una metrica riemanniana su una varietà $X$ è un campo di tensori di tipo (0,2) su $X$ simmetrici e definiti positivi. Cioè è una sezione tale che ogni tensore è simmetrico 
 e definito positivo: $\fundef[R]{X}{{T_{2}}^{0}}$ 
\end{defn}

\marginpar{Non è a caso come sembra, ma bisognerebbe mettere prima il teorema, poi il suo corollario, se no, il lettore (ma forse anche lo scrittore) non ne capisce più nulla! In particolare quà voleva dire che, esistendo un ebbending, posso vedere la varietà come immersa in un $R^n$ dotato del suo prodotto scalare canonico. Ad ogni punto $x$ prendo dunque come metrica il tensore associato al prodotto canonico ristretto all'immagine del $T_x M$ attraverso l'ebbending}


\begin{cor}
 Ogni $X$ compatta ha una metrica Riemanniana.
\end{cor}

\begin{teo}[Di embedding elementare di Whitney]
 Se $X$ è una varietà compatta allora esiste un $N$ abbastanza grande tale che $\exists$ un embedding $\fundef[f]{X}{\mathbb{R}^{N}}$
\end{teo}

\marginpar{Penso che non sia ${\phi_{i}(\ball[1]0)}$ ma ${\phi_{i}^{-1}(\ball[1]0)}$. Inoltre dovrebbe essere $\fundef[\lambda_i]{U_i\subseteq X}{[0,1]}$.}

\begin{proof}
 Poiché $X$ è compatta esiste un atlante finito $\set{(U_i, \phi_i)}$ tale che 
 \begin{itemize}
  \item $\ball[2]0 \subseteq {\phi}_{i}(U_{i})$
  \item ${\phi_{i}(\ball[1]0)}$ ricopre tutto $X$
  \end{itemize}
 Sia $\fundef[\lambda]{\R^N}{[0, 1]}$ la funzione a foruncolo relativa alle palle $B_{1}$ e $B_{2}$. Definisco la funzione $\fundef[\lambda_{i}]{\R^N}{[0, 1]}$
 come $\lambda \circ \phi_{i}$ in $U_{i}$ e $0$ in $X \setminus U_{i}$.
 
 ${B_{i}}$ ricopre $X$ e $B_{i} = {\lambda}^{-1} \subseteq U_{i}$. Poiché $\lambda_i$ vale 1 quando sono su ${\phi_{i}}^{-1} (\ball[1]0)$ ho un ricoprimento.
 
 
\marginpar{definisco ora $f_i$ come prodotto? la composizione non ha più senso con queste definizioni, in particolare dovrebbe essere $f_i:=\lambda \circ \phi_i\circ\phi_i $!}
 
 Definisco $\fundef[f]{X}{\R^N}$ come $\lambda_{i} \circ \phi_{i}$ in $U_{i}$ e $0$ in $X \setminus U_{i}$. Considero le funzioni $\fundef[g_{i}]{X}{\mathbb{R}^{N}}$
 definita come $x \mapsto (f_{i}(x), \lambda_{i}(x))$. Definisco ancora la funzione $g$ che: $x \mapsto (g_1, g_2, \dots, g_m)$
 
 Dico che $g$ così definita è un embedding, devo dunque dimostrare che è una immersione (cioè che il suo differenziale è iniettivo) e che la funzione stessa $f$ è omomorfismo cioè
 ha funzione inversa continua.
 
 \marginpar{Che le $g_i$ siano immersioni è semplicemente falso, ogni tanto sono la funzione costante nulla!!! Sono immersioni ogniuna nel suo $\psi^{-1}(B_1(0))$}
 
 \begin {itemize}
  \item è diffeomorfismo per costruzione
  \item tutte le $g_{i}$ per come son state costruite sono immersioni
  \item g è iniettiva: se ho $x \neq y$  ho due possibilità: $x, y \in B_{i}$ allora la $f_i$ coincide con la $\phi_i$ sulla palla $B_{i}$, che è iniettiva sulla palla. 
  Oppure $x, y$ stanno in due palle diverse in particolare $y \in B_{i}$ allora $\lambda_{i}(y) = 1$ e $\lambda_{i}(x)$ 
  \qedhere
 \end {itemize}
\end{proof}

Se ho una varietà $X$ con bordo $\boundary X \neq 0$ si può rafforzare la costruzione in modo che $(X, \boundary X)$ sia una sottovarietà del semipiano 
$(\mathbb{H}^{N}, \boundary \mathbb{H}^{N})$ con $\boundary X = X \cup \boundary \mathbb{H}^{N}$ e su $\boundary X$ ho $X \perp \boundary \mathbb{H}^{N}$.
Inoltre se $x_{1}, x_{2}, x_{3}, \dots, x_{k}$ sono punti della varietà con bordo non nullo si può rafforzare la costruzione in modo che un intorno di $x_{j}$ va in un piano di $\mathbb{R}^{N}$.


\marginpar{No, voglio munire $\epsilon$ di una topologia}

\titlet{Spazi di applicazioni lisce}
\begin{defn}[Topologia sulle applicazioni lisce tra varietà in $\mathbb{R}^{N}$]
Definisco $\mathcal{E} \is \setdef[\fundef{X}{Y}]{f\text{ liscia}}$, con $X$ e $Y$ varietà lisce. Voglio munire $X$ e $Y$ di una topologia. Se $X \subseteq \mathbb{R}^{N}$ e $Y = \mathbb{R}^{N}$ considero
la base di intorni data $U_{r, k, \epsilon}$ e $r \in \mathbb{N}$ e $K \subseteq X$ e $\epsilon > 0$ è un compatto. Questa base di intorni è definita da 
\[U_{r, K, \epsilon} \is \Setdef[{\fundef[g]{X}{\mathbb{R}^{N}}}]{\left\|\frac{\partial f}{\partial x_{i_{1}} \dots \partial x_{i_{k}}} - \frac{\partial g}{\partial x_{i_{1}} \dots \partial x_{i_{k}}}\right\| \eqslantless \epsilon}\]
\end{defn}

\marginpar{perchè ti restringi (solo in arrivo mi sembra di capire) a varietà reali?}
\marginpar{va bene che è banale, ma per avere una topologia sulle funzioni l'idea è di liberarsi dalla dipendenza dal compatto K preso in considerazione tappezzando la varietà di partenza con un ricoprimento compatto finito ${K_i}$ (che trovo se la varietà X è compatta) e prendere l'intersezione dei $U_{r,K_i, (U_i, \psi_i)(U'_i, \psi'_j)}$}


\begin{defn}[Topologia sulle applicazioni lisce tra varietà generiche]
Per definire una topologia tra varietà generiche mi riduco al caso di varietà reali.
Definisco una base di intorni $U_{r, K, \epsilon, (U, \phi), (U', \phi')} = \set{\fundef[g]{X}{\mathbb{R}^{N}}}$ tale che $g(U) \subseteq U'$ e valga 
$\phi \circ g \circ \phi^{-1} \in U_{r, K, \epsilon}(\phi \circ g \circ \phi^{-1})$ cioè che $\phi \circ g \circ \phi^{-1}$ sia intorno di una funzione da $\mathbb{R}^{N}$ a $\mathbb{R}^{N}$.
\end{defn}

\begin{oss}
 La topologia che ottengo è metrizzabile.
\end{oss}







