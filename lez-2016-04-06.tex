% Federico
%cusu

\newcommand*\tc{\ \text{t.c.} \ } % tale che
\newcommand*\dual{{^\ast}} % dual
\newcommand*\base[1][B]{\mathcal{#1}} % base

\titlet{Bho}
\begin{defn}[Equivalenza tra fibrati in termini di coclicli]
Sia $M$ una varietà differenziabile con atlante massimale $\set{(U_i, \phi_i)}$ e sia $\set{ \fundef[\mu_{ij}]{U_i \cup U_j}{G} }$ un cociclo a valori in un gruppo $G \subseteq \Aut(F)$ (con $F$ varietà liscia). Ripetendo la costruzione usata per realizzare il fibrato tangente $T(M)$ (per cui s'era usato $G = \GL(n, \R)$ e $F=\R^n$, $n = \dim M$), ottengo un fibrato $\Fundef[\pi]{E}{M}$ di fibra $F$ e gruppo strutturale $G$.
Siano dati due fibrati su $M$ con lo stesso gruppo di struttura e la stesa fibra: $\fundef[\pi_1]{E_1}{M}$ e  $\fundef[\pi_2]{E_2}{M}$.
Siano $\set{ \fundef[\mu_{ij}]{U_i \cup U_j}{G} }$ e $\set{ \fundef[\lambda_{ij}]{U_i \cup U_j}{G} }$ rispettivamente coclici di $E_1$ e $E_2.
$E_\$ ed $E_2$ sono fibrati equivalenti (nel senso dei coclici) se $\exists \fundef[\gamma_{i}]{U_i}{G}$ tale che $\forall x U_i \cup U_j$ dati $\lambda_{ij}$ e $\mu_{ij}$
sia $\lamda_{ji} = {\gamma_j}^{-1} \times \mu_{ji} \times  {\gamma_i}^{-1}$.  (Quest'ultima è la moltiplicazione tra elementi del gruppo G)
\end{defn}

\begin{defn}[Immersione liscia]
Sia $\fundef[f]{X}{Y}$, $f$ è una immersione se $\forall x \in X \fundef[Df_{x}]{T_{x}X}{T_{f(x)} X}$ (restrizione dell'applicazione tangente alla fibra) è iniettiva.
\end{defn}

\begin{defn}[Embedding] 
 Una immersione $f$ è un embedding se $\fundef[f]{X}{f(X)}$  è un omomorfismo (funzione continua fra spazi topologici tale che ha inversa continua).
\end{defn}

%Nn mi è chiaro cosa volesse dire con questa frase...
%Se $X$ è compatta e $f$ è iniettiva e suriettiva allora $f$ è un embedding.

\begin{defn}[Metrica Riemanniana] %a caso
 Una metrica riemanniana su una varietà $X$ è un campo di tensori di tipo (0,2) su $X$ simmetrici e definiti positivi. Cioè è una sezione tale che ogni tensore è simmetrico 
 e definito positivo: $\fundef[R]{X}{{T_{2}}^{0}}$ 
\end{defn}

\begin{cor}
 Ogni X compatta ha una metrica Riemanniana.
\end{cor}
\begin{dim}

\begin{teo}[Di embedding elementare di Whitney]
 Se $X$ è una varietà compatta allora esiste un $N$ abbastanza grande tale che \exist un embedding $\fundef[f]{X}{\mathbb{R}^{N}}$
\end{teo}

\begin{dim}
 Poiché $X$ è compatta esiste un atlante finito $\set{(U_i, \phi_i)}$ tale che 
 \begin{itemize}
  \item $B_2(0) \subseteq {\phi}_{i}(U_{i})$
  \item ${\phi_{i}(B_1(0))}$ ricopre tutto X
  \end{itemize}
 Sia $\fundef[\lambda]{\mathbb(R)^N}{[0, 1]}$ la funzione a foruncolo relativa alle palle $B_{1}$ e $B_{2}$. Definisco la funzione $\fundef[\lambda_{i}]{\mathbb(R)^N}{[0, 1]}$
 come $\lambda \circ \phi_{i}$ in $U_{i}$ e $0$ in $X \backslash U_{i}$.
 
 ${B_{i}}$ ricopre $X$ e $B_{i} = {\lambda}^{-1} \subseteq U_{i}$. Poiché $\lambda_i$ vale 1 quando sono su ${\phi_{i}}^{-1} (B_{1}(0))$ ho un ricoprimento.
 
 Definisco $\fundef[f]{X}{\mathbb(R)^N}$ come $\lambda_{i} \circ \phi_{i}$ in $U_{i}$ e $0$ in $X \backslash U_{i}$. Considero le funzioni $\fundef[g_{i}]{X}}{\mathbb{R}^{N}}$
 definita come $x \rightarrow (f_{i}(x), \lambda_{i}(x))$ . Definisco ancora la funzione $g$ che : $x \rightarrow (g_1, g_2, ..., g_m)$
 
 Dico che $g$ così definita è un embedding, devo dunque dimostrare che è una immersione (cioè che il suo differenziale è iniettivo) e che la funzione stessa $f$ è omomorfismo cioè
 ha funzione inversa continua.
 
 \begin {itemize}
  \item è diffeomorfismo per costruzione
  \item tutte le $g_{i}$ per come son state costruite sono immersioni
  \item g è iniettiva: se ho $x \notequal y$  ho due possibilità: $x, y \in B_{i}$ allora la $f_i$ coincide con la $\phi_i$ sulla palla $B_{i}$,che è iniettiva sulla palla. 
  oppure $x, y$ stanno in due palle diverse in particolare $y \in B_{i}$ allora $\lambda_{i}(y) = 1$ e $\lambda_{i}(x)$ 
  
 \end {itemize}
\end{dim}

Se ho una varietà $X$ con bordo $\delta X \notequal 0$ si può rafforzare la costruzione in modo che (X, $\delta X$) sia una sottovarietà del semipiano 
$(mathbb{H}^{N}, \delta mathbb{H}^{N})$ con $\delta X = X \cup \delta \mathbb{H}^{N}$ e su $\delta X$ ho $X \perp \delta \mathbb{H}^{N}$.
Inoltre se $x_{1}, x_{2}, x_{3}, ..., x_{k}$ sono punti della varietà con bordo non nullo si può rafforzare la costruzione in modo che un intorno di $x_{j}$ va in un piano di $\mathbb{R}^{N}$.

\titletit{Spazi di applicazioni lisce}
\begin{def}[Topologia sulle applicazioni lisce tra varietà in $\mathbb{R}^{N}$]
Definisco $\mathcal{E} = {\fundef[f]{X}{Y} | f liscia}$, con $X$ e $Y$ varietà lisce. Voglio munire $X$ e $Y$ di una topologia. Se $X \subseteq \mathbb{R}^{N}$ e $Y = \mathbb{R}^{N}$ considero
la base di intorni data ${U_{r, k, \epsilon}$ e $r \in \mathbb{N}$ e $K \subseteq X$ e $\epsilon > 0$ è un compatto. Questa base di intorni è definita da 
${U_{r, K, \epsilon} = {\fundef[g]{X}{\mathbb{R}^{N}}$ tale che $||\fract{\delta f}{\delta x_{i_{1}} ... \delta x_{i_{k}}} - \fract{\delta g}{\delta x_{i_{1}} ... \delta x_{i_{k}}}|| \eqslantless \epsilon$
\begin{def}[Topologia sulle applicazioni lisce tra varietà generiche]
Per definire una topologia tra varietà generiche mi riduco al caso di varietà reali.
Definisco una base di intorni ${U_{r, K, \epsilon, (U, \phi), (U', \phi')} = {\fundef[g]{X}{\mathbb{R}^{N}}$ tale che $g(U) \subseteq U'$ e valga 
$\phi \circ g \circ \phi^{-1} \in U_{r, K, \epsilon}(phi \circ g \circ \phi^{-1})$ cioè che $phi \circ g \circ \phi^{-1}$ sia intorno di una funzione da $\mathbb{R}^{N}$ a $\mathbb{R}^{N}$.
\end{def}

\begin{oss}
 La topologia che ottengo è metrizzabile.
\end{oss}







