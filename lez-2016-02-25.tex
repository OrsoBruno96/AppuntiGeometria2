%autore: Andrea


% questa parte si può anche dare per nota, la commento
% In questa sezione si userà la nozione di immagine e di preimmagine di un sottoinsieme. In generale siano dati $X$ e $Y$ insiemi, $\fundef[f]{X}{Y}$ una funzione fra essi.\\
% Dato un sottoinsieme $X'$ di $X$ definisco immagine di $X'$ secondo $f$: \marginpar{come lo uso qui l'ambiente \texttt{defn}?}
% \begin{equation*}
% \img_f(X') \is f(X') \is \setdef[y \in Y]{\exists x \in X' \et f(x) = y}
% \end{equation*}
% Dualmente sia $Y'\subseteq Y$ definisco preimmagine di $Y'$ : \marginpar{dovrò ricordarmi di non saltare tra preimmagine e controimmagine}
% \begin{equation*}
% \img_f^{-1}(Y') \is f^{-1}(Y') \is \setdef[x\in X]{f(x)\in Y'}
% \end{equation*}
% Si noti che la definizione è coerente anche non chiedendo che $f$ sia invertibile!
% % \vspace{.8cm} ma non andrebbero messi questi spazi, se ti serve una spaziatura grossa per ragioni logiche forse non stai fittando il contenuto con la struttura titoli-paragrafi-teoremi-formule, ma vabbé non ho voglia di discuterne in realtà

\subtitlet{Compattezza e funzioni continue}

\begin{lemma}
	L'immagine continua di un compatto è compatta.
\end{lemma}

\begin{proof}
	Sia $X$ compatto e $\fundef XY$ continua. Vogliamo dimostrare che $f(X)$ con la topologia di sottospazio è compatto.

	Prendiamo dunque un ricoprimento aperto $\mathcal A$ di $f(X)$.
	Consideriamo $\mathcal A_X\is\setdef[f^{-1}(A)]{A\in\mathcal A}$.
	Per definizione di continuità, gli elementi di $\mathcal A_X$ sono aperti, mentre per ragioni insiemistiche ricoprono $X$.
	Allora da $\mathcal A_X$ posso estrarre un sottoricoprimento finito $\mathcal A_X^F$.
	Tornando indietro, l'insieme $\setdef[f(A)]{A\in\mathcal A_X^F}$ è un sottoricoprimento finito di $\mathcal A$.
\end{proof}

\begin{lemma}
	L'immagine continua di un compatto per successioni è compatta per successioni.
\end{lemma}

\begin{proof}
	Siano $X$ compatto per successioni e $\fundef XY$ continua, vogliamo dimostrare che $f(X)$ è compatto per successioni.
	
	Sia $(y_n)_{n\in N}$ una successione in $Y$. Per ognuno degli $y_n$ scelgo un $x_n\in f^{-1}(y_n)$, definendo così una successione in $X$.
	Estraggo da essa una sottosuccessione convergente $(x_{n_k})$ (esiste poiché $X$ è compatto per successioni).
	Tornando indietro, $(f(x_{n_k}))$ è una sottosuccessione di $(y_n)$ e converge per continuità di $f$.
\end{proof}

\begin{lemma}
	Se $\fundef X\R$ è continua e $X$ è compatto allora $f$ ammette massimo e minimo assoluti.
\end{lemma}

\begin{proof}
	L'immagine continua di compatti è compatta, dunque $f(X)$ è compatto. Ma in $\R$ compatto equivale a chiuso e limitato. Dalla limitatezza segue che esiste, in $\R$, estremo superiore e inferiore, dalla chiusura che essi sono in $f(X)$.
\end{proof}

\begin{lemma}
	Se $X$ è compatto e  $\fundef XY$ è continua e invertibile, allora $f^{-1}$ è continua (corollario: è un omeomorfismo).
\end{lemma}

\begin{proof}
	La dimostrazione si spezza in due passi:
	per prima cosa si dimostra che $F^{-1}$ è continua $\iff F$ è chiusa (i.e. $F$ manda chiusi in chiusi).

	Prendo un aperto $A\subseteq X$. Il suo complementare $\comp A$ è un chiuso di $X$, allora la sua immagine $F(\comp A)$ è un chiuso di $Y$. Allora il complementare  $\comp{F(\comp A)}$ è un aperto ma per invertibilità $\comp{F(\comp A)}=F(A)$, dunque controimmagine di aperti di $X$ secondo $F^{-1}$ sono aperti.

	Viceversa preso un chiuso $C \subseteq X$ il suo complementare è aperto, mandato nella controimmagine tramite $F^{-1}$ rimane aperto. Si può dunque passare al complementare e ottenere la tesi.

	Ma se $X$ è compatto un chiuso $K\subseteq X$ è compatto. Dato che immagine continua di compatti è compatta $F(K)$ è un compatto di $Y$, dunque è anche un chiuso.
\end{proof}

\titlet{Topologia prodotto e topologia quoziente}
Ora si possono definire delle operazioni fra spazi topologici che ci permettano di costruirne altri in maniera ``naturale''.

\begin{defn}[Topologia prodotto]
Siano $(X,\tau)$ e $(Y, \sigma)$ spazi topologici. Definiamo il prodotto $X\times_T Y$ lo spazio topologico definito sul prodotto cartesiano $X\times Y$ con la topologia meno fine possibile tale che date $\fundef[p_1]{X\times Y}{X}$ e $\fundef[p_2]{X\times Y}{Y}$ le proiezioni canoniche:
\begin{equation}
 p_1((x, y))=x \text{  e  } p_2((x, y))=y
\end{equation} 
siano continue.
\end{defn}

\begin{oss}
Si vede come la condizione sulle proiezioni non sia impossibile da soddisfare. Infatti la topologia discreta sul prodotto cartesiano soddisfa banalmente.

Non è altrettanto palese che effettivamente esista una topologia minimale rispetto all'inclusione che soddisfi il requisito. Il punto non è stato chiarito a lezione. È chiaro comunque che la topologia di cui stiamo parlando è quella generata dalle preimmagini secondo $p_1$ o $p_2$ di aperti di $X$ o di $Y$.
\marginpar{non sarebbe male esplicitare che è $\setdef[A_X\times A_Y]{...}$}
\end{oss}

\begin{oss}
Tutte le proprietà comuni ai due spazi di partenza si propagano allo spazio prodotto (fatto non dimostrato in classe), in particolare il prodotto di spazi compatti è compatto, il prodotto di spazi connessi è connesso, il prodotto di spazi $T_2$ è $T_2$.
\end{oss}

\begin{defn}[Topologia quoziente]
Sia $(X,\tau)$ uno spazio topologico, $Y$ un insieme e $\fundef{X}{Y}$ una funzione surgettiva.

Definiamo la relazione di equivalenza $\sim_f$ come $a\sim_f b \means f(a)=f(b)$ (si verifica che è una relazione di equivalenza) e definiamo $\quoset X{\sim_f}$ l'insieme quoziente rispetto a questa relazione di equivalenza (moralmente questo insieme è proprio $Y$ in quanto preimmagini di singoli elementi in $Y$ sono classi di equivalenza in $X$).

Definiamo infine $\Tquoset X{\sim_f}$ lo spazio topologico definito su $\quoset X{\sim_f}$ dotato della topologia più fine possibile che renda l'immersione $x \mapsto [x]$ (moralmente $f$) continua.
\end{defn}

\begin{oss}
Anche questa volta, se l'esistenza di topologie che rendano continua l'immersione è palese (basti pensare alla topologia banale), non è affatto scontato che esista una ``topologia più fine possibile''.
\end{oss}

\begin{defn}[Insieme $f$-saturo]
Data una funzione $\fundef{X}{Y}$ si dice $f$-saturo un $X'\subseteq X$ tale che $f^{-1}(f(X'))=X'$.
\end{defn}

\begin{prop}
La topologia di $\Tquoset X{\sim_f}$ è data da tutte le immagini attraverso $f$ di aperti $f$-saturi di $X$.
\end{prop}

\begin{oss} \marginpar{queste sarebbero da controllare!!!}
Non tutte le proprietà si propagano al quoziente. Fortunatamente si propagano la connessione e la compattezza.
\end{oss}



\titlet{Varietà topologiche}


Ora anticipiamo un po' di quello che faremo la prossima volta (non so, forse lo ha fatto solo per rendere un po' meno organiche queste note!)!!

\marginpar{!!!!!!!!!!!!!!}

\begin{defn}[Varietà topologica $n$-dimensionale]
Fissato un intero $n$ una varietà topologica $n$-dimensionale è uno spazio topologico $V$ $T_2$ localmente omeomorfo ad aperti di $R^n$.

Più propriamente si definisce varietà topologica la coppia $(V, M)$ dove $V$ è lo spazio topologico e $M$ è l'insieme degli omeomorfismi.
\end{defn}
