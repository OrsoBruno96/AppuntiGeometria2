%autore: Andrea


%\begin{defn}
In questa sezione si userà la nozione di immagine e di preimmagine di un sottoinsieme. In generale siano dati $X$ e $Y$ insiemi, $\fundef[f]{X}{Y}$ una funzione fra essi.\\
Dato un sottoinsieme $X'$ di $X$ definisco immagine di $X'$ secondo $f$: 
\begin{equation*}
Img_f(X') = f(X') =\set{y \in Y | \exists x \in X' \wedge f(x) = y} 
\end{equation*}
Dualmente sia $Y'\subseteq Y$ definisco preimmagine di $Y'$ :
\begin{equation*}
Img_f^{-1}(Y')=f^{-1}(Y')=\set{x\in X | f(x)\in Y'}
\end{equation*}
Si noti che la definizione è coerente anche non chiedendo che $f$ sia invertibile!
%\end{defn}
\\
\\
\begin{lemma}
Immagine continua di compatti è compatta.
\end{lemma}

\begin{proof}
Siano $(X, \tau_X)$ e $(Y, \tau_Y)$ spazi topologici, inoltre sia $X$ compatto e sia $\fundef[f]{X}{Y}$ continua. Noi voglimo dimostrare che $Img(X)$ come spazio topologico è compatto.\\ 
Prensimo dunque un ricoprimento aperto $A$ di $Y\cap f(X)$. Ogni aperto di $A$ deve essere immagine di un aperto di $X$ per definizione di continuità. Ovviamente l'insieme delle preimmagini di tutti gli aperti di $f(X)$ ricopre $X$, dunque posso estrarne un sottoricoprimento finito. Questo sottoricoprimento copre $X$, inoltre tutte le immagini di elementi di questo sottoricoprimento sono aperti in $Y\cap f(X)$ (sono in $A$). L'insieme delle immagini di queste preimmagini ricoprono inoltre $Y\cap f(X)$ (ricoprendo $X$).
\end{proof}

\begin{lemma}
Se $X$ è compatto per successioni e $\fundef[F]{X}{Y}$ è una funzione continua, allora $F(X)$ è compatto per successioni.
\end{lemma}
\begin{proof}
Sia $(y_n)_{n\in N}$ una successione in $Y$. Per ogniuno degli $y_n$ scelgo un $x_n$ in $X$ fra le sue preimmagini, definendo così una successione in $X$. Estraggo da essa una sottosuccessione convergente (esiste poichè $X$ è compatto per successioni). La mando in $Y$ con $F$. Qusta nuova successione è convergente (se $x_n \Rightarrow x$ allora $f(x_n)\Rightarrow f(x)$ per continuità di $F$) ed è una estratta di   $(y_n)_{n\in N}$.
\end{proof}

\begin{lemma}
Se $\fundef[F]{X}{R}$ è continua e $X$ è compatto allora $F$ ammette massimo e minimo assoluti.
\end{lemma}
\begin{proof}
Immagine continua di compatti è compatta, dunque $F(X)$ è compatto. Ma in $R$ essere compatti è equivalente a essere chiusi e limitati. Dalla limitatezza segue che esiste, in $R$, massimo e minimo, dalla chiusura che essi sono in $F(X)$.
\end{proof}

\begin{lemma}
Se X è un compatto e  $\fundef[F]{X}{Y}$ una funzione continua e invertibile, allora $F^{-1}$ è continua (è un omeomorfismo).
\end{lemma}
\begin{proof}
La dimostrazione si spezza in due passi:\\
Per prima cosa si dimostra che $F^{-1}$ è continua $\implies F$ è chiusa (i.e. $F$ manda chiusi in chiusi).\\ 
Prendo un aperto $A\subseteq X$. Il suo complementare $A^c$ è un chiuso di $X$, allora la sua immagine $F(A^c)$ è un chiuso di $Y$. Allora il complementare  $F(A^c)^c$ è un aperto ma per invertibilità $F(A^c)^c=F(A)$, dunque controimmagine di aperti di $X$ secondo $F^{-1}$ sono aperti.\\
Vicevers preso un chiuso $K \subseteq X$ il suo complementare è aperto, mandato nella controimmagine tramite $F^{-1}$ rimane aperto. Si può dunque passare al complementare e ottenere la tesi.\\
Ma se $X$ è compatto un chiuso $X'\subseteq X$ è compatto. Dato che immgine continua di compatti è compatta $F(X')$ è un compatto di $Y$, dunque è anche un chiuso.    
\end{proof}
\titlet{Topologia prodotto e topologia quoziente}
Ora si puossono definire delle operazioni fra spazi topologici che ci permettano di costruirne altri in maniera "naturale".\\

\begin{defn}
(Topologia prodotto)\\
Siano $(X,\tau)$ e $(Y, \sigma)$ spazi topologici. Definiamo il prodotto $X\times_T Y$ lo spazio topologico definito sul prodotto cartesiano $X\times Y$ con la topologia meno fine possibile tale che date $\fundef[p_1]{X\times Y}{X}$ e $\fundef[p_2]{X\times Y}{Y}$ le proiezioni canoniche:
\begin{equation}
 p_1((x, y))=x \text{  e  } p_2((x, y))=y
\end{equation} 
siano continue.
\end{defn}

\begin{oss}
Si vede come la condizione sulle proiezioni non sia impossibile da soddisfare. Infatti la topologia discreta sul prodotto cartesiano soddisfa banalmente.\\
Non è altrettanto palese che effettivamente esista una topologia minimale rispetto all'inclusione che soddisfi il requisito. Il punto non è stato chiarito a lezione. \'E chiaro comunque che la topologia di qui stiamo parlando è quella generata dalle preimmagini secondo $p_1$ o $p_2$ di aperti di $X$ o di $Y$.\\
\end{oss}

\begin{oss}
Tutte le proprietà comuni ai due spazi di partenza di propagano allo spazio prodotto (fatto non dimostrato in classe), in particolare il prodotto di spazi compatti è compatto, il prodotto di spazi connessi è connesso, il prodotto di spazi $T_2$ è $T_2$.
\end{oss}

\begin{defn}
(Topologia quoziente)\\
Sia $(X,\tau)$ uno spazio topologico, $Y$ un insieme e $\fundef{X}{Y}$ una funzione surgettiva. \\
Definiamo la relazione di equivalenza $\sim_f$ come $a\sim_f b \implies f(a)=f(b)$ (si verifica che è una relazione di equivalenza) e definiamo $X/\sim_f$ l'inisieme quoziente rispetto a questa relazione di equivalenza( moralmente questo insieme è proprio $Y$ inquanto preimmagini di singoli emementi in $Y$ sono classi di equivalenza in $X$).\\
Definiamo infine $X/_T \sim_f$ lo spazio topologico definito su $X/\sim_f$ dotato della topologia più fine possibile che renda l'immersione $x \rightarrow [x]$ (moralmente $f$) continua.
\end{defn}

\begin{oss}
Anche questa volta, se l'esistenza di topologie che rendano continua l'immersione è palese (basti pensare alla topologia banale), non è affatto scontato che esista una "topologia più fine possibile".\\
\end{oss}

\begin{defn}
(insieme f-saturo)\\
Data una funzione $\fundef{X}{Y}$ si dice f-saturo un $X'\subseteq X$ tale che $f^{-1}(f(X'))=X'$.
\end{defn}

\begin{prop}
La topologia di $X/\sim_t f$ è data da tutte le immagini attraverso $f$ di aperti f-saturi di $X$.
\end{prop}

%queste sarebbero da controllare!!!
\begin{oss}
Non tutte le proprietà si propagano al quoziente. Fortunatamente si propagano la connessione e la compattezza.
\end{oss}



\titlet{Varietà topologiche}


Ora anticipiamo un po' di quello che faremo la prossima volta (non so, forse lo ha fatto solo per rendere un po' meno organiche queste note!)!!



\begin{defn}
(Varietà topologica n-dimensionale)\\
Fissato un intero $n$ una varietà topologica n-dimensionale è uno spazio topologico $V$ $T_2$ localmente omeomorfo ad aperti di $R^n$.\\
Più propriamente si definisce varietà topologica la coppia $(V, M)$ dove $V$ è lo spazio topologico e $M$ è l'insieme degli omeomorfismi.\\ 
\end{defn}
