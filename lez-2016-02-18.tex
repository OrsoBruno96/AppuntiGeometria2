% trascrizione: Petrillo

\begin{lemma}
	2-numerabile $\implies$ 1-numerabile
\end{lemma}

\begin{proof}
	Sia $\mathcal B$ una base di aperti di $X$ numerabile.
	$\forall x\in X$ sia $\mathcal B_x\is\setdef[A\in\mathcal B]{x\in A}$.
	$\mathcal B_x$ è numerabile ed è una base di intorni di $x$.
\end{proof}

\begin{defn}[Densità]
	Un sottospazio è denso se interseca tutti gli aperti:
	\[\text{$Y$ denso in $X$}\means
	\begin{cases}
		Y\subseteq X\\
		\forall A\text{ aperto},A\neq\nullset:A\cap Y\neq\nullset
	\end{cases}\]
\end{defn}

\begin{lemma}
	\label{th:metrdens2num}
	Uno spazio metrizzabile che contiene un denso numerabile è 2-numerabile:
	\[\begin{rcases}
		X\text{ metrizzabile}\\
		\text{$Y$ denso in $X$ e numerabile}
	\end{rcases}\implies
	X\text{ 2-numerabile}\]
\end{lemma}

\begin{proof}
	\marginpar{FIGURA}
	Mostriamo che $\setdef[{\ball[1/n]y}]{y\in Y\et n\in\N}$ è una base di aperti di $X$.
	Ogni aperto $A$ si può scrivere come $A=\union_{x\in A}\ball[r_x]x$.
	Consideriamo $\ball[r_x/3]x$: è un aperto non vuoto quindi $\exists\,y_x\in Y\cap\ball[r_x/3]x$.
	Scegliamo $n_x$ tale che $\frac r3\le\inv{n_x}\le\frac{2r}3$
	e poniamo $A'\is\union_{x\in A}\ball[1/n_x]{y_x}$.
	Allora $A'\subseteq A$ perché $\ball[1/n_x]{y_x}\subseteq\ball[r_x]x$
	e $A\subseteq A'$ perché $x\in\ball[1/n_x]{y_x}$,
	quindi $A=A'$.
\end{proof}

\begin{es}
	$\Q$ è denso in $\R$ e numerabile, quindi $\R$ è 2-numerabile.
	Si noti che nel dimostrare il \autoref{th:metrdens2num} abbiamo usato la densità di $\Q$.
\end{es}

Costruiamo ora una topologia che sia 1-numerabile ma non 2-numerabile:

\begin{defn}
	Sia $\tau_s$ una topologia su $\R$ generata dagli intervalli chiusi a sinistra e aperti a destra:
	\[\tau_s\is\Setdef[\union\mathcal I]{\mathcal I\subseteq\setdef[[a;b)]{a\leq b}}\]
\end{defn}

\begin{prop}
	$\tau_s$ è più fine di $\tau_E$.
\end{prop}

\begin{proof}
	Infatti la topologia euclidea è generata dagli intervalli aperti,
	e ogni intervallo aperto si può scrivere come:
	\[(c;d)=\union_{\substack{a>c\\a<d}}[a;d)
	\qedhere\]
\end{proof}

\begin{prop}
	$\tau_s$ è 1-numerabile.
\end{prop}

\begin{proof}
	Infatti $\Setdef[\left[x;x+\inv n\right)]{n\in\N}$ è una base di intorni di $x$ numerabile.
\end{proof}

\begin{prop}
	$\Q$ è denso in $\R$ secondo $\tau_s$.
\end{prop}

\begin{proof}
	Segue banalmente dalla densità di $\Q$ secondo $\tau_E$.
\end{proof}

\begin{prop}
	$(\R,\tau_s)$ è $T_2$.
\end{prop}

\begin{proof}
	Presi $x<y\,\exists z:x<z<y$ quindi $[x;z)\cap[y;\infty)=\nullset$.
\end{proof}

\begin{prop}
	$(\R,\tau_s)$ non è 2-numerabile
\end{prop}

\begin{proof}
	Sia $\mathcal B$ una base di aperti.
	Abbiamo che:
	\[\forall x\in\R\,\exists B_x\in\mathcal B:
	\begin{cases}
		x\in B_x\\
		B_x\subseteq [x;\infty)
	\end{cases}\]
	Quindi $y\neq x\so B_y\neq B_x$,
	cioè l'applicazione $(x\mapsto B_x)$ da $\R$ in $\mathcal B$ è iniettiva,
	ovvero $\card B\geq\card\R$.
\end{proof}

\begin{oss}
	Quindi 1-numerabilità non implica 2-numerabilità.
\end{oss}

Seguono due proposizioni analoghe a quelle mostrate per la proprietà $T_2$:

\begin{prop}
	Le proprietà di numerabilità passano ai sottospazi.
\end{prop}

\begin{prop}
	Le proprietà di numerabilità sono invarianti per omeomorfismo.
\end{prop}

Gli spazi numerabili si possono studiare usando le successioni, cioè le applicazioni con dominio $\N$.
Indichiamo con $(a_n)$ l'applicazione $(n\mapsto a_n)$:

\begin{defn}[Convergenza]
	Si dice che $(a_n)$ \emph{converge a $x$} se è definitivamente contenuta in ogni intorno di $x$:
	\[a_n\convarrow x\means
	\forall U_x\,\exists N\,\forall n\geq N:a_n\in U_x\]
\end{defn}

% gli intorni annidati li ha fatti il 2016-02-22 ma solo per completare questa dimostrazione, quindi li metto qui.
Diamo ora una definizione che ci servirà per dimostrare la \autoref{th:1numaccsucc}:

\begin{defn}
	Una \emph{base di intorni annidati} è una base di intorni numerabile ordinata per inclusione:
	\[\set{U_n}_{n\in\N}\text{ base di intorni annidati}\means
	\begin{cases}
		\set{U_n}\text{base di intorni}\\
		\forall n:U_{n+1}\subseteq U_n
	\end{cases}\]
\end{defn}

\begin{prop}
	Una base di intorni numerabile induce una base di intorni annidati.
\end{prop}

\begin{proof}
	Sia $\set{U_n}_{n\in\N}$ una base di intorni.
	Definiamo $U'_k=\inters_{n=0}^kU_n$.
	$U'_k$~è un intorno perché è un'intersezione finita di intorni.
	$\set{U'_k}_{k\in\N}$ è una base di intorni annidati perché gli $U'_k$ sono tutti contenuti negli $U_n$, e per costruzione sono annidati.
\end{proof}

\begin{prop}
	\label{th:1numaccsucc}
	In uno spazio 1-numerabile, i punti di accumulazione di un insieme sono quelli a cui converge una sottosuccessione a valori nell'insieme:
	\[\begin{rcases}
		X\text{ 1-numerabile }\\
		Y\subseteq X\\
		x\in X
	\end{rcases}\implies
	\big(x\text{ accumulazione per }Y\iff
	\exists(a_n)\subseteq Y:a_n\convarrow x\big)\]
\end{prop}

\begin{proof}
	Mostriamo le due implicazioni:
	\begin{description}
		\item[\proofrightarrow]
			Sia $\set{U_n}_{n\in\N}$ una base di intorni annidati di $x$.
			Poiché $x$ è di accumulazione per $Y$,
			$\forall n\,\exists a_n\in(U_n\setminus\set x)\cap Y$.
			La successione $(a_n)$ converge a $x$ perché gli $U_n$ sono annidati.
		\item[\proofleftarrow]
			Infatti in ogni intorno di $x$ ci saranno infiniti $a_n\in Y$.
		\qedhere
	\end{description}
\end{proof}

\titlet{Connessione}

\begin{defn}[Connessione]
$X$ sconnesso $\means\neg(X\text{ connesso})\means\exists A,B\text{ aperti}:X=A\cup B\et A\cap B=\nullset\et A,B\neq\nullset$
\end{defn}

\begin{prop}
$X\text{ connesso}\et A\subseteq X\et A\text{ aperto e chiuso}\implies A=X\vel A=\nullset$
\end{prop}

\begin{prop}
$Y\subseteq\R$ connesso secondo $\tau_E\iff Y$ è un intervallo
\end{prop}
\begin{proof}
Mostriamo l'implicazione verso sinistra; per assurdo:
\begin{align*}
Y\text{ sconnesso}\so&Y=A_1\cup A_2\text{ aperti disgiunti}\\
&\exists x_i\in A_i\wlg x_1<x_2\\
Y\text{ intervallo}\so&[x_1;x_2]\subseteq Y\\
&\xi\is\sup(A_1\cap[x_1;x_2])\so\\
\so&\xi\in\clos A_1,\clos A_2\os\xi\text{ accumulazione}\\
&\text{Ma }\clos A_i=A_i\absurd
\end{align*}
Mostriamo l'implicazione verso destra; per assurdo:
\begin{align*}
&Y\text{ non è un intervallo}\so\\
\so&\exists x_0\in(\inf Y;\sup Y):x_0\not\in Y\so\\
\so&Y=(Y\cap(-\infty;x_0))\cup(Y\cap(x_0;\infty))\absurd \qedhere
\end{align*}
\end{proof}
