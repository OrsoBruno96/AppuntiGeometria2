% trascrizione: Petrillo

\begin{lemma}
2-numerabile $\then$ 1-numerabile
\end{lemma}
\begin{proof}
Sia $\mathcal B$ una base di aperti di $X$ numerabile. $\forall x\in X$ sia $\mathcal B_x\is\setdef[A\in\mathcal B]{x\in A}$. $\mathcal B_x$ è numerabile ed è una base di intorni di $x$.
\end{proof}

\begin{defn}[Densità]
$Y$ denso in $X\means Y\subseteq X\et\forall A\text{ aperto}:A\cap Y\neq\emptyset$
\end{defn}

\begin{lemma}
$X$ metrizzabile $\et Y$ denso in $X$ e numerabile $\then X$ 2-numerabile
\end{lemma}
\begin{proof}
$\setdef[{\ball[1/n]y}]{y\in Y\et n\in\N}$ è una base di aperti di $X$ numerabile.
\end{proof}

\begin{es}
$\Q$ è denso in $\R$ e numerabile, quindi $\R$ è 2-numerabile.
\end{es}

\begin{defn}
Sia $\tau_s$ una topologia su $\R$: $\tau_s\is\Setdef[\union I]{I\subseteq\setdef[[a;b)]{a\leq b}}$
\end{defn}

\begin{prop}[Finezza di $\tau_s$]
$\tau_E\subseteq\tau_s$
\end{prop}
\begin{proof}
Infatti $(c;d)=\union_{\substack{a>c\\a<d}}[a;d)$.
\end{proof}

\begin{prop}
$\tau_s$ è 1-numerabile
\end{prop}
\begin{proof}
Infatti $\Setdef[\left[x;x+\inv n\right)]{n\in\N}$ è una base di intorni di $x$ numerabile.
\end{proof}

\begin{prop}
$\Q$ è denso in $\R$ secondo $\tau_s$
\end{prop}

\begin{prop}
$(\R,\tau_s)$ è separabile
\end{prop}

\begin{prop}
$(\R,\tau_s)$ non è 2-numerabile
\end{prop}
\begin{proof}
\begin{align*}
&\forall\mathcal B\text{ base di aperti},x\in\R:\\
&\exists B_x\in\mathcal B:x\in B_x\et B_x\subseteq [x;\infty)\\
&\forall y\neq x:B_y\neq B_x\so\\
\so&\fundef[(x\mapsto B_x)]\R{\mathcal B}\text{ è iniettiva}\so\card B\geq\card\R \qedhere
\end{align*}
\end{proof}

\begin{oss}
Quindi 1-numerabilità non implica 2-numerabilità.
\end{oss}

\begin{prop}
Le proprietà di numerabilità passano ai sottospazi
\end{prop}

\begin{prop}
Le proprietà di numerabilità sono invarianti per omeomorfismo
\end{prop}

\begin{defn}[Convergenza]
$(a_n)$ converge a $x\means(a_n)\convarrow x\means\forall U(x)\,\exists N\,\forall n\geq N:a_n\in U(x)$
\end{defn}

\begin{prop}
$X\text{ 1-numerabile }\et Y\subseteq X\et x\in X\then(x\text{ accumulazione per }Y\leftrightarrow\exists(a_n)\subseteq Y:a_n\convarrow x)$
\end{prop}
\begin{proof}
Mostriamo l'implicazione verso destra:
\begin{align*}
x\text{ accumulazione per }Y\so&\forall U(x):(U\setminus\set x)\cap Y\neq\emptyset\\
X\text{ 1-numerabile}\so&\exists\setdef[U_n(x)]{n\in\N}\text{ base di intorni di }x\text{ numerabile}\\
\so&\forall n\in\N\,\exists a_n\in(U_n\setminus\set x)\cap Y \qedhere
\end{align*}
\end{proof}

\titlet{Connessione}

\begin{defn}[Connessione]
$X$ sconnesso $\means\neg(X\text{ connesso})\means\exists A,B\text{ aperti}:X=A\cup B\et A\cap B=\emptyset\et A,B\neq\emptyset$
\end{defn}

\begin{prop}
$X\text{ connesso}\et A\subseteq X\et A\text{ aperto e chiuso}\then A=X\vel A=\emptyset$
\end{prop}

\begin{prop}
$Y\subseteq\R$ connesso secondo $\tau_E\iff Y$ è un intervallo
\end{prop}
\begin{proof}
Mostriamo l'implicazione verso sinistra; per assurdo:
\begin{align*}
Y\text{ sconnesso}\so&Y=A_1\cup A_2\text{ aperti disgiunti}\\
&\exists x_i\in A_i\wlg x_1<x_2\\
Y\text{ intervallo}\so&[x_1;x_2]\subseteq Y\\
&\xi\is\sup(A_1\cap[x_1;x_2])\so\\
\so&\xi\in\clos A_1,\clos A_2\os\xi\text{ accumulazione}\\
&\text{Ma }\clos A_i=A_i\absurd
\end{align*}
Mostriamo l'implicazione verso destra; per assurdo:
\begin{align*}
&Y\text{ non è un intervallo}\so\\
\so&\exists x_0\in(\inf Y;\sup Y):x_0\not\in Y\so\\
\so&Y=(Y\cap(-\infty;x_0))\cup(Y\cap(x_0;\infty))\absurd \qedhere
\end{align*}
\end{proof}
