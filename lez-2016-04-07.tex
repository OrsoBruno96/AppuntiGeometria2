% trascrizione Bob

%gnegne

Date due varietà $X$ e $Y$ consideriamo $\xi(X,Y)=\{ \fundef{X}{Y} \; liscie \} $ con la topologia definita precedentemente. Restringiamoci al caso $X$ compatta.\\Consideriamo alcuni sottoinsiemi:
\begin{itemize}
\item le immersioni $Imm(X,Y) =\{\fundef{X}{Y} \; immersioni\}$\\ovvero $\forall x \in X \; \fundef[Df_x]{T_xX}{T_{f(x)}Y}$ è iniettiva \\(la funzione tangente ristretta alla fibra $x$ che manda lo spazio tangente a $x$ nello spazio tangente a $f(x)$ è iniettiva)
\item gli embedding $Emb(X,Y)=\{\fundef{X}{Y} embedding\}$\\ovvero, poichè $X$ compatta $f$ è un'immersione iniettiva.
\item i diffeomorfismi $Diff(X,Y)=\{\fundef{X}{Y} diffeomorfismi\}$
\end{itemize}

\begin{oss}
Questi insiemi possono anche essere vuoti.
\end{oss}

\begin{teo}
Questi elencati sono sottoinsiemi aperti di $\xi(X,Y)$
\end{teo}

Moralmente, se una funzione è "abbastanza vicina" a un'immersione (o embedding o diffeo) è essa stessa un'immersione (o embedding o diffeo).
\begin{proof}
\noindent
\begin{itemize}
\item $Imm(X,Y)$ La condizione di essere immersione è una condizione che si verifica sul comportamento di iniettività delle tangenti che è una condizione aperta. (La formalizzazione di questa idea è lasciata per esercizio)
\item $Emb(X,Y)$ Suppongo $g$ funzione "vicina" ad un embedding $f$. Per il punto precedente $g$ un immersione, resta da mostrare che è iniettiva. 
\\ Supponiamo per assurdo che sia falso. Allora potrò costruire una successione $g_n \rightarrow f$ e tali che esistano due successioni di punti distinti $x_n$ e $y_n$ tali cge $g_n(x_n) = g_n(y_n)$. 
Poichè $X$ compatto, posso estrarre due sottosuccessioni $x_n \rightarrow x_0$ e $y_n \rightarrow y_0$. Se per assurdo $x_0 \neq y_0$, valuto $f$ in questi punti e dovrei avere $f(x_0)=f(y_0)$, che assurdo poichè $f$ è iniettiva, ma allora $x_0=y_0$ e entrambe le serie convergono allo stesso punto $x_0$.
\\ Leggendo tutto attraverso una carta intorno a $x_0$ posso pensare $x_0$ e le successioni $x_n$ e $y_n$ in $\R^n$ .
Poichè $g_n(y_n)-g_n(x_n)=0$, per il teormea del valor medio $\exists z_n \; t.c.\; d_{z_n}g_n[y_n-z_n]=0$. Ho inoltre che $z_n \rightarrow x_0$ perchè $z_n \in [x_n,y_n]$.
\\ Considero adesso $v_n \is \frac{y_n-x_n}{\left \|y_n - x_n \right \|} \in S^{n-1}$. Poichè $S^n$ è compatto a meno di estrarre una sottosuccessione $v_n \rightarrow v_0 \in S^{n-1}$.
Segue che $d_{x_0}f[v_0]=0$, assurdo perchè anche il differenziale di f è iniettivo. $\absurd$
\item $Diff(X;Y)$ Per i punti precedenti presa $g$ funzione "vicina" a $f$ diffeomorfismo, $g$ è un embedding. Mi resta da mostrare la sugettività.\\
Poichè $g$ è un embedding $g(x) \subseteq Y$ è aperto, inoltre $X$ è compatta, quindi $g(x)$ è anche chiuso (nelle ipotesi in cui lavoriamo compatto = chiuso).
\\ Ma se $X$ e $Y$ sono connesse (ipotesi di comodo) allora $g(x) = Y$, ovvero $g$ è diffeomorfismo.\\
(In mancanza dell'ipotesi di comodo mi posso restringere alle singole componenti connesse in partenza e in arrivo...)
\end{itemize}
\end{proof}

\titlet{Orientazione}

Siano $\mathbb{B}$ e $\mathbb{B'}$ due basi di $\R^n$, e sia $M^{\mathbb{B}}_{\mathbb{B'}}$ la matrice di cambiamento di base.
Due basi inducono la stessa orientazione se $det M^{\mathbb{B}}_{\mathbb{B'}}>0$

\begin{prop}
Questa è una relazione di equivalenza sulle basi di $\R^n$
\end{prop}
\begin{proof}
Infatti grazie alle proprietà del determinante e Binet:
\begin{itemize}
\item $det M^{\mathbb{B}}_{\mathbb{B}} = det Id = 1$
\item $det M^{\mathbb{B}}_{\mathbb{B''}} = det ( M^{\mathbb{B}}_{\mathbb{B'}}  M^{\mathbb{B'}}_{\mathbb{B''}})= det  M^{\mathbb{B}}_{\mathbb{B'}} \cdot det  M^{\mathbb{B'}}_{\mathbb{B''}}$
\item $det M^{\mathbb{B}}_{\mathbb{B''}} = det  M^{\mathbb{B'}}_{\mathbb{B}}$
\end{itemize}
\end{proof}

Ci sono quindi due classi di equivalenza.

\begin{defn}[Orientazione]
Una \emph{orientazione} di $\R^n$ è una classe di equivalenza per la relazione prima definita.
\end{defn}

Estendiamo questa definizione alle varietà.

\begin{defn}[Atlante orientato]
Sia $X$ è una varietà e $A = \{U_j, \phi_j\}$ un atlante di $X$ (non necessariamente il massimale). Diciamo che tale \emph{atlante} è \emph{orientato} se il cociclo $\{\fundef[det \lambda_{ji}]{U_i \cap U_j}{\R_{\textbackslash\{0\}}}\}$ è in effetti a valori in $\R^+$.\\(ovvero lo stesso cociclo usato per definire il fibrato tangente, ristretto però all'atlante)
\end{defn}

\begin{defn}[Atlanti compatibili]
Due atlanti orientati (ammesso che esistano) sono \emph{compatibili} se la loro unione è un atlante orientato.
\end{defn}

\begin{defn}[Orientazione su varità]
Un' \emph{orientazione} su una varietà è determinata da un atlante orientato massimale. Una varietà è \emph{orientabile} se ammette un'orientazione, \emph{non orientabile} altrimenti.
\end{defn}

\begin{oss}
Se una varietà è orientabile potrebbe avere più di una orientazione
\end{oss}

\begin{prop}
Se una varietà è connessa e orientabile, allora ha esattamente due orientazioni.
\end{prop}

DA FINIRE


