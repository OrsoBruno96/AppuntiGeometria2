% autore: Costa

\begin{teo}
 L'immagine continua di connessi è connessa:
 \[\begin{rcases}
  \fundef[f]{X}{Y}\text{ continua} \\
  X\text{ connesso}
 \end{rcases}\implies
 f(X)\text{ connesso}\] 
\end{teo}

\begin{proof}
%  Non è restrittivo supporre $f(X)=Y$: infatti $\fundef[f]{X}{Y}$ continua $\Leftrightarrow \fundef[f]{X}{f(X)}$ continua.
 Per assurdo supponiamo $f(X) = B_1 \cup B_2, B_1,B_2$ aperti disgiunti non vuoti.
 $A_1\is f^{-1}(B_1), A_2\is f^{-1}(B_2)$ aperti per continuità di $f$.
 Inoltre sono non vuoti, disgiunti e $X=A_1\cup A_2$, assurdo.
\end{proof}

\begin{oss}
 La connessione è invariante per omeomorfismo, ma \emph{non} passa ai sottospazi. 
\end{oss}

\begin{defn}[Arco]
 Un arco in $X$ è un'applicazione $\fundef[\gamma]{[0,1]}{X}$ continua con topologia euclidea sul dominio.
\end{defn}

Diciamo che una arco $\gamma$ \emph{connette} due punti $x_0$, $x_1$ se $\gamma(0)=x_0$ e $\gamma(1)=x_1$.

\begin{defn}[Connessione per archi]
 Uno spazio topologico è \emph{connesso per archi} se ogni sua coppia di punti è connessa da un arco:
 \[\text{$X$ connesso per archi}\means
 \forall x_0, x_1 \in X \,\exists \fundef[\gamma]{[0,1]}{X}:\begin{cases}
  \text{$\gamma$ continua} \\
  \gamma(0)=x_0 \\
  \gamma(1)=x_1
 \end{cases}\]
\end{defn}

\begin{prop}
 $X$ connesso per archi $\implies X$ connesso
\end{prop}

\begin{proof}
 Supponiamo per assurdo $X=A_0\cup A_1$ aperti non vuoti disgiunti.
 Siano $x_0\in A_0$ e $x_1\in A_1$,
 e $\gamma$ un arco che li connette.
 L'intervallo $[0,1]$ è connesso quindi $\gamma([0,1])$ è connesso per continuità di $\gamma$.
 Ma posso anche scriverlo come
 $\gamma([0,1])=\big(A_0\cap \gamma([0,1])\big) \cup \big(A_1\cap(\gamma([0,1])\big)$
 che è sconnesso \absurd.
\end{proof}

\begin{es}
 Non è vero il viceversa! Si prendano gli insiemi in $\R^2$:
 \[Y_1\is\setdef[(0,y)]{-1\le y \le 1}\qquad
 Y_2\is\Setdef[\left(x,\sin\frac{1}{x}\right)]{x\in \mathbb R}\]
 $Y=Y_1\cup Y_2$ è connesso ma non per archi. 
 \marginpar{dimostrarlo}
\end{es}

\begin{prop}
 Sia $A\subseteq (\mathbb R^n, \tau _E)$ aperto.
 $A$ connesso $\iff A$ connesso per archi
\end{prop}
\begin{proof}
 La freccia verso sinistra è ovvia. Dimostriamo l'altra.
 Introduciamo la seguente relazione di equivalenza:
 \[x\sim y \means \exists \fundef[\gamma]{[0,1]}{X},\quad \gamma(0)=x,\quad \gamma(1)=y\text{ continua.}\]
 (È facile verificare che è una relazione di equivalenza)
 
 %fatti dare la dimostrazione a modino
 Andiamo a mostrare che, fissato $x\in A$, la sua classe di equivalenza $\left [x\right ]$ è un insieme sia aperto che chiuso.
 
 Sia $y\in A,\quad y\sim x$. Per ipotesi esiste una palla aperta $\ball[r]{y}\subseteq A$. Ogni $z\in \ball[r]{y}$ può essere collegata a $y$ tramite arco radiale.
 
 Dunque $\left[x\right]$ è aperto (considero $\ball[r/2]{y}\subseteq [x]$)
 
 Per mostrare che l'insieme è chiuso basta vedere che il complementare è aperto. Ma se un punto $y$ non è collegato a $x$, allora quelli nella palletta intorno a lui non possono essere collegati a $x$:
 altrimenti potrei collegare $y$ a $x$ tramite il punto nella palletta (come sopra).
\end{proof}
\begin{defn}[Componenti connesse per archi]
 Su uno spazio topologico $X$ arbitrario, le classi di equivalenza per la relazione sopra definita sono dette componenti connesse per archi di X.
\end{defn}
\begin{defn}[Compattezza per ricoprimenti]$\\$
 $X$ compatto $\means \forall$ ricoprimento di aperti di $X$ $\exists$ un sottoricoprimento finito.
\end{defn}
\begin{oss}
 Quando parliamo di compattezza, ci riferiamo sempre a spazi $T_2$
\end{oss}
\begin{defn}[Compattezza per successioni]$\\$
 $X$ compatto per successioni $\means \forall \fundef[a]{\N}{X} \, \exists$ sottosuccessione $a_{n_j}\convarrow x_0\in X$
\end{defn}
\begin{prop}
 $X$ compatto, $Y\subseteq X$,  $Y$ compatto $\iff Y$ chiuso
\end{prop}
\begin{proof}
Mostriamo le due implicazioni:
\begin{description}
	\item[\proofleftarrow]
		Sia $ \{ A_j  \}$ famiglia di aperti di $X$ tale che $Y\subseteq \union_{j\in J}A_j$.

		Allora $ \{ A_j  \} \cup (X\setminus Y)$ è ricoprimento di tutto $X$.
		Per compattezza posso estrarre ricoprimento finito che ricopre $X$, e quindi $Y$. (Si osservi che non abbiamo usato l'essere $T_2$).
 \item[\proofrightarrow]
		Voglio mostrare che $X\setminus Y$ è aperto sfruttando la definizione.
		Fisso $y\in X\setminus Y$. Per $T_2$, $\forall x\in Y\, \exists U_x, W_x$ aperti disgiunti, $x\in U_x$, $y\in W_x.$
		 
		Si osservi che $\union_{x\in Y}U_x$ è ricoprimento di $Y$. Per compattezza, posso estrarre un sottoricoprimento finito $U_1,\dots, U_n$.
		Allora considero i rispettivi insiemi aperti $W_1,\dots, W_n$ che assieme agli $U_i$ separavano i punti.

		Ma $W\is \inters_{i=1}^nW_i$ è aperto non vuoto, che contiene solo punti di $X\setminus Y$ e tale che $y\in W$, da cui la tesi. \qedhere
	\end{description}
\end{proof}
 
\begin{prop}
 $X$ $T_2$, 2-numerabile, $ X$ compatto $\iff X$ compatto per successioni
\end{prop}
 \begin{lemma}
  $X$ compatto, $Y\subseteq X, Y$ infinito. Allora esistono punti di accumulazione per $Y$ in $X$.
 \end{lemma}
 \begin{proof}
  Supponiamo per assurdo che tutti gli $x\in X$ non siano di accumulazione per $Y$. Abbiamo due casi:
	\begin{itemize}
	 \item $x\in X\setminus Y\implies \exists$ intorno $U_x$ tale che $U_x\cap Y = \nullset$
	 \item $x\in Y\implies \exists$ intorno aperto $U_x$ di $x$ tale che $U_x\cap Y = \{ x \}$
	\end{itemize}
  Considero un ricoprimento di $X$ $\{ U_x \}$. Per compattezza $\exists U_1,\dots, U_n$ sottoricoprimento finito.
  
  Allora $Y\subseteq \union_{i=1}^nU_i\implies Y\subseteq \left \{ x_1,\dots,x_n\right \}$ che è finito, assurdo.
 \end{proof}







