% autore: Costa

\begin{teo}
 $\fundef[f]{X}{Y}$ continua, $X$ connesso $\implies f(X)$ connesso 
\end{teo}
\begin{proof}
  Non è restrittivo supporre $f(X)=Y$: infatti $\fundef[f]{X}{Y}$ continua $\Leftrightarrow \fundef[f]{X}{f(X)}$ continua.
 Per assurdo supponiamo $Y = B_1 \cup B_2, B_1,B_2$ aperti disgiunti non vuoti.
 $A_1\is f^{-1}(B_1), A_2\is f^{-1}(B_2)$ aperti per continuità di $f$.
 Inoltre sono non vuoti, disgiunti e $X=A_1\cup A_2$, assurdo.
\end{proof}
\begin{oss}
 La connessione è invariante per omeomorfismo. La connessione non passa ai sottospazi. 
\end{oss}
\begin{defn}[Arco]
 Un arco in $X$ è una applicazione continua $\fundef[\gamma]{[0,1]}{X}$
\end{defn}
\begin{defn}[Connessione per archi]$\\$
 $X$ connesso per archi $\means \forall x_0, x_1 \in X \,\exists \fundef[\gamma]{[0,1]}{X},\quad \gamma(0)=x_0,\quad \gamma(1)=x_1$
\end{defn}
\begin{prop}
 $X$ connesso per archi $\implies X$ connesso
\end{prop}
\begin{proof}
 Supponiamo per assurdo $X=A_0\cup A_1$ aperti non vuoti disgiunti.
 $\exists \fundef[\gamma]{[0,1]}{X},\quad \gamma(0)=x_0,\quad \gamma(1)=x_1\\$
 $[0,1]$ connesso $\implies \gamma([0,1])$ connesso per continuità di $\gamma\\$
 $\gamma([0,1])=(A_0\cap \gamma([0,1])) \cup (A_1\cap(\gamma([0,1]))$ che è sconnesso. Assurdo
\end{proof}
\begin{es}
 Non è vero il viceversa!$\\$
 $Y_1\is\setdef[(0,y)]{-1\le y \le x}\\$
 $Y_2\is\Setdef[\left(x,\sin\left(\frac{1}{x}\right)\right)]{x\in \mathbb R}\\$
 $Y=Y_1\cup Y_2$ è il controesempio 
 %scrivere la dim
\end{es}
\begin{prop}
 Sia $A\subseteq (\mathbb R^n, \tau _E)$ aperto.
 $A$ connesso $\iff A$ connesso per archi
\end{prop}
\begin{proof}
 La freccia verso sinistra è ovvia. Dimostriamo l'altra.
 Introduciamo la seguente relazione di equivalenza:$\\$
 $x\sim y \means \exists \fundef[\gamma]{[0,1]}{X},\quad \gamma(0)=X,\quad \gamma(1)=y$ continua.$\\$ 
 (È facile verificare che è una relazione di equivalenza)
 
 %fatti dare la dimostrazione a modino
 Andiamo a mostrare che, fissato $x\in A$, la sua classe di equivalenza $\left [x\right ]$ è un insieme sia aperto che chiuso.
 
 Sia $y\in A,\quad y\sim x$. Per ipotesi esiste una palla aperta $\ball[r]{y}\subseteq A$. Ogni $z\in \ball[r]{y}$ può essere collegata a $y$ tramite arco radiale.
 
 Dunque $\left[x\right]$ è aperto (considero $\ball[r/2]{y}\subseteq [x]$)
 
 Per mostrare che l'insieme è chiuso basta vedere che il complementare è aperto. Ma se un punto $y$ non è collegato a $x$, allora quelli nella palletta intorno a lui non possono essere collegati a $x$:
 altrimenti potrei collegare $y$ a $x$ tramite il punto nella palletta (come sopra).
\end{proof}
\begin{defn}[Componenti connesse per archi]
 Su uno spazio topologico $X$ arbitrario, le classi di equivalenza per la relazione sopra definita sono dette componenti connesse per archi di X.
\end{defn}
\begin{defn}[Compattezza per ricoprimenti]$\\$
 $X$ compatto $\means \forall$ ricoprimento di aperti di $X$ $\exists$ un sottoricoprimento finito.
\end{defn}
\begin{oss}
 Quando parliamo di compattezza, ci riferiamo sempre a spazi T2
\end{oss}
\begin{defn}[Compattezza per successioni]$\\$
 $X$ compatto per successioni $\means \forall \fundef[a]{\N}{X} \, \exists$ sottosuccessione $a_{n_j}\convarrow x_0\in X$
\end{defn}
\begin{prop}
 $X$ compatto, $Y\subseteq X,\quad Y$ compatto $\iff Y$ chiuso
\end{prop}
\begin{proof}
 $\\\proofleftarrow$\\
 Sia $ \{ A_j  \}$ famigla di aperti di $X$ tale che $Y\subseteq \union_{j\in J}A_j$
 
 Allora $ \{ A_j  \} \cup (X\setminus Y)$ è ricoprimento di tutto $X$.
 Per compattezza posso estrarre ricoprimento finito che ricopre $X$, e quindi $Y$. (Si osservi che non abbiamo usato l'essere T2).
 $\\\proofrightarrow$\\
 Voglio mostrare che $X\setminus Y$ è aperto sfruttando la definizione.
 Fisso $y\in X\setminus Y$.Per T2, $\forall x\in Y\quad \exists U_x, W_x$ aperti disgiunti, $x\in U_x$, $y\in W_x.\\$ 
 Si osservi che $\union_{x\in Y}U_x$ è ricoprimento di $Y$. Per compattezza, posso estrarre un sottoricoprimento finito $U_1,\dots U_n$
 Allora considero i rispettivi insiemi aperti $W_1,\dots W_n$ che assieme agli $U_i$ separavano i punti.
  
 Ma $W\is \inters_{i=1}^nW_i$ è aperto non vuoto, che contiene solo punti di $X\setminus Y$ e tale che $y\in W$, da cui la tesi  
 \end{proof}
\begin{prop}
 $X$ T2, 2-numerabile, $ X$ compatto $\iff X$ compatto per successioni
\end{prop}
 \begin{lemma}
  $X$ compatto, $Y\subseteq X, Y$ infinito. Allora esistono punti di accumulazione per $Y$ in $X$.
 \end{lemma}
 \begin{proof}
  Supponiamo per assurdo che tutti gli $x\in X$ non siano di accumulazione per $Y$. Abbiamo due casi:
	\begin{itemize}
	 \item $x\in X\setminus Y\implies \exists$ intorno $U_x$ tale che $U_x\cap Y = \nullset$
	 \item $x\in Y\implies \exists$ intorno aperto $U_x$ di $x$ tale che $U_x\cap Y = \{ x \}$
	\end{itemize}
  Considero un ricoprimento di $X$ $\{ U_x \}$. Per compattezza $\exists U_1,\dots U_n$ sottoricoprimento finito.
  
  Allora $Y\subseteq \union_{i=1}^nU_i\implies Y\subseteq \left \{ x_1,\dots,x_n\right \}$ che è finito, assurdo.
 \end{proof}







