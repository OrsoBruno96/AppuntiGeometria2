%Federico

\titlet{Fine della lezione sulla trasversalità}

Sia $\fundef[f]{X}{Y}$ e $A \in Y$ tale che $f \pitchfork A$, le varietà sono tutte orientate, allora $Z = f^{-1}(A)$ è orientabile, nel seguito daremo una procedura per orientarla.
Sia $A = \set{y_0}$, $y_0$ è un valore regolare per $f$. Localmente ho la situazione linearizzata del teorema della funzione implicita surgettiva.

\marginpar{Aggiungere disegno incomprensibile 1}

Metto come orientazione sulla linea verticale l'unica che mi da l'orientazione dell'ambiente.

Sia $i$ una immersione cioè $X \hookrightarrow Y \supset A i^{-1} (A) = X \pitchfork A$. Si ha anche $T_{x}X + T_{x}A = T_{x}Y$ e $T_{x}(X \cup A) = T_{x}X \cup T_{x}Y$. Ho che $\mathbb{R}^{N}$
\newcommand*\base{\mathcal B}
è lo spazio ambiente e vale $V+W = \mathbb{R}^{N}$. Sia $\base$ una base dell'intersezione $V \cup W $. Gli insiemi $(\base, \base_{V})$ e  $(\base, \base_{W})$ sono rispettivamente basi di V e W
e producono l'orientazione per queste sottovarietà. L'insieme $(\base, \base_V, \base_W)$ è base di $\mathbb{R}^{N}$ e produce quindi una orientazione di $\mathbb{R}^{N}$.
L'orientazione della base $\base$ di $V \cup W$ è quella per cui la procedura specificata produce una base per $\mathbb{R}^{N}$ che ha la stessa orientazione data in partenza.

\begin{oss}
 Se scambio l'ordine delle varietà ottengo $A \pitchfork X$ e $X \pitchfork A$. Si può verificare che vale la relazione: $ A \pitchfork X = (-1)^{\codim(X) \cdot \codim(Y)} X \pitchfork A$
 Dove il segno significa semplicemente l'orientazione (stessa orientazione o orientazione opposta).
\end{oss}

Se è $\fundef[f]{X}{Y}$ con $A \subseteq Y$ generico ci si riconduce al caso in cui $A = \set{y_{0}}$ e a questo punto si applica il teorema della funzione implicita surgettiva.

\marginpar{Aggiungere disegno incoprensibile 2}

\begin{defn}[Trasversalità di funzioni]
Siano date $\fundef[f_{1}]{X_{1}}{Y}$ e $\fundef[f_{2}]{X_{2}}{Y}$. Voglio dare un senso alla scrittura $f_{1} \pitchfork f_{2}$.
Definiamo $X_{1} \times X_{2} \rightarrow Y \times Y$ e $(x_{1}, x_{2}) \mapsto (f(x_{1}), f(x_{2}))$. Considero l'insieme diagonale $\Delta = \setdef[(y_{1}, y_{2}) \in Y \times Y]{y_{1} = y_{2}}$
Diciamo $f_{1} \pitchfork f_{2} \means f_{1} \times f_{2} \pitchfork \Delta$.
\end{defn}

\begin{teo}
Data $f_{2} \rightarrow Y$, definiamo $\pitchfork(A, f_{2}) \is \setdef[\fundef{X_{1}}{Y}] {f_{1} \pitchfork f_{2}}$ l'insieme così definito è un aperto denso di $\mathcal{E}(x_{1}, Y)$.
\end{teo}

Abbiamo visto nelle lezioni precedenti:

\begin{itemize}
 \item Non esistono retrazioni $\fundef[r]{X}{\boundary X}$ diffeomorfismi, da cui segue il teorema del punto fisso di Brouer
 \item Vale il teorema di embedding cioè ogni varietà $X^{n}$ di dimensione $n$ si può immergere con un embedding in $\mathbb{R}^{2n+1}$ se non richiedo iniettività della funzione 
 posso comunque trovare un'immersione in $\mathbb{R}^{2n}$. 
\end{itemize}

\begin{teo} [Di Whithney versione difficile] 
$\forall X^{n}$ trovo una immersione $X^{n} \hookrightarrow \mathbb{R}^{2n}$.
\end{teo}

\begin{oss}
 Per dimostrarlo non basta usare considerazioni di trasversalità. Devo definire una procedura per eliminare tutte le autointersezioni della varietà immersa.
\end{oss}

\begin{oss} %molto a caso
 In tutte le dimensioni ($n \neq 4$) ogni varietà omeomorfa a $\mathbb{R}^{m}$ è diffeomorfa a  $\mathbb{R}^{m}$. In $\mathbb{R}^{4}$ questo non vale e c'è un continuo di varietà 
 non diffeomorfe tra loro.
\end{oss}

\titlet{Teoria del grado}
$\fundef[f]{X^{n}}{Y^{n}}$ con $X^{n}$ compatta e chiusa e $Y^{n}$ connessa. Voglio definire il grado della funzione $f$, dimostreremo che è un invariate per cobordismo.

\begin{defn}[Versione orientata]

\end{defn}

\begin{defn}[Versione non orientata]
Procederemo seguendo una serie di semplici passi:
\begin{itemize}
 \item Fisso $y_{0} \in Y$.
 \item Prendo $g$ vicina a $f$ nel senso della topologia definita su $\mathcal{E}(X, Y)$ tale che $g \pitchfork {y_{0}}$ si vede che $g$ è 
 omomorfa a $f$ dai teoremi dimostrati nelle lezioni precedenti sulla trasversalità.
 \item $g^{-1}(y_{0}) = {x_{1}, x_{2}, \dots, x_{k}}$ è un insieme di cardinalità finita.
 \item Definisco $\grad_{2}(f) = k \mod 2$
\end{itemize}
\end{defn}

Dunque vediamo che il grado di una funzione tra varietà non orientate è una funzione $ \mathcal{E}(X, Y) \funarrow \quoset\Z{2\Z}$.

\begin{teo}[Buona definizione del grado]
 la quantità così associata a $f$: $\grad_{2}(f)$ è ben definita cioè non dipende né dalla scelta di $y_{0}$ né dalla scelta di $g$. 
\end{teo}

\begin{teo}
 Se $\fundef[f_{1}]{X_{0}}{Y}$ e $\fundef[f_{2}]{X_{0}}{Y}$ sono cobordanti allora $\grad_{2}(f_{1}) = \grad_{2}(f_{0})$. In particolare questo vale quando sono omotope,
 infatti l'omotopia è un particolare cobordismo.
\end{teo}

\begin{proof}
 Dimostreremo che a $y_{0}$ fissato il grado non dipende dalla scelta della $g$. Siano $g_{1}$ e $g_{2}$ tali che $g_1 \pitchfork y_0$ e $g_{2} \pitchfork y_{0}$ allora la $g_1$ e la $g_2$ 
 sono tra loro omotope.
 
\marginpar{Inserisco il disegnino cancer di g1 e g2 che sono cobordanti e mostro eplicitamente dove stanno gli insiemi di cui sto parlando}
 
 Uso il teorema di alcune lezioni precedenti per cui trasversalità coimplica cobordismo, dunque $g_1 \pitchfork y_0$ implica $g_1$ omotopa a $f_{1}$ e similmente
  $g_2 \pitchfork y_0$ implica $g_2$ omotopa a $f_{2}$. Dunque per ipotesi valeva $f_{1} \sim f_{2}$ dunque vale anche  $g_{1} \sim g_{2}$, poiché il cobordismo è una relazione di equivalenza.
  Ricorco che è $\fundef[g_{1}]{X_{1}}{Y \supset Y_{0}}$ e $\fundef[g_{1}]{X_{2}}{Y \supset Y_{0}}$ inoltre le controimmagini le chiamo $Z_{1} = g_{1}^{-1}(y_{0})$ e $Z_{2} = g_{2}^{-1}(y_{0})$
  si vede che i due insiemi finiti $Z_{1}$ e $Z_{2}$ sono cobordanti dunque la loro cardinalità ha la stessa parità.
 
 %non so se funziona quanetsa notazione per primare le cose
Ora devo eliminare l'arbitrarietà del punto $y_{0}$. Prendo due punti $x$ e $y$ faremo vedere nella prossima lezione che utilizzando il fatto che $Y$ è connessa per costruire una isotopia
$\fundef[H]{Y \times [0,1]}{Y}$ tale che posto $H_{t} = H_{Y \times {t}}$ si ha che $H_{0} = \id$. $H_{t}$ è un diffeomorfismo di $Y$ $\forall t \in [0,1]$  e vale $H_{0} = x$ e $H_{1} = y$.
Comunque prendo due punti esiste un diffeomorfismo che manda un punto nell'altro, in questo senso nessun punto è privilegiato. Questo significa che una varietà è omogenea, cioè a meno di 
diffeomorfismi ogni punto è equivalente.
\end{proof}



