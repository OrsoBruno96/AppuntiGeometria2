%questo documento è in ogni sua parte opera originale del sottoscritto
%Alberto Bordin

\begin{verse}
C'era una volta un nero serpentello\\
sul verde schermo di un telefonino\\
e mosso da veloce polpastrello\\
guatava i frutti del suo bel giardino.

Non viveva chiuso da alcun cancello:\\
nessun lato fermava il suo cammino,\\
e se da questo spariva, da quello\\
andava verso il prossimo spuntino.

Di vivere in un ritto piano il moro\\
serpente credeva, poiché era dura\\
immaginare che se io accosto

dello schermo ogni lato a quello opposto\\
di quel mondo si svela la natura:\\
non ha forma di un piano, ma di un toro.
\footnote{Il sonetto è lasciato per esercizio ma non è richiesto nel corso.}
\end{verse}

\titlet{Definizioni di base}

\begin{defn}[$n$-euclideo]
$X$ s.t. è detto \emph{$n$-euclideo} se $\forall x \in X$ esiste un intorno aperto $U$ di $x$ e un omeomorfismo $\fundef UW \subseteq (\R^n, \tau _E)$ con $W$ aperto.
\end{defn}

\begin{defn}[varietà] $\\$
Una \emph{$n$-varietà topologica} è uno s.t. $n$-localmente euclideo, $T_2$ e 2-numerabile
\end{defn}

\begin{defn} $\\$
$(U,f)$ è detta \emph{carta locale} di $X$ intorno a $x$ \\
$\fundef [f^{-1}] WU$ è detta \emph{parametrizzazione locale} di $X$ intorno a $x$ \\
Un'unione di tutte le carte locali è detto \emph{atlante (completo)}
\end{defn}

\begin{prop}
localmente euclideo $\notimplies$ 2-numerabile
\end{prop}
\begin{proof} \marginpar{creare una macro per i controesempi}
Si consideri $X = (\R^n, \tau _E) \times (\R, \tau _D)$.
Tale spazio è localmente $n$-euclideo, ma non 2-numerabile.

\begin{figure}
	\centering
	\input{figura1.pdf_tex}
	\caption{$(\R,\tau_E)\times(\R,\tau_D)$ è localmente 1-euclideo ma non 2-numerabile}
\end{figure}

Un controesempio minimale è $(X, \tau _D)$ con $X$ non numerabile, questi è 0-localmente euclideo, ma non 2-numerabile.
\end{proof}

\begin{prop}
localmente euclideo $\notimplies T_2$
\end{prop}
\begin{proof}
Prese 2 copie di $(\R^n, \tau _E)$ sia $(x,0) \in (\R^n_0, \tau _0)$  e $(x,1) \in (\R^n_1, \tau _1)$, e sia $X = (\R^n_0 \cup \R^n_1, \tau _0 \cup \tau _1)$. Si consideri la relazione di equivalenza $(x,0)\sim(x,1)$ se $x>0$ mentre gli altri elementi fanno classe a sé. Si consideri l'insieme quoziente $\quoset X\sim$ e la funzione $\fundef{(-\varepsilon,\varepsilon)}{\quoset X\sim}$ che manda $t \mapsto [(t,0)]$. Si verifica che $f$ è un omeomorfismo (esercizio) che rende $\quoset X\sim$ localmente 1-euclideo. Tuttavia $\quoset X\sim$ non è $T_2$ in quanto non è possibile separare i punti $[(0,0)]$ e $[(0,1)].$
\begin{figure}
	\centering
	\input{figura2.pdf_tex}
	\caption{$\R\djcup\R$ quozientato in modo che l'unione non sia disgiunta sulla semiretta positiva; i punti nell'origine sono disgiunti ma non separati}
\end{figure}
\end{proof}

\titlet{Esempi di varietà}

\begin{itemize}
\item $(\R^n, \tau_E)$
\item Ogni aperto di $\R^n$
\item Le $n$-sfere: \, $S^n \subseteq (\R^{n+1}, \tau _E) \qquad S^n=\setdef[x\in \R^{n+1}]{x^2 _1 + x^2 _2 + \dots + x^2 _{n+1} = 1}$

\begin{oss}
$S^n$ è compatto, $T_2$ e 2-numerabile

Dobbiamo esibire un atlante $\mathcal{A}$.

Siano
\, \( N =  \begin{pmatrix}
0\\
0\\
\vdots\\
1\end{pmatrix} \;$e$ \;\; S = \begin{pmatrix}
0\\
0\\
\vdots\\
-1\end{pmatrix}$
il polo Nord e il polo Sud.

$\mathcal{A} = \set{\, (S^n \setminus \set N, f_N),\, (S^n \setminus \set S, f_S)\,}$ \\
Dove $f_N$ e $f_S$ sono le \emph{proiezioni stereografiche} di centro N ed S così definite:
\begin{align*}
f_N:(S^n \setminus \set N) &\funarrow \R^n \\
%\fundef[f_N]{(S^n \setminus \set N)}{\R^n} \\
\begin{pmatrix}
x_1\\
\vdots\\
x_n\\
x_{n+1}\end{pmatrix} &\mapsto \begin{pmatrix}
\dfrac{x_1}{1-x_{n+1}}\\
\vdots\\
\dfrac{x_n}{1-x_{n+1}}\\
0\end{pmatrix}
\end{align*}

\begin{figure}
	\centering
	\input{figura3.pdf_tex}
	\caption{la proiezione stereografica $f_N$ di centro $N$}
\end{figure}

\end{oss}
\end{itemize}